\documentclass[10pt,a4paper]{report}
\usepackage[margin=3cm]{geometry}
\usepackage{fancyhdr}
\usepackage{amssymb}
\usepackage{amsthm}
\usepackage{amsfonts}
\usepackage{bbold}
\usepackage[utf8]{inputenc}
\usepackage{parskip}
\usepackage{mathtools}
\usepackage[english]{babel}
\usepackage{tikz-cd}
\usepackage{bbm}
\usepackage{stackrel}
\usepackage[linktoc=all, colorlinks=true,linkcolor=blue,citecolor=blue,urlcolor=blue]{hyperref}
%\usepackage[shortlabels]{enumitem}
%\usepackage[dvipsnames]{xcolor}

\bibliographystyle{alpha}

%\hypersetup{
%  linktoc=section,      % Only the number in the ToC is a link
%  colorlinks=true,
%  linkcolor=Blue,        % or any color you want
%  citecolor=Green,
%}

%\renewcommand\thesubsection{\thesection.\alph{subsection}}

\newtheorem{theorem}{Theorem}[section] % part
\newtheorem{lemma}{Lemma}[section] % part
\newtheorem{definition}{Definition}[section] % part

\newcommand{\D}[2]{#1^\prime(#2)}
\newcommand{\norm}[1]{\left\lVert#1\right\rVert}
\newcommand{\abs}[1]{\left|#1\right|}
\newcommand{\scalar}[2]{\langle#1, #2\rangle}

\DeclareMathOperator{\Ima}{Im}
\DeclareMathOperator{\Gal}{Gal}
\DeclareMathOperator{\colim}{colim}
\DeclareMathOperator{\Hom}{Hom}
\DeclareMathOperator{\Set}{Set}
\DeclareMathOperator{\Frm}{Frm}
\DeclareMathOperator{\Top}{Top}
\DeclareMathOperator{\Bool}{Bool}
\DeclareMathOperator{\CAlg}{CAlg_R}
\DeclareMathOperator{\CAlgZ}{CAlg_\mathbb{Z}}
\DeclareMathOperator{\CAlgk}{CAlg_k}
\DeclareMathOperator{\Ring}{Ring}
\DeclareMathOperator{\Aff}{Aff}
\DeclareMathOperator{\Fun}{Fun}
\DeclareMathOperator{\Nat}{Nat}
\DeclareMathOperator{\Ob}{Ob}
\DeclareMathOperator{\Mor}{Mor}
\DeclareMathOperator{\kTop}{kTop}
\DeclareMathOperator{\kHaus}{kHaus}
\DeclareMathOperator{\Haus}{Haus}
\DeclareMathOperator{\ev}{ev}
\DeclareMathOperator{\Lift}{Lift}
\DeclareMathOperator{\kAlg}{kAlg}
\DeclareMathOperator{\GSet}{G-Set}
\DeclareMathOperator{\GFSet}{G-FinSet}
\DeclareMathOperator{\ket}{k_{\text{ét}}}
\DeclareMathOperator{\op}{op}
\DeclareMathOperator{\Ult}{Ult}
\DeclareMathOperator{\clop}{Clop}
\DeclareMathOperator{\FinSet}{FinSet}
\DeclareMathOperator{\prim}{prime}
\DeclareMathOperator{\Spat}{SpatLoc}
\DeclareMathOperator{\Sob}{SobTop}
\DeclareMathOperator{\Spatt}{Spat}
\DeclareMathOperator{\Sobb}{Sob}
\DeclareMathOperator{\DLat}{DLat}
\DeclareMathOperator{\Loc}{Loc}
\DeclareMathOperator{\Coh}{Coh}
\DeclareMathOperator{\CohLoc	}{CohLoc}
\DeclareMathOperator{\CohTop}{CohTop}
\DeclareMathOperator{\Stone}{Stone}
\DeclareMathOperator{\pt}{pt}
\DeclareMathOperator{\Idl}{Idl}
\DeclareMathOperator{\Spec}{Spec}
\DeclareMathOperator{\Ind}{Ind}
\DeclareMathOperator{\Pro}{Pro}
\DeclareMathOperator{\Lat}{Lat}
\DeclareMathOperator{\charr}{char}
\DeclareMathOperator{\const}{const}
\DeclareMathOperator{\Boolf}{FinBool}
\DeclareMathOperator{\id}{id}
\DeclareMathOperator{\atom}{atom}
\DeclareMathOperator{\primIdl}{primeIdl}
\DeclareMathOperator{\FinStone}{FinStone}
\DeclareMathOperator{\Aut}{Aut}
\DeclareMathOperator{\BG}{BG}
\DeclareMathOperator{\Grp}{Grp}







\def\HomA{\ensuremath\mathcal{A}}
\def\HomB{\ensuremath\mathcal{B}}
\def\HomC{\ensuremath\mathcal{C}}
\def\HomD{\ensuremath\mathcal{D}}

\def\concA{\ensuremath(\mathcal{A},U)}
\def\concB{\ensuremath(\mathcal{B},V)}


\def\HomFrm{\ensuremath\Hom_\Frm}
\def\HomTop{\ensuremath\Hom_\Top}
\def\t{\ensuremath\tilde}

\DeclarePairedDelimiter\ceil{\lceil}{\rceil}
\DeclarePairedDelimiter\floor{\lfloor}{\rfloor}
\renewcommand\qedsymbol{$\blacksquare$}
% https://q.uiver.app/#q=WzAsMyxbMCwwLCJcXG1hdGhjYWx7QX1ee29wfSJdLFsyLDAsIlxcbWF0aGNhbHtCfSJdLFsxLDAsIlxcYm90Il0sWzAsMSwiVCIsMCx7ImN1cnZlIjotMn1dLFsxLDAsIlMiLDAseyJjdXJ2ZSI6LTJ9XV0=
\title{\textsc{An overview of \\ Concrete Dualities}
\\[3em]
{\Large {Bachelor Thesis in Mathematics}}\\[2em]
{\normalsize Advisor: Dr. Georg Lehner\\ Second Examiner: Prof. Dr. Holger Reich}}
\author{
    Kasra Sammak\\
    % {\small (Matr. 5563286)}
}
\date{2025 Winter Semester}

\begin{document}
\maketitle

\newpage

\tableofcontents
%\setcounter{page}{0}
\newpage
%\setcounter{page}{1}

%\pagestyle{fancy}
%\fancyhf{}
%\rhead{Kasra Sammak}
%\lhead{Concrete Dualities}
\rfoot{\thepage}
\chapter{Introduction}

Concrete dualities appear throughout mathematics in forms that seem unrelated at first glance: between topological spaces and certain classes of lattices, between commutative rings and sets with a group action, between rings and affine schemes, and so on. A particularly accessible example is the classical self-duality of finite-dimensional vector spaces via the dual space construction. A common feature of these dualities is that they arise from a dual adjunction in which a single object carries compatible structure on both sides. Such an object — often called \emph{schizophrenic} or \emph{dualizing} — induces a particularly well defined dual adjunction and by proxy an equivalence between the associated categories of structured sets. The guiding theme of this thesis is that many familiar dualities can be understood uniformly through this mechanism.

The purpose of the present work is expository rather than exploratory. We do not aim to uncover new dualities or to analyze the deeper theory within each subfield. Instead, the goal is to make the general framework of concrete dualities transparent, and to demonstrate that several classical examples indeed fit this structure. Although these examples are well known in the literature, the technical ingredients that ensure they conform to the general mechanism are often treated as routine or left to the reader. Here we attempt to make these points explicit—or at least visible—even when a complete treatment would lead too far beyond the intended scope.

This choice comes with trade-offs. Not every construction can be unpacked to completion, and we occasionally move at a higher level of abstraction to preserve a coherent narrative. At other points, we allow ourselves brief detours—when the situation is both feasible and illuminating, for instance when the functors restrict to equivalences and understanding why sheds conceptual light on the framework. These detours are secondary to our main objective but hopefully serve to build conceptual familiarity with the framework.

By the end of the thesis, the reader should be able to recognize the mechanism behind a concrete duality: how a dual adjunction together with a schizophrenic object organizes structure and induces a well-behaved dual adjunction that underlies classical dualities across mathematics. Even without mastering every detail of the individual examples, seeing why they genuinely instantiate this mechanism constitutes the intended payoff.
\chapter{Preliminaries}

Throughout this thesis we will be using the language of category theory, for which a mild introduction is in order. We plan to only define the terms that are particularly relevant to the thesis in this section, so as to not distract the reader with basic definitions.

To that degree, we assume basic knowledge of category theory, such as the definitions of categories, (full/faithful) functors, natural transformations, (cofiltered/codirected) limits, (filtered/directed) colimits, initial object, terminal object, adjunctions, etc, and well known theorems which can be found in any introductory text of category theory, such as \emph{MacLane's Categories for the Working Mathematician} \cite{maclane:71}.  We may choose to define certain terms as necessary as they relate to our discussion. 

This thesis will be divided into two parts. In the first part we describe the general framework for arbitrary concrete categories. We will be answering the questions of what a mathematical duality is, what makes it concrete, and also what is the minimal datum that ascertains its existence. 

In the second part we should like to describe some examples. As these examples span mathematics itself, we shall assume basic knowledge in each subfield to which our respective example pertains. Before each example we will make more explicit what is assumed, as well references for background and/or further reading. One may expect general knowledge of topology, commutative algebra, group theory, lattice theory, and Galois theory to be assumed.


\section{Concrete Categories}

In the course of this thesis we will be considering a certain type of category, called a concrete category. We do not assume knowledge about this on behalf of the reader, therefore we would like to introduce some definitions  of the basic kinds of objects and constructions we will be working with, as they will be quite central to this discussion.

We source these definitions from \emph{Adámek, Herrlich, and Strecker's Abstract and Concrete Categories: the Joy of Cats} \cite{adamek1990abstract}.


Before all else, we should define a concrete category:
\\
\begin{definition}[Concrete Category]
	Let $\mathcal{X}$ be a category. A \textbf{concrete category} over $\mathcal{X}$ is a pair $(\mathcal{A}, U)$ where $\mathcal{A}$ is a category and $U: \mathcal{A} \to \mathcal{X}$ is a faithful functor. Sometimes $U$ is called the \textbf{underlying functor} over $\mathcal{X}$ and $\mathcal{X}$ is called the \textbf{base category} for $(\mathcal{A},U)$. 
	
	A concrete category over $\Set$ is called a \textbf{construct}.
\end{definition}

In this thesis the only concrete categories we consider are constructs, so that when we use the term 'concrete category' we always mean concrete over $\Set$.

The underlying functor is also sometimes called the \emph{\textbf{forgetful functor}}.
\\
\begin{definition} [Concrete functor]
Let $(\mathcal{A}, U)$ and $(\mathcal{B}, V)$ be concrete categories over $\mathcal{X}$. A \textbf{concrete functor} from $(\mathcal{A}, U)$ to $(\mathcal{B},V)$ is a functor $F: \mathcal{A} \to \mathcal{B}$ such that $U = V \circ F$. 
\end{definition}
A concrete functor is necessarily faithful, since $V$ is faithful. In general, for any functors $F: \mathcal{A} \to \mathcal{B}$ and $G: \mathcal{B} \to \mathcal{C}$, if $G \circ F$ is faithful, then $F$ is. The proof is easy: consider elements  $f, g \in \HomA(A, A')$ such that $Ff = Fg$. Then $GFf = GFg$ and since $G\circ F$ is faithful, then $f = g$. 
\section{Types of Arrows}
\begin{definition}[Source and sink]
	 A \textbf{source} is a pair $(Y, (f_i))$ consisting of an object $Y$  of a category $\mathcal{C}$, and a family of morphisms $(Y \stackrel{f_i}{\to} X_i)_{i \in I}$ over a (possibly empty) class $I$. Equivalently it is a family of objects in the under category  $\mathcal{C}_{Y/}$. 
	 
	 For short we will use the notation $(Y \to X_i)_I$
	 
	 If $I$ is a finite index set $\{1, \cdots, n\}$, we call our source an \textbf{$n$-source}. 
	 
	 The dual concept to a source is called a \textbf{sink}. 
	 \\
\end{definition}
\
\begin{definition}
	A source $\mathcal{S} = (A \stackrel{f_i}{\to}A_i)_I$ is called a \textbf{mono-source} if it cancels from the left, i.e., if for any two parallel morphisms $B \stackrel[h]{g}{\rightrightarrows A}$ the equation $ \mathcal{S} \circ g = \mathcal{S} \circ h$, that is, if $f_i \circ g = f_i \circ h$ for all $i \in I$, implies $g = h$.
\end{definition}
For the following definitions, let $G: \mathcal{A} \to \mathcal{B}$ be a functor, and $B \in \Ob(\mathcal{B})$.

\

\begin{definition}[$G$-structured map]
		A \textbf{$G$-structured arrow with domain $B$} is a a pair $(f, A)$ consisting of an $\mathcal{A}$-object $A$ and a $\mathcal{B}$-morphism $f: B \to GA$. 
\end{definition}
\
\begin{definition}[$G$-structured lift]
	Let $(B \stackrel{\varphi_i}{\to} GA_i)_I$ be a $G$-structured source. If there exists an object $A$ and a map of morphisms  $(A \stackrel{f_i}{\to} A_i)_I$, for which there exists a map $GA \stackrel{h}{\to} B$ such that $Gf_i = \varphi_i \circ h$ for all $i \in I$, we call $A$ a \textbf{$G$-structured lift} of the source. 
	
	For notation we will interchangably refer to $A$ or the source $(A \to A_i)_I$ as the lift.
\end{definition}
\
\begin{definition}[Morphism of $G$-structured lifts]
	We call an  $\mathcal{A}$-morphism $A' \stackrel{\phi}{\to} A$ a \textbf{morphism of $G$-structured lifts} if there exists another lift $(A' \stackrel{f_i'}{\to} A_i)_I$ such that $GA' \to GA_i$ factors  through $GA \to GA_i$ for all $i \in I$. 
	
	That is, there exists a  morphism $GA' \stackrel{h'}{\to} B$ such that $h' = h \circ G\phi$ and $f_i' = f_i \circ \phi$. 
	
	
	\end{definition}
\begin{definition}
	The lift $A \in \Ob(\mathcal{A})$ of $(B \stackrel{\varphi_i}{\to} GA_i)_I$ is called a \textbf{$G$-initial lift} if  every $G$-structured lift $(A' \stackrel{f_i'}{\to} A_i)_I$ factors uniquely through $(A \stackrel{f_i}{\to} A_i)_I$.
\end{definition}
	
For the previous definitions we refer to the following commutative diagram for clarity:

\[\begin{tikzcd}
	{A'} && A && {} && {A_i} \\
	\\
	{GA'} && GA && B && {GA_i}
	\arrow["\phi", from=1-1, to=1-3]
	\arrow["{f_i'}", bend left, from=1-1, to=1-7]
	\arrow[from=1-1, to=3-1]
	\arrow["{f_i}", from=1-3, to=1-7]
	\arrow[from=1-3, to=3-3]
	\arrow[from=1-7, to=3-7]
	\arrow["{G\phi}", from=3-1, to=3-3]
	\arrow["{h'}", bend right, from=3-1, to=3-5]
	\arrow["h", from=3-3, to=3-5]
	\arrow["{Gf_i}", bend right, from=3-3, to=3-7]
	\arrow["{\varphi_i}", from=3-5, to=3-7].
\end{tikzcd}\]

\emph{Remark}. Often $h$ is the identity map, in which case our lift is called \textbf{strict}. However in practice we assume that $h$ is the identity unless stated otherwise.

\

Consider that here we mean initial in the sense of the initial or induced topology (in $(\Top, U)$ as  construct), however applied to arbitrary concrete category $(\mathcal{A}, U)$ over arbitrary category $\mathcal{X}$. 

That is, this lift is initial in the poset $(\Lift(B), \subseteq)$  of $G$-structured lifts of $B$ where the preorder is given by $A\subseteq A'$ if and only if $A'$ factors through $A$ as a lift, i.e. the morphisms are $A' \stackrel{\phi}{\to} A$ such that $(GA' \stackrel{h'}{\to}B) = (GA' \stackrel{G\phi}{\to}GA\stackrel{h}{\to}B) $. 

This is a very important concept throughout this paper, which deserves a remark about its intuition, which here should come from topology, where the initial topology is the limit topology, i.e., the coarsest topology on $GA$ making all $(A \stackrel{f_i}{\to}A_i)_I$ continuous. 

So for arbitrary category, we are looking for the weakest or initial $\mathcal{A}$-structure on $GA$ such that $(A \stackrel{f_i}{\to}A_i)_I$ are $\mathcal{A}$-morphisms, which ensures that for any $\mathcal{A}$-structure on $GA'$ such that $(A' \stackrel{f_i'}{\to}A_i)_I$ are $\mathcal{A}$-morphisms which factor through the lift $(A\stackrel{f_i}{\to} A_i)_I$, that this factorization $(A' \stackrel{\phi}{\to}A)$ is unique.
 
Formally this is a limit  in $\mathcal{A}$ over $\mathcal{A}$-structures on $GA$ with the property that all $(A \stackrel{f_i}{\to}A_i)_I$ are $\mathcal{A}$-morphisms.

\
%
%\begin{definition}[cogenerator]
%A \textbf{cogenerator} of a category $\mathcal{C}$ is an object $c \in \mathcal{C}$ such that the $\Hom$-set functor $\HomC(-, c): \mathcal{C}^{\op} \to \Set$ is injective.
%
%That is, given two maps $e \stackrel[f_2]{f_1}{\rightrightarrows} d$, if the induced maps $\HomC(d, c) \stackrel[- \circ f_2]{- \circ f_1}{\rightrightarrows}  \HomC(e,c)$ are equal, then $f_1 = f_2$, i.e., for any $\phi: d \to c$ it holds that $\phi \circ f_1 = \phi \circ f_2 \Rightarrow f_1 = f_2$. 
%
%The dual concept is of a \textbf{generator}, which applies to $\HomC(c, -)$.
%\end{definition}
\section{Topological Categories}
In the following we  introduce topological categories. 

Let $(A, U)$ be a concrete category over $\mathcal{X}$. 
\begin{definition}[Topological functor]
A functor $\mathcal{A} \stackrel{G}{\to}\mathcal{B}$ is called \textbf{topological} if every $G$-structured source has a unique $G$-initial lift. 
 \end{definition}
 
 \begin{definition}[Topological category]
 A concrete category $(\mathcal{A}, U)$ over $\mathcal{X}$ is said to be a \textbf{topological category} if $U$ is a topological functor. 
 \end{definition}
 
 
 Replacing "source" with "mono-source" gives us the definition of a \emph{\textbf{mono-topological}} category.
 
 
 Notice here again that our intuition of such a category should come from topology, which is even reflected in its name, namely, from the concrete category $(\Top, U)$,  over which every source lifts initially, since the arbitrary intersection of topologies is a topology. That is, we are taking the intersection of all topologies on $A$ such that all $(UA \stackrel{f_i}{\to} UA_i)_I$ are continuous.
 
  That means there is a systematic way to choose open sets of $UA$. In other words,  a subset $S \subseteq UA$ lifts to an open set in $A$ if and only if $S$ equals the pre-image $Uf_i^{-1}(U(V))$ for some  $f_i \in (UA \stackrel{f_i}{\to} UA_i)_I$ and $V \in  \Omega(A_i)$, where $\Omega(X)$ for topological space $X$ denotes the set of open sets of $X$.. 
 

\section{Free Objects and the Free-Forget Adjunction}

In the course of this thesis we will normally consider concrete categories $(\mathcal{A},U)$ which contain a representing object $A_0$ such that $\HomA(A_0,A) \cong U(A)$. As such $A_0$ is a free object on one free generator, which we can conceptualize via the idea that morphisms in $\HomA(A_0,A)$ are uniquely determined by where they send the free generator to. 

However this set isomorphism is categorical in nature. We do not necessarily need to redefine what we mean by free generator for each new category, as we get the isomorphism by virtue of the adjunction. This free object is constructed using the the left adjoint functor $F: \Set \to \mathcal{A}$ to the forgetful functor $U$, which is called the \emph{\textbf{free functor}}. 

The left adjoint functor does not have to exist. For our discussion we only consider categories where the left adjoint object of $U$ at $\{pt\} \in \Set$ exists. For the purpose of explication let us assume that a left adjoint $F$ exists, that is, we have the object $A_0 = F(\{\text{pt}\})$. Such an assumption is fair since the properties of the adjunction by definition of the left adjoint object will still hold (in particular the middle equality of the equation below), regardless of existence of the functor as a whole: \begin{align*}
	\HomA(A_0, A) = \HomA(F(\{\text{pt}\}), A) = \HomB(\{pt\}, U(A)) = U(A).
\end{align*}

We can phrase this, without reference to a left adjoint functor, as the following universal property: given a concrete category $(\mathcal{A},U)$ and $\mathcal{A}$-object $A_0$ with an injective set map $\{pt\} \stackrel{\psi}{\hookrightarrow} U(A_0)$  we call $A_0$ the free object on $\{pt\}$ if and only if for every $\mathcal{A}$-object $A$ and set map $\{pt\} \stackrel{\varphi}{\rightarrow} U(A)$ there exists a unique $\mathcal{A}$-morphism $f \in \HomA(A_0, A)$ such that $U(f) \circ \psi = \varphi$..

In the  examples which are to be discussed in this thesis, we only check that such a left adjoint object does exist in the relevant category by showing what they are and that they satisfy the universal property.

Note that in each example we discuss (except for one), the underlying set functor we consider is the usual forgetful  functor, that takes the set-datum of the categorical objects which are built into their construction, and forgets the rest of the structure. Therefore we assume the existence of $U$ at the outset, unless otherwise stated. 

However there is no reason a priori that the underlying set functor should be the forgetful functor; for an illuminating exercise consider the category $\BG$ with one object where $\BG(\{pt\},\{pt\}) = U_{\Grp}(G)$ for some $G \in (\Grp, U_{\Grp})$.
\chapter{Concrete Dualities and the Schizophrenic Object}	

In this chapter we want to describe the general setting of a concrete duality, and prove some important theorems which we will use as a basis for describing our examples. We will be following \emph{Porst and Tholen's Concrete Dualities} \cite{MR1147921} very closely.

Let $(\mathcal{A}, U)$  and $(\mathcal{B},V)$ be two concrete categories over the following representable functors with representing objects $A_0 \in \mathcal{A}$ and $B_0 \in \mathcal{B}$, our free objects on one free generator:
\begin{align*}
	&\HomA(A_0, -) \cong U: \mathcal{A} \to \Set\\
	&\HomB(B_0, -) \cong V: \mathcal{B} \to \Set.\\
\end{align*}
Let $T: \mathcal{A} \to \mathcal{B}$ and $S: \mathcal{B} \to \mathcal{A}$ be contravariant functors with natural transformations $\eta: 1_\mathcal{B} \to TS$ and $\epsilon :  1_\mathcal{A} \to ST$. We can view this as the following \textbf{\emph{dual adjunction}}
\[\begin{tikzcd}
	{\mathcal{A}} & \bot & {\mathcal{B}^{\op}}
	\arrow["T", bend left, from=1-1, to=1-3]
	\arrow["S", bend left, from=1-3, to=1-1].
\end{tikzcd}\]

We also refer to a general dual adjunctions as a \textbf{\emph{duality}}. When these natural transformations are natural isomorphisms we are speaking of a \textbf{\emph{dual equivalence}}. 

Like any adjunction we have the following triangle equalities \begin{align*}
	&T{\epsilon_A} \circ \eta_{TA} = 1_{TA} \ &\text{and}& \ &  S{\eta_B} \circ \epsilon_{SB} = 1_{SB}&
\end{align*}
and $\Hom$-set isomorphisms \begin{align*}
	\HomA(A, SB) \cong \HomB^{\op}(TA,B) \cong \HomB(B, TA).
\end{align*}

Given some $\mathcal{A}$-morphism $f: A\to A'$, consider the map $Uf: UA \to UA'$. More explicitly, this is the map \begin{align*}
 	\HomA(A_0, -) (f) : \HomA(A_0, A) \stackrel{f \circ (-)}{\to}& \HomA(A_0, A')\\
 	\t x \mapsto& f\circ \t x.
 \end{align*}
We shorten this with the notation $[f]: [A] \to [A']$.

The following pair of objects will be of central importance to this thesis, which are defined as the following:
\begin{align*}
	\t A := SB_0&&
	\t B := TA_0.
\end{align*}

We will come to refer to a dual adjunction in this situation as a \textbf{\emph{concrete duality}}.

From these characteristics we can deduce how $S$ and $T$ should be defined, to which a few lemmas will illuminate the bigger picture.
\begin{lemma}
\begin{align}\label{eq:underlying-hom}
		VT \cong \HomA(-, \t  A)&&
	US \cong \HomB(-, \t  B)
\end{align}
\end{lemma}
\begin{proof}
Since our representable functors and adjoint functors are natural in $A$ and $B$, we may compute $V$ and $VT$ (respectively $U$ and $US$) pointwise:
\begin{align*}
	VT(A) = \HomB(B_0, TA) \cong \HomA(A,SB_0) = \HomA(A, \t A) \\
	\implies VT \cong \HomA(-, \t A)
\end{align*}
\begin{align*}
	US(B) =  \HomA(A_0, SB) \cong  \HomB(B,TA_0) =  \HomB(B, \t B) \\
	\implies US \cong \HomB(-, \t B).
\end{align*}
\end{proof}
Should we have strict identities
\begin{align*}
		VT = \HomA(-, \t  A)&& US = \HomB(-, \t  B)
\end{align*} 

we say that the adjunction is \emph{strictly represented} by $\t A$ and $\t B$. 

This result should already give us an idea of how our adjunction is to be induced. That is, our adjoint functors should be regarded as lifts of the $\Hom$-set functors in \ref{eq:underlying-hom} to the relevant categories.\footnote{In the course of this introduction we will often prove results about $\mathcal{A}$, respectively $T: \mathcal{A}\to \mathcal{B}$ with unit $\epsilon:1_\mathcal{A} \to ST $, from which the results about $\mathcal{B}$, respectively $S: \mathcal{B} \to \mathcal{A}$ with counit $\eta: 1_\mathcal{B} \to TS$, follow completely analogously. Unless we state that results do not follow analogously, we assume this to be the case.}  The goal of this introduction to concrete dualities is to make this notion precise. 



\begin{lemma}
	$V \t B \cong U \t A$
\end{lemma}
\begin{proof}
	\begin{align*}
		V \t B \stackrel{Def. V}{=}& \HomB(B_0,\t B) \stackrel{Def. \t B}{=} \HomB(B_0, TA_0)\\ \stackrel{Adjunction}{\cong}& \HomA(A_0,SB_0) \stackrel{Def. \t A}{=}\HomA(A_0,\t A) \stackrel{Def. U}{=} U\t A
	\end{align*} 
	\end{proof}

In plain English, the previous lemma said that the underlying sets of our adjoint functors are given by "homming into" $\t A$ and $\t B$, respectively, and this lemma shows that the underlying sets of $\t A$ and $\t B$ are the same (up to canonical bijection). 

	
	Now we will show how the objects $\t A$ and $\t B$ actually induce the adjunction $T \dashv S$. To do this we must first show that the unit and counit of our adjunction are given \emph{by evaluation}, that is as lifts of evaluation maps:\begin{align*}
		[[\epsilon_A](x)]: \HomA(A, \t A) &\to [\t B]& [[\eta_B](y)] : \HomB(B, \t 
		B) &\to \t A\\
		f & \mapsto f(x)& g &\mapsto g(y).
	\end{align*}
	
	In the following we define the canonical "evaluation" maps, $\varphi_{A, x}$ and $\psi_{B, y}$ and the canonical bijections $\tau$ and $\sigma$:  
\begin{align*}
	\varphi_{A, x} : \HomA(A, \t A) &\to [\t A ]& \psi_{B, y} : \HomB(B, \t B) &\to [\t B ]\\
	 s &\mapsto [s](x) &t &\mapsto [t](y) \\\\
	 \tau: [\t A] &\to [\t B]&\sigma: [\t B] &\to [\t A]\\
	 \t x &\mapsto  [[\epsilon_{\t A}](\t x)](1_{\t A})&\t y &\mapsto  [[\eta_{\t B}](\t y)](1_{\t B})
\end{align*}
which evaluate the maps $[s]$ and $[t]$ at $x$ and $y$ respectively:
\begin{align*}
	[s]: [A] &\to [\t A]&[t]: [B] &\to [\t B]\\
	x &\mapsto [s](x)& y &\mapsto [t](y)
\end{align*}
as for any $s \in \HomA(A, \t A) $, we have the induced map $[s]: [A] \to [\t A]$.

So for every $x \in \HomA(A_0,  A)$ we have the following diagram.
\[\begin{tikzcd}
	\HomA(A, \t A) && \HomA(A_0, \t A) && \HomB(B_0, \t B) \\
	&& \HomA(A_0, SB_0) && \HomB(B_0, TA_0)\\
	s && {[s](x)} && {\tau([s](x))} \\
	\arrow[maps to, from=3-1, to=3-3]
	\arrow[maps to, from=3-3, to=3-5]
	\arrow["\varphi_{A,x}", from=1-1, to=1-3]
	\arrow["\tau", from=1-3, to=1-5]
	\arrow["=",from=1-3, to=2-3]
	\arrow["=",from=1-5, to=2-5]
	\arrow["\cong",from=2-3, to=2-5]
\end{tikzcd}\]
While this is all very technical, our first example will make this machinery explicit, so to be able to see exactly how these definitions for $\tau$ and $\sigma$ arise via our unit and counit. 

The first example will also serve to make the notion that the unit and counit are given by evaluation digestible, while the following lemma makes this precise.
 \begin{lemma}
	$\tau$ and $\sigma$ are inverses, and the following identities hold:
	\begin{align*}
		&[[\epsilon_A](x)] = \tau \varphi_{A,x}& & [[\eta_B](y)]= \sigma \psi_{B,y}
	\end{align*}
\end{lemma}
\begin{proof}
	First we check the identities, as understanding them will help us prove that $\tau$ and $\sigma$ are inverses. We only check the left identity, and as the right identity will follow analogously. 
	
	First we have $\tau \varphi_{A,x}(s) = \tau([s](x))$ by definition of $\varphi_{A,x}$. But then by definition of $\tau$ we have $\tau([s](x)) = [[\epsilon_{\t A}][s](x)](1_{\t A})$. 
	
	Since $\epsilon$ is a natural transformation, we have the following commutative diagram for all $A, A' \in \mathcal{A}$ such that there exists a map $A \to A'$. In particular, given $s: A \to \t A$ we have:
\[\begin{tikzcd}
	{1_\mathcal{A}(A)} && {ST(A)} \\
	\\
	{1_\mathcal{A}(\t A)} && {ST(\t A)}
	\arrow["{\epsilon_A}", from=1-1, to=1-3]
	\arrow["s"', from=1-1, to=3-1]
	\arrow["STs", from=1-3, to=3-3]
	\arrow["{\epsilon_{\t A}}"', from=3-1, to=3-3]
\end{tikzcd}\]
so that $[[\epsilon_{\t A}][s](x)](1_{\t A}) = [[STs][\epsilon_A](x)](1_{\t A})$. 

By Lemma 3.1, we know that $[STs] = US(Ts)  = \HomB(Ts, \t B)$, which is just a map  $\HomB(TA, \t B) \to \HomB(T\t A, \t B)$, induced by $Ts: T\t A \to TA$. \emph{Notice that $US(-)$ and $T(-)$ are both contravariant, so that $UST(-)$ is covariant. }

Now as  $[\epsilon_A](x) \in \HomB(TA, \t B)$, we have the induced precomposition

\[\begin{tikzcd}
	&& {} \\
	{T\t A} && {\t B} \\
	{TA}\\
	\arrow["{[\epsilon_A](x)\circ Ts}", dashed, from=2-1, to=2-3]
	\arrow["Ts"', from=2-1, to=3-1]
	\arrow["{[\epsilon_A](x)}"', from=3-1, to=2-3]
\end{tikzcd}\]
which can be otherwise phrased as a right action of $Ts$ on $[\epsilon_A](x)$ so that \begin{align}
[[STs][\epsilon_A](x)](1_{\t A}) = [[\epsilon_A](x)\circ Ts](1_{\t A}).
 \end{align}
% 
% Now we look more closely at what these maps do: Firstly, Ts is simply a precomposition map \begin{align*}
% 	Ts = \HomA(-,\t A): \HomA(\t A, \t A) &\stackrel{- \circ s}{\to} \HomA( A, \t A)\\ f &\mapsto f \circ s
% \end{align*}
% while $US(Ts)$ is the map
% \begin{align*}
% 	US(Ts) = \HomB(-,\t B)(Ts): \HomB(\HomA(A, \t A), \t B) &\stackrel{- \circ Ts}{\to}\HomB(\HomA(\t A, \t A), \t B)\\
% 	g^* &\mapsto g^* \circ Ts.
% \end{align*}
% The map $\epsilon_A(x)$ is given by evaluation, or in other words, it is that map that sends $p \in \Hom(A, \t A)$ to $p(x) \in [\t B]$, up to canonical isomorphism $[\t A] \cong [\t B]$ given by Lemma 1.1.
% 
% Now $Ts$ is sent by $1_{\t A}$ to $1_{\t A} \circ s$, and we see then that $[\epsilon_A](x) \circ Ts$ sends $1_{\t A} \mapsto 1_{\t A} \circ s(x) = s(x)$. 
% 
% Now we can see that the maps $[[\epsilon_A](x)\circ T(-)](1_{\t A})$ and $[[\epsilon_A](x)](-)$ both send $s \mapsto s(x)$, so that $[[\epsilon_A](x)\circ Ts](1_{\t A})=[[\epsilon_A](x)](s)$.
% 

From Lemma 3.1 we know that $VT = \HomA(-, \t A)$ so that we get the induced diagram
\[\begin{tikzcd}
	&& {} \\
	{\HomA(\t A, \t A)} && {[\t B]} \\
	{\HomA(A, \t A)}\\
	\arrow["{[[\epsilon_A](x)\circ Ts]}", dashed, from=2-1, to=2-3]
	\arrow["\text{[$Ts$]}"', from=2-1, to=3-1]
	\arrow["{[[\epsilon_A](x)]}"', from=3-1, to=2-3]
\end{tikzcd}\]
Notice that then $[Ts]$ becomes an evaluation of the precomposition functor $\HomA(-, \t A)$ at $s$, i.e. $[Ts] = - \circ s$, which sends $1_{\t A} \mapsto s$. Therefore it holds that \begin{align*}
 	[[\epsilon_A](x)\circ Ts](1_{\t A})=[[\epsilon_A](x)](s).
 \end{align*}
 All together we have \begin{align*}
 	\tau \circ \varphi_{A,x}(s)= &\tau([s](x))\\
 	=&[[\epsilon_{\t A}][s](x)](1_{\t A})\\
 	 =& [[STs][\epsilon_A](x)](1_{\t A})
 	\\=&[[\epsilon_A](x)\circ Ts](1_{\t A})\\
 	=&[[\epsilon_A](x)](s)
 \end{align*}
 which gives us the desired identity.
 
 Now we check that $\tau$ and $\sigma$ are inverses. 
 
 The above identity gives us the particular instance \begin{align*}
 	\tau \varphi_{S\t B, 1_{\t B}}(s) = [[\epsilon_{S\t B,\t y}](1_{\t B})](s)
 \end{align*}
 for all $s \in \HomA(S\t B, \t A)$.
 
 For $s = [\eta_{\t B}](\t y)$ with $\t y \in [\t B]$, noticing the maps \begin{align*}
 	[\eta_{\t B}](\t y): S\t B &\to \t A & \ \ \ \varphi_{S\t B, 1_{\t B}}: \HomA(S\t B, \t A) \to & [\t A]\\
 	1_{\t B} &\mapsto 1_{\t B}(\t y)& s \mapsto& [s](1_{\t B})
 \end{align*}
 we see that $[[\eta_{\t B}](\t y)](1_{\t B}) = \varphi_{S \t B, 1_{\t B}}(s)$ and so we  have \begin{align*}
 	\tau \sigma (\t y) =& \tau ( [[\eta_{\t B}](\t y)](1_{\t B}))\\
 	 =&  \tau(\varphi_{S\t B, 1_{\t B}}(s))\\
 	 =& [[\epsilon_{S\t B}](1_{\t B})][\eta_{\t B}](\t y) = \t y.
 \end{align*}
 We only need to show the last equality. 
 
 Consider the triangle equality $S\eta_{\t B} \circ \epsilon_{S\t B} = 1_{S \t B}$ which induces $[S\eta_{\t B}] [\epsilon_{S\t B}] = 1_{[S \t B]}$.
 
 We can extrapolate from (1)  using the result $US = \HomB(-, \t B)$ from Lemma 3.1 that for some $A \in \mathcal{A}$, $x \in A$, $B \in \mathcal{B}$ and $f \in \HomB(B, TA)$ the left action of $[Sf]$ on $[\epsilon_A](x) \in \HomB(TA, \t B)$ becomes a right action of $f$ on $[\epsilon_A](x)$, i.e. \begin{align*}
 	[Sf][\epsilon_A](x) = [\epsilon_A](x) \circ f \in \HomB(B, \t B).
 \end{align*}
 Therefore \begin{align*}
 	1_{\t B} = 1_{\HomB(\t B, \t B)}(1_{\t B}) =1_{[S \t B]}(1_{\t B}) = [S\eta_{\t B}] [\epsilon_{S\t B}](1_{\t B}) = [\epsilon_{S\t B}](1_{\t B}) \circ \eta_{\t B}.
 \end{align*}
 
 In other words, the induced map is the identity on $[\t B]$:\begin{align*}
 	[[\epsilon_{S\t B}](1_{\t B})][ \eta_{\t B}] = 1_{[\t B]}.
 \end{align*}
 
  Thus we have $\tau \sigma = 1_{[\t B]}$ as desired.
\end{proof}
% Recall that the following map is defined as evaluating at $1_{\t B}$:\begin{align*}
% 	[[\epsilon_{S\t B}](1_{\t B})]: [\HomA(\HomB(\t B , \t B), \t A)] &\to [\t B]\\
% 	p &\mapsto p(1_{\t B})
% \end{align*}
%% with \begin{align*}
%% 	[\epsilon_{S\t B}]:\HomB(\t B, \t B) &\to \HomB(\HomA(\HomB(\t B, \t B), \t A), \t B)\\
%%1_{\t B} & \mapsto (p \to p(1_{\t B}))\\
%% \end{align*}
%which means $[[\epsilon_{S\t B}](1_{\t B})][\eta_{\t B}](\t y) =[\eta_{\t B}](\t y)(1_{\t B})$ and so
%\begin{align*}
% 	[[\epsilon_{S\t B}](1_{\t B})][\eta_{\t B}]: [\t B] &\to [\t B]\\
% 	\t y &\mapsto [\eta_{\t B}](\t y)(1_{\t B}) = 1_{\t B}(\t y) = \t y.
% \end{align*}

% Remembering that $[S\eta_{\t B}][\eta_{S \t B}] = 1_{[S \t B]}$ 
% since \begin{align*}
% 	\eta_{\t B}: \t B \to \HomA(\HomB(\t B, \t B),\t A)
% \end{align*}
% means \begin{align*}
% 	S\eta_{\t B}: \HomB(\HomA(\HomB(\t B, \t B), \t A), \t B) \to \HomB(\t B, \t B)
% \end{align*}
% and \begin{align*}
% 	\epsilon_{S \t B}: \HomB(\t B, \t B) \to \HomB(\HomA(\HomB(\t B, \t B), \t A), \t B)
% 	 \end{align*}
% 	 evaluated at $1_{\t B}$ gives $[\epsilon_{S \t B}](1_{\t B}) = 1_{\t B}$, from which we \emph{somehow} deduce that \begin{align*}
% 	 	\tau \sigma = 1_{[\t B]}
% 	 \end{align*} (\textcolor{red}{TODO}: why?)
Lemma 3.3 makes precise the notion that the underlying maps of our unit and counit are given "by evaluation":
\begin{align*}
	[[\epsilon_A](x)]: \HomA(A, \t A) &\to [\t B] & [[\eta_B](y)]:\HomB(B, \t B) &\to [\t A]\\
	f &\mapsto f(x) & g &\mapsto g(y) 
\end{align*}
and that these maps induce the canonical bijections $\tau$ and $\sigma$. 

So far we have described our  situation only on underlying sets. In particular we have just shown that the underlying set map of the unit and counit is given by
\begin{align*}
	[\epsilon_A]:  [A]&\to \HomB(\HomA(A, \t A),\t B) & [\eta_B]: [B]&\to \HomA(\HomB(B, \t B),\t A)\\
	x &\mapsto (f \to f(x)) & y &\mapsto (g\to g(y) ).
\end{align*}
In the following we will want to show how these evaluation maps actually lift as maps in the corresponding category. We want to see that for every $A \in \mathcal{A}$ the evaluation map $(f \to f(x))$ lifts to a $\mathcal{B}$-morphism, where $\HomA(A,\t A)$ is viewed as a $\mathcal{B}$-object, and thus the unit and counit
\begin{align*}
	\epsilon_A:  A&\to STA & \eta_B: B&\to TSB\\
	x &\mapsto (f \to f(x)) & y &\mapsto (g\to g(y) )
\end{align*}
are well-defined.

In our examples, we will describe explicitly how we can impose $\mathcal{B}$ structure on our evaluation maps, which we will often denote by  $\ev_{A,x}$ for readability.
\section{Schizophrenic Objects}
In general, what we want for our set up is that the composition of these maps induces a $U$-structured lift, i.e. for every $A \in \mathcal{A}$ and  $x \in A$ there exists an $e_{A,x} \in \HomB(TA, \t B)$, that induces  $[e_{A,x}]: [TA] \to [\t B]$, such that $[e_{A,x}] = \tau \varphi_{A,x}$, i.e., the following diagram commutes: 
\[\begin{tikzcd}
	& TA && {\t B} \\
	\\
	{\HomB(B_0, TA)} & {\HomA(A, \t A)} && {\HomB(B_0, \t B)}
	\arrow["e_{A,x}",dashed, from=1-2, to=1-4]
	\arrow["U", from=1-2, to=3-2]
	\arrow["U", from=1-4, to=3-4]
	\arrow["{=}", shift right=1, draw=none, from=3-1, to=3-2]
	\arrow["{\tau \varphi_{A,x}}", from=3-2, to=3-4]
\end{tikzcd}\]
However, as we will see, this will not be enough to induce $(T \dashv S)$ given the triple $(\t A, \tau, \t B)$. For this we will want to additionally impose the constraint that these lifts are initial. We call such lifts \emph{\textbf{natural}} (not to be confused with naturality of a natural transformation), while scenarios where we have $U$-structured  lifts that are not initial are called \emph{\textbf{non-natural}}.

\subsection{Natural Dual Adjunction}

We are now ready to introduce the central notion of our thesis: the schizophrenic object. From here, we simply start by assuming that $(\mathcal{A},U)$ and $(\mathcal{B},V)$ are concrete categories whose faithful functors are determined by representable objects given by the free object on one free generator.
\\


\begin{definition}
	A triple $(\t A, \tau, \t B)$ with a pair of objects $(\t A, \t B) \in \mathcal{A} \times \mathcal{B}$ and a bijective map $\tau: [\t A] \to [\t B]$ is called a \textbf{schizophrenic object}  if the following conditions are satisfied:
\begin{enumerate}
		\item[SO1.] For every $A \in \mathcal{A}$ the family $(\tau\varphi_{A,x}: \HomA(A, \t A) \to [\t B])_{x \in [A]}$ admits a $V$-initial lifting $(e_{A,x}:TA \to \t B)_{x \in [A]}$
		\item[SO2.] For every $B \in \mathcal{B}$ the family $(\sigma\psi_{B,y}: \HomB(B, \t B) \to [\t A])_{y \in [B]}$ admits a $U$-initial lifting $(d_{B,y}:SB \to \t A)_{y \in [B]}$
	\end{enumerate}
\end{definition}

The $V$-structured lifting property yields the existence of  a $\mathcal{B}$-morphism $e_{A,x} \in \HomB(TA, \t B)$ for every $A \in \mathcal{A}$ and $x \in A$, such that $[TA] = \HomA(A, \t A)$ and $[e_{A,x}] = \tau \varphi_{A,x}$.

The $V$-initiality means that for any $Z \in \mathcal{B}$ and a map $h: [Z] \to [TA]$, if  all composite maps $\tau \varphi_{A,x} \circ h$ are the underlying-set maps for $\mathcal{B}$-morphisms in $\HomB(Z, \t B)$, then there exists a unique $\mathcal{B}$-morphism $h' \in \HomB(Z, TA)$  whose underlying set map is $h$.

In other words, the $V$-structured lift is initial among all such lifts: if $Z$ is any other $\mathcal{B}$-object whose underlying set maps into $\HomA(A, \t A)$ in a way that is compatible with all $\tau \varphi_{A,x}$ composites, then that map factors uniquely through $TA$ in $\mathcal{B}$. 



\[\begin{tikzcd}
	Z && TA && {\t B} \\
	\\
	{[Z]} && {\HomA(A, \t A)} && {[\t B]}
	\arrow["{\exists ! h'}", dashed, from=1-1, to=1-3]
	\arrow["{ ({\tau\varphi_{A,x}\circ h})'}", bend left, from=1-1, to=1-5]
	\arrow[from=1-1, to=3-1]
	\arrow["{e_{A,x}}", from=1-3, to=1-5]
	\arrow["V", from=1-3, to=3-3]
	\arrow["V", from=1-5, to=3-5]
	\arrow["h", from=3-1, to=3-3]
	\arrow["{\tau  \varphi_{A,x}}", from=3-3, to=3-5]
\end{tikzcd}\]


We now show a central theorem to this thesis.

\begin{theorem}
	Every schizophrenic object $(\t A, \tau , \t B)$ induces a natural dual adjunction strictly represented by $(\t A, \t B)$, such that $\tau$ and $\sigma = \tau^{-1}$ are the canonical bijections defined in the previous section.
\end{theorem}
\begin{proof}
	First we show that $T$ and $S$ are well defined functors. The conditions (SO1.) and (SO2.) show us how $T$ and $S$ act on objects up to underlying-set isomorphism. 
	
	Now we show how $T$ acts on morphisms. To that effect, given some $f: A \to A'$, we seek to show the existence of $Tf: TA' \to TA$ whose underlying set map is $[Tf] = \HomA(f, \t A): \HomA(A', \t A) \to \HomA(A, \t A)$, which sends $s \mapsto s \circ f$. As we have just seen, by $(SO1)$ it suffices to show that $\tau \varphi_{A,x}\circ \HomA(f, \t A)$ are the underlying set maps of  $\mathcal{B}$-morphisms in $\HomB(TA', \t B).$
	
	Considering that $[Tf]$ is simply the precomposition map $- \circ f$, we see that given some $s \in \HomA(A', \t A)$, it holds that $\tau \varphi_{A,x}\circ \HomA(f, \t A)(s) =\tau \varphi_{A,x}(sf)$. But since $[sf](x) = [s][f](x)$, where $[f](x) \in [A']$, we have $\tau \varphi_{A,x}(sf) = \tau \varphi_{A',[f](x)}(s) = [e_{A',[f](x)}](s)$, which is the underlying set map of a $\mathcal{B}$-morphism in $\HomB(TA', \t B)$ by definition, and which exists by the lifting property given by $(SO1.)$. 
	
	Therefore $T$ and $S$ are well-defined functors, where preservation of the identity and the composition law follow the same logic as above, using $V$-initiality and the fact that the underlying-set map is defined by precomposition.
	
	Now we show that $T$ and $S$ are adjoint, and to do that we shall construct  unit and counit maps $\epsilon$ and $\eta$. In order to establish the existence of $\eta_B$ by playing the same game we first define $[\eta_B]: [B] \to [TSB]$ and show that each $\tau \varphi_{SB, t} \circ [\eta_B]$ with $t \in [SB]$, can be lifted along $V$, i.e. that it is the underlying set map of a $\mathcal{B}$-morphism in $\HomB(B, \t B)$. So we define in light of Lemma 3.3 under (SO2.)
	\begin{align*}
		[\eta_B]:[B] &\to \HomA(SB, \t A)\\
		y &\mapsto d_{B,y}.
	\end{align*}
Then by definitions and (SO2.) we have  \begin{align*}
	\tau \varphi_{SB, t} \circ [\eta_B](y) &= \tau \varphi_{SB, t} (d_{B,y})\\ &= \tau [d_{B,y}](t)\\ &= \tau \sigma \psi_{B,y}(t)\\ &= [t](y)
\end{align*}
which shows that $\tau \varphi_{SB, t} \circ [\eta_B]: [B] \to [\t B]$ is the underlying set map of a $\mathcal{B}$-morphism in $\HomB(B, \t B)$, proving the existence of $\eta_B$. 

Furthermore we see that \begin{align}
	e_{SB, t} \circ \eta_B = t \ \ \text{ for all } t \in [SB]= \HomB(B, \t B).
\end{align}
Since $[\eta_B]$ lifts for all $B$, we may verify naturality on underlying-set maps; we verify that given a $\mathcal{B}$-morphism $f: B \to B'$, that we have $ [\eta_{B'}]\circ (f \circ -) = [TSf] \circ [\eta_B]$, or in other words: \begin{align}
	d_{B', f \circ y} = [TSf] \circ d_{B,y}.
\end{align} 
Remember that the left action of $[TSf]$ on $d_{B,y}$ is a right action of $Sf$ on $d_{B,y}$. But the underlying map of $d_{B,y} \circ Sf$ is $\sigma \psi_{B,y} \circ \HomB(f, \t B)$ which is, up to the bijection $\sigma$, just the evaluation map of a morphism in $\HomB(B, \t B)$ at some $y \in [B]$ precomposed with $f \in \HomB(B, B')$, which yields the evaluation map of a morphism in $\HomB(B',\t B)$ at $f \circ y \in [B']$. In other words $[d_{B,y}] [Sf] = [\sigma \psi_{B', f \circ y}] = [d_{B', f \circ y}]$, which is clearly the underlying set map of $d_{B', f \circ y}$, giving us (4) by uniqueness of the lift. 



The definition of $S$ gives us that $[S\eta_B][\epsilon_{SB}](t) = \HomB(\eta_B, \t B)(e_{SB, t})$ and by (3) we have $\HomB(\eta_B, \t B)(e_{SB, t}) = e_{SB,t} \circ \eta_B = t$. Since $U$ is faithful, any map $[Sf] \in \Set([SB'], [SB])$ is the underlying-set map to a unique map $Sf \in \HomA(SB', SB)$, and we deduce that $[S\eta_B][\epsilon_{SB}] = 1_{[SB]}$ is the underlying set map of the triangle identity $S\eta_B \circ \epsilon_{SB} = 1_{SB}$.

Finally, to show that $\tau$ is induced by this adjunction it suffices to see that it maps $\t x \mapsto [[\epsilon_{\t A}](\t x)](1_{\t A})$ as desired. For every $\t x \in [\t A]$ we have
\begin{align*}
	[[\epsilon_{\t A}](\t x)](1_{\t A}) = \tau \varphi_{\t A, \t x}(1_{\t A}) = \tau([1_{\t A}](\t x)) = \tau(\t x).
\end{align*}
\end{proof}
A dual adjunction induced by a schizophrenic object in this way is called a \textbf{\emph{natural dual adjunction}}. In the following we shall briefly discuss dual adjunctions which are non-natural, as in our examples we will see that some modifications are in order to make it natural. 
\section{Non-Natural Dual Adjunction}

Let there be a dual adjunction $(S', T')$ that satisfies the situation described in $\S 3$. This adjunction already determines a triple $(\t A, \tau, \t B)$ such that the following weakened conditions of $SO1$ and $SO2$ are fulfilled:
\begin{enumerate}
		\item[WSO1.] For every $A \in \mathcal{A}$ the family $(\tau\varphi_{A,x}: \HomA(A, \t A) \to [\t B])_{x \in [A]}$ admits a $V$-structured lift $(e_{A,x}:T'A \to \t B)_{x \in [A]}$ which extends functorially, i.e., for every $A \stackrel{f}{\to} A'$ in $\mathcal{A}$ there exists a $\mathcal{B}$-morphism $T'A' \stackrel{T'f}{\to}T'A$ with $[T'f] = \HomA(f,\t A)$. 
		\item[WSO2.] For every $B \in \mathcal{B}$ the family $(\sigma\psi_{B,y}: \HomB(B, \t B) \to [\t A])_{y \in [B]}$ admits a $U$-structured lift $(d_{B,y}:S'B \to \t A)_{y \in [B]}$ which extends functorially.
	\end{enumerate}
	
Though $(S',T')$ determins a triple, such a triple does not necessarily induce $(S',T')$ like the schizophrenic object induces $(S,T)$. 

However, if we are in this situation, there are potential modifications which may give us a natural dual adjunction. There are in particular two methods which we will remark. 

Firstly, one may use the triple $(\t A, \tau, \t B)$ to induce a natural dual adjunction $(S,T)$ on the concrete categories $(\mathcal{A}, U)$ and $(\mathcal{B},V)$. Such a method requires additional assumptions on $(\mathcal{A},U)$ and $(\mathcal{B},V)$ which we will not discuss. For more details, see [1 1-D]. 

The second method is the one which we will later see in action in our examples, which is to restrict our  adjunction to full subcategories of our categories under which we have an equivalence. 

In the next part we will discuss a situation which induces a non-natural dual adjunction between concrete categories, as it is relevant to an important example.
\section{Internal Hom-Functors}\label{sec: int hom}

Let $(\mathcal{A}, U)$ be a concrete category. An \textbf{internal hom-functor} is a functor $H: \mathcal{A}^{\op} \times \mathcal{A} \to \mathcal{A}$ such that $UH = \HomA(-,-)$. Moreover 
\begin{align*}
	\text{all evaluation maps } \phi_{A,A',x}: \HomA(A,A') &\to [A']\\
	h &\mapsto [h](x)\\
	\text{lift to $\mathcal{A}$-morphisms } p_{A,A',x}: H(A,A') &\to A' \text{ for all } A, A' \in \mathcal{A}, x \in [A]
\end{align*}
Any cartesian closed concrete category which admits function spaces is a good example of a category with internal hom-functors. Recall the following definitions (from [2]):

\
\begin{definition}[Cartesian closed category] A category $\mathcal{A}$ is \textbf{cartesian closed} if it has finite products and for each $\mathcal{A}$-object $A$ the functor $(A \times -)$ has a right adjoint  $(-)^A$, called the \textbf{Heyting implication}. For $B \in \mathcal{A}$ we call $B^A$ an \textbf{exponentiable object}. 
\end{definition}
\

\begin{definition}
	A concrete category $ \concA$ is said to  \textbf{admit function spaces} if $\mathcal{A}$ is cartesian closed, $\concA$ admits finite concrete products, and the evaluation morphisms $A \times B^A \stackrel{\ev}{\to} B$ can be chosen in such a way that $U(B^A)= \HomA(A,B)$ where $\ev$ is the restriction of the canonical evaluation map in $\Set$.  \end{definition}

A cartesian closed concrete category which admits function spaces admits an internal hom-functor by definition, since we can choose $B^A$ to be our lift.

 
Now we want to see how and under what conditions can  a category which admits an internal hom-functor  induce a dual adjunction. Let $\concA$ admit an internal hom-functor $H$, and let there be a concrete category $\concB$ and a concrete functor $\lvert - \lvert : \mathcal{B} \to \mathcal{A}$ such that $\HomB(B, C) \hookrightarrow \HomA(\lvert B \lvert, \lvert C \lvert)$ lift to $\mathcal{A}$-morphisms\footnote{$\HomB_\mathcal{A}(-,-)$ is notation for the $\Hom$-set in $\mathcal{B}$ as $\mathcal{A}$-object} $\gamma_{B,C}: \HomB_\mathcal{A}(B,C) \to H(\lvert B \lvert, \lvert C\lvert)$.  In a monotopological category, this can be done by lifting initially.

Given $\t B \in \mathcal{B}$ with $\t A := \lvert \t B \lvert $ and $\tau = 1_{[\t A]}$, we can check that $WSO2$ is fulfilled. In other words we are seeking a lift of the map $\sigma \psi_{B,y}: \HomB(B, \t B) \to [\t A]$. But such a lift is given by the internal hom-functor, given that the bottom part of the following diagram commutes:
\[\begin{tikzcd}
	{\HomB_\mathcal{A}(B, \t B)} && {H(\lvert B\lvert, \t A)} && {\t A} \\
	\\
	{\HomB(B, \t B)} && {\HomA(\lvert B \lvert , \t A) } && {[\t A]}
	\arrow["{\gamma_{B, \t B}}", from=1-1, to=1-3]
	\arrow[from=1-1, to=3-1]
	\arrow["{p_{\lvert B \lvert, \t A, y}}", from=1-3, to=1-5]
	\arrow[from=1-3, to=3-3]
	\arrow[from=1-5, to=3-5]
	\arrow["\iota", hook, from=3-1, to=3-3]
	\arrow["{\sigma \varphi_{B,y}}", bend right, from=3-1, to=3-5]
	\arrow["{\phi_{\lvert B\lvert , \t A,x }}", from=3-3, to=3-5]
\end{tikzcd}\]

That is, if $\sigma \varphi_{B,y} = \phi_{\lvert B\lvert, \t A, x} \circ \iota$ then we have $d_{B, y} = p_{\lvert B \lvert, \t A, y} \circ \gamma_{B, \t B}$, which gives us $WSO2$. 

The question is if the concrete functor commutes with the evaluation map, since $\t A = \lvert \t B \lvert $ and $\tau$ (and therefore also $\sigma$) is the identity.

\[\begin{tikzcd}
	{\HomB(B, \t B)} && {[\t B]} \\
	\\
	{\HomA(\lvert B \lvert, \t A) } && {[\t A]}
	\arrow["{\varphi_{B, y}}", from=1-1, to=1-3]
	\arrow["i"', hook', from=1-1, to=3-1]
	\arrow["\sigma", from=1-3, to=3-3]
	\arrow["{\phi_{\lvert B \lvert, \t A, x}}"', from=3-1, to=3-3]
\end{tikzcd}\]

In other words, we check $\sigma([f](y)) = [\lvert f \lvert ](\iota(y))$, but this follows from faithfulness of the concrete functor $\lvert - \lvert $. 

Therefore we have a contravariant functor $S(B) = \HomB_\mathcal{A}(B, \t B)$. 

Now if for every $A \in \mathcal{A}$ we can lift the $\mathcal{A}$-source $(p_{A, \t A, x}: H(A, \t A) \to \t A)$ along $\lvert - \lvert$ functorially, then we also have WSO1, inducing the contravariant functor $T: \mathcal{A} \to \mathcal{B}$ such that $\lvert T(A) \lvert = H(A, \t A)$, and as such we have the desired dual adjunction $(S, T)$. We will however only show this example-wise. 

\subsection{Internal Hom in Topological Categories}

In topological categories with internal $\Hom$-functor, the induced adjunction is almost never natural. 

Consider that if $(\mathcal{A}, U)$ is a topological concrete category with a terminal object $\top$, then the underlying set of $\top$ must be the point. To see this, consider any empty source out of it $(\top \stackrel{\emptyset}{\rightarrow} A)_{\emptyset}$. Then $([\top] \stackrel{\emptyset}{\rightarrow} [A])_{\emptyset}$ is a $U$-structured source, so that it admits a unique initial lift. Now our original source satisfies the initial property, since any $\mathcal{A}$-object $B$ whose underlying set is compatible with our $U$-structured source admits a unique morphism into $\top$ by terminality. 

However, since our source is empty, compatibility imposes no restrictions. That means \emph{every} set map $[B] \stackrel{g}{\rightarrow} [\top]$ is compatible with the empty source.  Uniqueness of our lift gives us a unique map $f$ for every $g \in \Set([B],[\top])$,  but $\top$ is terminal, so that every $g$ lifts to a single $f$. But $U$ is faithful, so we have $g = [f]$ for all $g$, and therefore it is also unique.

That means we obtain a universal condition:  for every $B \in \mathcal{A}$, there exists a unique underlying set map $[B] \stackrel{g}{\rightarrow} [\top]$. This is reflected by the diagram below:

\[\begin{tikzcd}
	{B} && {\top}  && {A} \\
	\\
	{[B]} && {[\top]} && {[A]}
	\arrow["{\exists !f}", dashed, from=1-1, to=1-3]
	\arrow[from=1-1, to=3-1]
	\arrow["{\emptyset}", from=1-3, to=1-5]
	\arrow[from=1-3, to=3-3]
	\arrow[from=1-5, to=3-5]
	\arrow["{\exists !g=[f]}", from=3-1, to=3-3]
	\arrow["{\emptyset}", from=3-3, to=3-5].
\end{tikzcd}\]


Setting $B = \top$, we obtain $\Set([\top],[\top]) = \{pt\}$, which can only be the case if  $\lvert [\top] \lvert = 1$, and thus $[\top] = \{pt\}$. By our free-forget adjunction, this means for all $A \in \mathcal{A}$ it holds that $\HomA(\top, A) = [A]$.

Now if $(\mathcal{A},U)$ admits function spaces, naturality of the adjunction would give us, for any set map $[f]:[A'] \to [A]$ a unique lifting diagram

\[\begin{tikzcd}
	{H(\top, A')} && {H(\top,A)} && A \\
	\\
	{\HomA(\top,A')}=[A'] && {[A]=\HomA(\top, A)} && {[A]}
	\arrow["{f}", dashed, from=1-1, to=1-3]
	\arrow[from=1-1, to=3-1]
	\arrow["{e_{\top, \ast}}", from=1-3, to=1-5]
	\arrow[from=1-3, to=3-3]
	\arrow[from=1-5, to=3-5]
	\arrow["{[f]}", from=3-1, to=3-3]
	\arrow["{\ev_{\top, \ast}}", from=3-3, to=3-5].
\end{tikzcd}\]
In other words, we would be able to lift arbitrary set maps to $\mathcal{A}$-morphisms. For non-trivial categories this is almost never true. For intuition consider $\Top$ where this translates to the obviously false claim that every set map is continuous.

%(\textcolor{red}{TODO} check this for yourself)
\section{Motivation}
We are now in the position to understand what this schizophrenic object really affords us. What is interesting about it is that the object itself induces the adjunction via $\Hom$-sets in our respective categories. 

So our question for the examples will be the following: given a good candidate  $(\t A, \tau, \t B)$ for the schizophrenic object, does this object lend itself to a description of a $\mathcal{B}$-structure on $\HomA(A, \t A)$ for every $A \in \mathcal{A}$ in a way that is \emph{natural} (and similarly for $\t B$)? 

If the desired lifts exist, then by Theorem 4.1 we will have the following adjunction
\[\begin{tikzcd}
	{\mathcal{A}} & \bot & {\mathcal{B}^{\op}}
	\arrow["{T(-):=\HomA_{\mathcal{B}}(-, \t A)}",  bend left, from=1-1, to=1-3, from=1-1, to=1-3]
	\arrow["{S(-):=\HomB_{\mathcal{A}}(-, \t B)}", bend left, from=1-3, to=1-1, from=1-3, to=1-1].
\end{tikzcd}\]

An arguably equally important question which we explore in the examples is if we can give concrete descriptions of $ \HomA_{\mathcal{B}}(-,\t A)$ and $\HomB_{\mathcal{A}}(-,\t B)$ as functors which are either well-known to that category, or that we can understand given the known machinery of that category.

Often times we do get a concrete description, which will help solidify our intuition about these adjunctions. However, when we don't get such a description, we can still explore concrete properties of $\HomA_{\mathcal{B}}(A,\t A)$ as $\mathcal{B}$-object.


Therefore our motivation in the following is two-fold, to show that our candidate is indeed schizophrenic, and to provide intuition by analyzing the adjunctions a little further.


\chapter{Examples}
The rest of our thesis shall be devoted to an analysis of several specific examples (and non-examples) of natural dual adjunctions. Our goal is not only to show that our candidates are indeed schizophrenic objects, but to build intuition for the schizophrenic object through a careful analysis of the induced adjunctions. 

We are going to start with an example which will serve as our leading example: we will elucidate its schizophrenic nature as it ties directly to the generality we have learned, so that we can start to get a sense for in what direct way we can lift $\Hom$-sets.

We will subsequently connect this example to a duality called the Stone duality, by which we plan to give concrete descriptions of these $\Hom$-set functors, for which we had just given a more general description using only universal properties and  $\Hom$-sets. Though we do lead with the general into the concrete, we may otherwise view our elucidation of the concrete as the mental abstraction, since we must show the actual mechanics behind why we can think of these concrete objects as lifts of $\Hom$-sets.

This will then lend itself to the intuition of the following examples, which we can in some sense think of extensions or special cases of the given dualities. 

For the examples note that we may sometimes by abuse of notation and for lack of a better functorial description use $\HomA(-,\t A)$ to denote $T$, unless we denote otherwise.
\section{The $\Top \rightleftarrows \Frm^{\op}$ Duality}

%\emph{A note about notation: When speaking about $\Hom$ sets of a general category $A$, we use the notation as before $\HomA(-,-)$. However to shorten notation and ease readability, when speaking of specific categories, such as $\Frm, \Top,$ or $\Set$, we use the notation $\Frm(-,-), \Top(-,-),$ $ \Set(-.-).$}
In this part we assume basic knowledge about topological spaces and frames. For now we will only use the definitions of objects and morphisms in $\Top$ and $\Frm$.  Our reference text when it comes to any discussion about posets, lattices, frames, or locales, is \emph{Johnstone}'s \emph{Stone Spaces} \cite{zbMATH03940199}. 


The aim of this discussion is two-fold: first we want to concretize the maps using our general description in the first part of this paper. To this end we aim to be very explicit in this example, so that we just once get to see what these maps look like without dropping any notation via identification, which we  do on later examples for readability. 

Secondly, we want to provide argumentative insight for the coming examples. For this reason we will refer to this example as the leading example.

To that end we start with our candidate for the schizophrenic object: $(\mathbb{S}, 1_{\mathbb{S}},\mathbb{2})$. In the topological side we have the Sierpiński space $\mathbb{S} = (\{0,1\}, \Omega(\mathbb{S}))$, where the open sets are given by $\Omega(\mathbb{S}) = \{ \emptyset, \{1\}, \{0,1\}\}$. On the frame side we have the two point poset $\mathbb{2} := \{0 \leq 1\}$.

We aim to prove that this object is schizophrenic and as such induces the following natural dual adjunction

\[\begin{tikzcd}
	{\Top} & \bot & {\Frm^{\op}}
	\arrow["{\text{[T] = $\Top$(--, $\mathbb{S}$)}}" , bend left, from=1-1, to=1-3]
	\arrow["{\text{[S] = $\Frm$(--, $\mathbb{2}$)}}", bend left, from=1-3, to=1-1].
\end{tikzcd}\]

with the following bijections induced by the unit and counit
\begin{align*}
	 \tau: [\mathbb{S}]&\to [\mathbb{2}] &\sigma: [\mathbb{2}] &\to [\mathbb{S}] \\
	\t x &\mapsto [[\epsilon_{\mathbb{S}}](\t x)](1_{\mathbb{S}})&\t y &\to  [[\eta_{\mathbb{2}}](\t y)](1_{\mathbb{2}}).
\end{align*}
In this setting  the respective representable objects are $A_0 = \{pt\}$ and $B_0 = b$, where $b$ is here  notation for the free frame generated by one object, i.e., the free 3-chain $\{\bot \leq b \leq \top\}$. Remember that as a frame needs to be a complete lattice, it must include bottom and top elements $\bot$ and $\top$. 

Now we can see explicitly  that $[TA_0] = \Top(\{pt\}, \mathbb{S}) = [\mathbb{S}]$, since a continuous map from the point is completely determined by where the point gets sent, and $[SB_0] = \Frm(b, \mathbb{2}) = [\mathbb{2}]$, since a simple computation shows that frame homomorphisms must send bottom elements to bottom elements and top elements to top elements, so that it is completely determined by the image of $b$. 

Though we may obtain these equalities purely categorically, we now finally have an example to see in what sense the respective maps are completely determined by where they send the free generator.

However it less immediate  why $\Top(A, \mathbb{S})$ lifts to the category $\Frm$ and why $\Frm(B, \mathbb{2})$  lifts to the category $\Top$.

For a $\Frm$-structure on $\Top(A, \mathbb{S})$ we desire a lift which preserves the evaluation map $\sigma\varphi_{A,x}$, and which is initial among such lifts. This means we want the weakest $\Frm$-structure such that for all $x \in A$ and $A \in \Top$ evaluation maps $\sigma \varphi_{A,x}$ lift to frame homomorphisms $e_{A,x}$.

We claim that the evaluation is a frame homomorphism if and only if it preserves pointwise order in $\mathbb{2}$.  

For the forward direction we have \begin{align*}
	e_{A,x}(u \leq v) = e_{A,x}(u) \leq e_{A,x}(v) = u(x) \leq v(x),
\end{align*} and conversely a pointwise order in $\mathbb{2}$ determines limits and colimits by definition:  $u(x) \land v(x) \leq (u \land v)(x)$ holds since the pointwise intersection is less than or equal to $u(x)$ and $v(x)$ respectively, and $(u \land v)(x) \leq u(x) \land v(x)$ holds since the intersection in the frame is less than or equal to $u$ and $v$ respectively and  evaluating preserves pointwise order.
	
%
%(\textcolor{blue}{TODO} does a pointwise order preserving evaluation morphism into $\mathbb{2}$ preserve finite limits and colimits by some pointwise argument in the presheaf category? in other words, is there a way to abstract this argument so i dont have to prove it on elements?)



So for $A \in \Top$ we construct the $\Frm$-structure on $\Top(A, \mathbb{S})$ by a preorder that preserves the pointwise order in $\mathbb{2}$ for all $x \in A$. As we want the weakest such structure, we define it such that for  arbitrary $ u, v \in \Top(A, \mathbb{S})$ we have $u \leq v$ if and only if $u(x) \leq v(x)$ for all $x \in A$. 

Any other such frame structure must satisfy our pointwise definition, so that this is the initial $\Frm$-structure making all evaluation maps frame homomorphisms.




For a topology on $\Frm(A, \mathbb{2})$, we consider the family \begin{align*}
	\{ \  \{p \in \Frm(A, \mathbb{2}) \   \lvert \  p(x) = 1 \} \ \lvert \ x \in A \ \} 
\end{align*} 

Our adjunction has the counit
\begin{align*}
	\eta:& 1_{\Frm} \to TS\\
	&B \mapsto TSB
\end{align*}
whose underlying map 
\begin{align*}
	[\eta_B]: [B] &\to [\Top(\Frm(B, \mathbb{2}), \mathbb{S})] = \Top(\Frm(B, \mathbb{2}), \mathbb{S})\\
	y & \mapsto (p \to p(y))
\end{align*}
is given by evaluation
\begin{align*}
	[[\eta_B](y)]: \Frm(B, \mathbb{2}) &\to [\mathbb{S}]\\
	p &\mapsto p(y).
\end{align*}


Notice that the topology on $\Frm(B, \mathbb{2})$ is the initial topology making all $(\eta_B(y))_{y \in B}$ continuous, since continuous maps $\eta_B(y)$ into the Sierpiński space $\mathbb{S}$ are uniquely determined by the pre-image of $\{1\}$. One can check that a map $Z \stackrel{h}{\rightarrow}  \Frm(B, \mathbb{2})$ is continuous if and only if all composites $\eta_B(y) \circ h$ are. 

In defining initial lifts we mentioned that our intuition should come from topology, so we found it astute for the leading example to use this intuition on one side of the lift directly.

Keep in mind that for each side of this adjunction respectively, we will see this kind of reasoning more often to argue initiality of lifts.

With this we are done. We do not need to prove triangle equalities, we do not need to prove universality. We only prove that our candidate is schizophrenic, and Theorem 4.1 gives us the adjunction.

We would like however, only on this example, to compute and explicate the bijection of the schizophrenic object, since this example is quite straightforward and leads to a very familiar map, namely the identity. For further examples we do not find it very useful to compute this map, so we  implicitly refer to the abstraction to know it exists. 

We will also drop notation for underlying sets when $[\HomA(A,\t A)] = \HomA(A,\t A)$ is already a set. We will not, however, drop notation for $[\t A]$ and will refer explicitly to our $A_0$. However that is only for this example; after this we refer to elements of $[\t A]$ by abuse of notation not by maps from the free object, but by the images of the free generator under these maps.

When passing to $U$ and $V$, we have
\begin{align*}
	[\eta_{\t B}]: [\t B] &\to \Top(\Frm(\t B, \mathbb{2}), \mathbb{S})\\
	(b \stackrel{\t y}{\rightarrow} y) &\mapsto (p \to p(y))
\end{align*}
which when plugging in  $\t B = \mathbb{2}$ is equal to
\begin{align*}
	[\eta_{\mathbb{2}}]: [\mathbb{2}] &\to  \Top(\Frm(\mathbb{2}, \mathbb{2}), \mathbb{S})\\
	(b \stackrel{\t y}{\rightarrow} y) &\mapsto (1_{\mathbb{2}} \to 1_{\mathbb{2}}(y)),
\end{align*}
since $\Frm(\mathbb{2},\mathbb{2})$ only contains one map, the identity. So we have\begin{align*}
	[\eta_{\mathbb{2}}](\t y): \Frm(\mathbb{2},\mathbb{2}) &\to \mathbb{S}\\
	1_{\mathbb{2}} &\mapsto 1_{\mathbb{2}}(y)
\end{align*}
and now
\begin{align*}
	[[\eta_{\mathbb{2}}](\t y)]: \Frm(\mathbb{2},\mathbb{2}) &\to [\mathbb{S}]\\
	 1_{\mathbb{2}} &\mapsto (\{pt\} \to1_{\mathbb{2}}(y))
\end{align*}
so that 
\begin{align*}
	[[\eta_{\mathbb{2}}](\t y)](1_{\mathbb{2}}) = (\{pt\} \to 1_{\mathbb{2}}(y)).
\end{align*}
Now we can see that the map \begin{align*}
	\sigma: [\mathbb{2}] &\to [\mathbb{S}]\\
	(b \stackrel{\t y}{\rightarrow} y) &\mapsto (\{pt\} \to 1_{\mathbb{2}}(y))
\end{align*}
is the identity in $\Set$, as the underlying set on both sides is $[\mathbb{2}] = [\mathbb{S}] = \{0,1\}$ so that we are looking at the set map \begin{align*}
	\{0,1\} &\to \{0,1\}\\
	y &\mapsto 1_{\{0,1\}}(y)= y.
\end{align*}

This map is clearly the identity. This boils down to the fact that our choice of morphism from $\Frm(\mathbb{2},\mathbb{2})$ was easy to determine since the set $\Frm(\mathbb{2},B) = \{pt\}$, as $\mathbb{2}$ is an initial object in $\Frm$, and in particular it is clear that $\Frm(\mathbb{2},\mathbb{2}) = \{1_{\mathbb{2}}\}$. From this we can deduce that $\sigma$ is the identity on $[\mathbb{2}]$.

In general, our schizophrenic object will not necessarily be initial in arbitrary concrete duality, and as such, our choice $p \in \HomA(\t A, \t A)$ may not be unique nor easy to determine, so that $\sigma$, though always a bijection, is not necessarily always the identity.
 

\section{Stone Duality}

For the following section we assume a bit more familiarity with lattice theory than the mere definitions of objects and morphisms of the corresponding categories. To that effect, we assume familiarity with distributive lattices, sub join/meet-semilattices, (atomic) Boolean algebras,  ideals, filters, (principal) ultrafilters, (directed and codirected) posets, frames and locales.

Moreover, we follow the convention of \cite{zbMATH03940199} that a sub meet-semilattice is given by the datum $(L, \land, \top)$ and a sub join-semilattice is given by the datum $(L, \lor, \bot)$, so that a lattice is given by the datum $(L, \land, \lor, \top,\bot)$. There is good reason for this, as we want to consider the empty limit/colimit. Some authors do not necessitate this and choose to differentiate between $\DLat$ and $\DLat_{\text{bdd}}$, refering to what we call a distributive lattice as a bounded distributive lattice.

Before we begin we would like to call attention to the fact that the Stone duality is wrought with interesting subdualities, some of which are themselves examples. Though describing these subdualities may feel like a diversion, we find they actually elucidate the structure of the Stone duality as well as provide us with intuition for the following examples, so we plan to describe them.

Our strategy is as follows: first we describe a the dual adjunction between $\FinSet$ and $\Boolf$, which we will later see, can be obtained by restricting the Stone duality to full subcategories. However that is not how we will be constructing the Stone duality. 

We start with this example because it is easier to understand, gives us a candidate for the schizophrenic object of Stone, and provides some groundwork for the following dualities.

On the other hand, we will actually obtain the Stone duality via restriction and composition of the dualities  $\Top \rightleftarrows \Loc$, which we just saw, and  $\CohLoc \rightleftarrows \DLat^{\op}$, between the category of coherent locales and coherent maps between them, which we will later define, and the category of distributive lattices with lattice morphisms.

And finally we will show why the finite duality is a restriction of the Stone duality.

\subsection{The $\FinSet \rightleftarrows \Boolf^{\op}$ Duality}\label{sec:stone-duality}

Before we begin we would like to be clear about what we mean by subduality:



\begin{definition}[Subduality]
	Let $\mathcal{A}$ and $\mathcal{B}$ be two categories for which there exists a duality, i.e., an adjunction $\mathcal{A} \rightleftarrows \mathcal{B}^{\op}$, given by left and right adjoint functors $T: \mathcal{A} \to \mathcal{B}^{\op}$ and $S: \mathcal{B}^{\op} \to \mathcal{A}$, respectively.
	
	If $\mathcal{A}' \subseteq \mathcal{A}$ and $\mathcal{B}' \subseteq \mathcal{B}$ are subcategories on which $S$ and $T$ restrict to a duality, we call $\mathcal{A}' \rightleftarrows \mathcal{B}'^{\op}$ given by  $T \lvert_{\mathcal{A}'}: \mathcal{A}' \to \mathcal{B}'^{\op}$ and $S\lvert_{\mathcal{B}'}: \mathcal{B}'^{\op} \to \mathcal{A}'$ a \emph{\textbf{subduality}} of $\mathcal{A} \rightleftarrows \mathcal{B}^{\op}$. 
	
	If both $\mathcal{A}'$ and $\mathcal{B}'$ are full subcategories, we call the restriction a \emph{\textbf{full subduality}} of $\mathcal{A} \rightleftarrows \mathcal{B}^{\op}$. 

\end{definition} 

Notice that if $\mathcal{A} \rightleftarrows \mathcal{B}^{\op}$ is an equivalence, and $\mathcal{A}' \rightleftarrows \mathcal{B}'^{\op}$ is a full subduality, then it is also an equivalence,  as the unit and counit maps $\epsilon$ and $\eta$ remain natural isomorphisms, since all objects in the restriction, maps between them (fullness guarantees a map backwards), and naturality are all inherited from the ambient categories. This is however not a necessary condition, as we will see.

The first subduality we will discuss is a subduality of the Stone duality.

Consider the category $\Boolf$ of finite Boolean algebras and Boolean algebra homomorphisms between them. Let $\Ult(-)$ be the functor which sends a finite Boolean algebra $B$ to the set $\Ult(B)$ of ultrafilters on $B$ and Boolean algebra homomorphisms $B \stackrel{h}{\rightarrow} B'$ to the usual inverse image morphism $\Ult(B') \stackrel{h^{-1}}{\rightarrow}\Ult(B)$. A straightforward computation shows that $h^{-1}$ is well-defined.

Now let $\mathcal{P}(-)$ be the functor which sends a finite set $X$ to its power-set $\mathcal{P}(X)$ and set-maps $X \stackrel{f}{\rightarrow}X'$ to $\mathcal{P}(X') \stackrel{f^{-1}}{\rightarrow} \mathcal{P}(X)$, the usual inverse image morphism. 

We claim there is a dual adjunction between $\Boolf$ and $\FinSet$ given by $\Ult(-)$ and $\mathcal{P}(-)$:
\[\begin{tikzcd}
	 \FinSet& \bot & \Boolf^{\op}
	\arrow["{\mathcal{P}(-)}", shift left=2, bend left, from=1-1, to=1-3]
	\arrow["{\Ult(-)}", shift left=3, bend left, from=1-3, to=1-1].
\end{tikzcd}\]

Notice that $\mathcal{P}(X)$ is a Boolean algebra, where meets and joins are given by intersection and union, top and bottom elements by $X$ and $\emptyset$, and $\neg S := X - S$. The usual set theoretic laws shows the distributive property. \emph{As a side remark, this also holds for arbitrary sets, not just finite ones}.

Since finite ultrafilters are principal, points of $X$ correspond bijectively to ultrafilters on $\mathcal{P}(X)$, which are necessarily generated by singletons, so that the unit $X \to \Ult(\mathcal{P}(X))$, which sends $x \mapsto \{S \lvert \{x\} \subseteq S\}$, is an isomorphism.

On the other hand finite Boolean algebras are atomic, since every non-zero element is bounded below by finitely many non-zero elements, so there exists at least one minimal one, which we call an atom. Therefore ultrafilters in $\Boolf$ correspond exactly to the atoms. 

Given  $B \in \DLat$ and $a \in B$, let $\atom(B)$ be the set of atoms in $B$ and $\atom(a) := \{x \in \atom(B) \lvert x \leq a \}$. If $B \in \Bool_{\atom}$, one observes that  $\atom(a)=\atom(a')$ implies $a = a'$. Equivalently, every element of an atomic Boolean algebra is the join of atoms below it \cite{zbMATH03940199}. 

Since for distinct $b$ and $b'$, an atom below $b$ but not $b'$ generates an ultrafilter that separates them, the counit $B \to \mathcal{P}(\Ult(B))$ that sends $b \mapsto \{U \in \Ult(B) \lvert b \in U\}$ recovers  $b$ uniquely, and so it is an isomorphism. One needs only to check that the counit is indeed a Boolean algebra homomorphism, but this is straightforward.
%
%
%It is also the case that $\Ult(-)$ is  fully faithful as long as $A$ is a finite Boolean algebra, since ultrafilters on those are principle, meaning that they are generated by a single element, an element which we can then identify with the points of $X$ in a faithful way. 
%
%For finite power-sets, those generators are necessarily the singleton sets, so that the counit $X \to \Ult(\mathcal{P}(X)) $ of the adjunction is an isomorphism.
% 
%Similarly, $\mathcal{P}(-)$ is fully faithful , as sets have unique power sets up to isomorphisms, and their maps induce unique Boolean algebra homomorphisms (\textcolor{red}{TODO} show this), so that the unit $B \to \mathcal{P}(\Ult(B))$ is an isomorphism, and so the adjunction above is a dual equivalence.
%

To determine a candidate for the schizophrenic object first consider the free Boolean algebra on one generator $b$ is the set $\diamondsuit : =\{\bot, b, \neg b, \top \}$, since our generator needs to induce complements, as well as finite limits and finite colimits. On the other side, the free set on one generator is  $\{pt\}$. 

Now  an ultrafilter on $\diamondsuit$ is determined by the atoms $\{b, \neg b\}$, so that  $\Ult(\diamondsuit) = \mathbb{2}$, where we now view $\mathbb{2}$ as a $\Set$-object.\footnote{In this section we will see $\mathbb{2}$ and $\mathbb{S}$ in multiple categories, which will not always be reflected in the notation. The reader is advised to keep the context in mind.}

Meanwhile $\mathcal{P}(\{pt\}) = \{\emptyset, \{pt\}\} = \mathbb{2}$ as Boolean algebra under inclusion, since the point and the empty set are complements, and thus form the truth-value Boolean algebra which is an initial object of $\Bool$. 

 Now in the following we only want to see that a lift of $\Hom$-sets exists, and we check that by showing that the adjunction we gave serves as such a lift. 
 
For the power-set functor, it is easy to see that $\mathcal{P}(X) = \Set(X, \mathbb{2}) $ since any set map  $X \to \mathbb{2}$ is given uniquely by a subset $ S \subseteq X$, via characteristic functions \begin{equation*}
  \mathcal{X}_{ S}(x) = \begin{cases}
   1 & x \in  S \\
    0 & \text{else}.
  \end{cases}
\end{equation*}

On the other hand, we can obtain a $\Hom$-set description of $\Ult(-)$ via the following lemma:

\begin{lemma} The following map, where $\prim(-)$ sends a distributive lattice to its set of prime filters, is a bijection:
\begin{align*}
  	\phi': \prim(B) \to& \DLat(A, \mathbb{2})\\
  	F \mapsto& f_F &f_F(x) = \begin{cases}
   1 & x \in F \\
    0 & \text{else}
  \end{cases}
  \end{align*}

Furthermore, the map $\phi'$ restricts to a bijection $\phi$, when restricting $\DLat$ to $\Bool$:  
	\begin{align*}
  	\phi: \Ult(A) \to& \Bool(A, \mathbb{2})\\
  	F \mapsto& f_F &f_F(x) = \begin{cases}
   1 & x \in F \\
    0 & \text{else}
  \end{cases}
\end{align*}
\end{lemma} 
\begin{proof}
Obviously the map $f \mapsto f^{-1}(1)$ is an inverse, so we only check that these are well-defined.

The proof for $\phi'$ is straightforward and can be found in \cite{zbMATH03940199} at Proposition 2.2. It boils down to an equivalence between properties of lattice homomorphisms and prime filter axioms. That is, an order preserving, finite limit preserving, and finite colimit preserving morphism corresponds to a prime, upwards closed, sub meet-semilattice. 

Now when restricting to Boolean algebras (note that $\Bool$ is a \emph{full} subcategory of $\DLat$), prime filters correspond bijectively to ultrafilters, which, given the filter axioms, will additionally  require that $A \in F$ if and only if $\forall B \in F: B \cap A \neq \emptyset$. 

The property that a filter $F$ is prime is equivalent to $F$ being given above by $f^{-1}(1)$ for $f \in \DLat(A, \mathbb{2})$ which is further equivalent to the fact that its complement $I:= f^{-1}(0)$ is an ideal \cite{zbMATH03940199}. \emph{We write $\DLat$ here instead of $\Bool$ to highlight that we do not use complement preservation for this direction}.

So if $F$ is prime, then $\emptyset \notin F$ gives the forwards direction of the ultrafilter condition and for the other direction supposing $A \in I$, then $\neg A \in F $, but  $A \cap \neg A = \emptyset$ contradicting our assumption implying $A \in F$.

Conversely if $F$ is an ultrafilter, then it cannot contain the complement of any of its elements, which is just $\neg f(A) = f(\neg A)$, proving the bijection.
\end{proof}
  


%The ultrafilter condition implies that a filter cannot contain the complement of any of its elements, which is just $\neg f(A) = f(\neg A)$. And conversely to separate complements out of the filter via a lattice homomorphism gives the ultrafilter condition, since finite limit preservation and complementation implies $\emptyset \notin F := f^{-1}(1)$ for one direction, and for the other direction supposing $A \in I: = f^{-1}(0)$ then $\neg A \in F $, but  $A \cap \neg A = \emptyset$ contradicting our assumption implying $A \in F$. 


\emph{As a side remark, frame homomorphisms additionally require preservation of arbitrary colimits, which simply upgrades the prime condition to completely prime, i.e. $\bigvee a_i \in F$ implies $\exists i \in I: a_i \in F$.
}

Taking from tradition of topology, where $\{pt\}$ is a terminal object, and thus points of a topological space $X$ can be characterized by maps $\{pt\} \stackrel{f_x}{\rightarrow} X$, sending $\{pt\} \mapsto x$, we call any map $f \in \mathcal{L}(A, \mathbb{2})$ \textbf{\emph{a point}}, where $\mathcal{L}$ is any subcategory of $\DLat$, since finite limit and finite colimit preservation ensures that $\mathbb{2}$ is initial in $\mathcal{L}$.

Ultimately, what a point map  $A \stackrel{f}{\rightleftarrows} \mathbb{2}$ represents is the question of which subsets of $\mathcal{P}(A)$ are compatible with our desired structure restrictions:

\[\begin{tabular}{ |p{3cm}||p{3cm}|  }
 \hline
 \multicolumn{2}{|c|}{Correspondence} \\
 \hline
$\textbf{Set of points}$& $\textbf{Type of filter}$\\
 \hline
 \hline 
 $\DLat(A, \mathbb{2})$   &prime filter \\
 \hline
 $\Frm(A, \mathbb{2})$ &completely prime filter \\
 \hline 
 $\Bool(A, \mathbb{2})$    &ultrafilter \\
 \hline
\end{tabular}\]

Note that the unit and counit maps then follow the exact same logic as the leading example, and in particular, $\mathbb{2}$ is also initial in $\Bool$ so that by the same logic, our $\tau$ is equal to the set identity $1_{\mathbb{2}}$. 

 %\begin{lemma}
%	Let $I \subseteq P$ be an ideal of a lattice $P$. Then the following are equivalent 
%	\begin{enumerate}
%		\item The complement of $I$ is a filter $F \subseteq P$,
%		\item $1 \notin I$ and $I$ is prime.
%		\item $I$ can be given as the kernel of a lattice homomorphism $F \stackrel{f}{\rightleftarrows} \mathbb{2}$
%	\end{enumerate}
%\end{lemma}
%\begin{proof}
%	$(1) \implies (2)$: Since the complement of $I$ is a filter $F$, which contains the top element by upwards closure, we know the top element can't lie in $I$. Furthermore, $a \land b \in I$ implies that $a \in I$ or $b \in I$ by the fact that a filter is a sub meet-semilattice (if both $a, b \in F$ then so is $a \land b \in F$).
%	
%	$(2) \iff (3)$: One can check that  $f(a) =  \begin{cases}
%   1 & a \notin I \\
%    0 & a \in I
%  \end{cases}$ is a lattice homomorphism, and that the kernel of a lattice homomorphism defines a prime ideal. In fact, we will show this equivalence in the following lemma, but for frame homomorphisms. 
%  
%  \emph{Note that a lattice homomorphism is an order preserving, finite limit and finite colimit preserving morphism between lattices, while a frame homomorphism asks extra that it preserves arbitrary colimits, which we will see, upgrades the condition on the ideal (and dually to the filter) from being prime to being completely prime.}
%  
%  $(2) \& (3) \implies (1)$:  For ease we call $I = f^{-1}(0)$ and $F:= f^{-1}(1)$, and we implicitly use the fact that these are by definition set complements. Notice that the condition of being a prime ideal $a \land b \in I \implies a \in I$ or $b \in I$ is equivalent by contraposition to $a, b \in F \implies a \land b \in F$, so that $f^{-1}(1)$ is a sub meet-semilattice.
%  
%  Furthermore upwards closure of $F$ is derived directly from downwards closure of $I$. In other words, assuming $a, b \in P$ such that $a \leq b$, we have $a \in F \implies b \in F$ if and only if $ b \in I\implies a \in I$, which holds since $I$ is downwards closed.
%\end{proof}
%\begin{lemma}
%	There is a bijection \begin{align*}
%  	\phi: \Ult(A) \to& \Bool(A, \mathbb{2})\\
%  	F \mapsto& f_F &f_F(x) = \begin{cases}
%   1 & x \in F \\
%    0 & \text{else}
%  \end{cases}
%  \end{align*}
%  This restricts to a bijection $\phi': \prim(A) \to \Frm(A, \mathbb{2})$, where $\prim(A)$ is the set of all completely prime filters of $A$.
%\end{lemma}
%
%
% 
%\begin{proof}
% 	Our strategy will be to first identify  points of a frame $F \rightleftarrows \mathbb{2}$  with  completely prime filters on $A$ and then to show the ultrafilter condition in the Boolean setting. 
% 	
% 	First we show that $f_F$ is finite limit, colimit, and order preserving. 
%  
% That $F$ is upwards closed is the condition that $x \in F$, $y \in A \implies y \in F$. This implies that
%  \begin{align*}
% 	f_F(x \wedge y) = 1 &\iff x \wedge y \in F\\ &\iff x \in F\text{ and } y \in F\\ &\iff f_F(x) \wedge f_F(y) =1 \
% \end{align*}
% 
% 
%  That $F$ is a sub meet-semilattice is the condition $x, y \in F \implies x \land y \in F$. This implies that. \begin{align*}
% 	f_F(x \wedge y) = 0 &\iff x \wedge y \notin F\\ &\iff x \notin F\text{ or } y \notin F\\ &\iff f_F(x)  \wedge f_F(y) =0 \
% \end{align*}
%
%
%That $F$ is a completely prime filter is the condition that $\bigvee x_i \in F$ implies that there exists an $i \in I$ such that $x_i \in F$, which implies that \begin{align*}
%	f_F( \bigvee x_i) = 1 &\iff \bigvee x_i \in F\\ &\iff \exists i \in I : x_i \in F\\
%	&\iff \bigvee f_F(x_i) = 1
%\end{align*}
%and similarly \begin{align*}
%	f_F( \bigvee x_i) = 0 &\iff \bigvee x_i \notin F\\ &\iff \forall i \in I:x_i \notin F \\
%	&\iff \bigvee f_F(x_i)  = 0
%\end{align*}
%Consider that all these statements are actually equivalences, since you can obtain each condition by swapping the two way implications from what was our condition---the second two way implication of each of these chains of equivalences---to what we have proven---each chains end statement.
%
%However note that when we swap we must think about $f_F$ as arbitrary Boolean algebra (or frame) homomorphism $f$, and consider $I$ and $F$ as nothing but pre-images under $F$, for example,
%
%\begin{align*}
%	\bigvee x_i \in f^{-1}(1)&\iff f( \bigvee x_i) = 1 \\ 
%	&\iff \bigvee f(x_i) = 1 \\
%	&\iff \exists i \in f^{-1}(0) : x_i \in f^{-1}(1)
%\end{align*}
%
%To check that $f_F$ is order preserving, consider that the only type of map that contradicts the condition is $ 1 =f_F(a) \leq f_F(b) = 0$, so we only need to check that for any $a \in F$ and $b \in P$ with $a \leq b$ it holds that $b \in F$. But this is exactly the condition of being upwards closed. 
%
%This shows that $\phi'$ is well defined and has an inverse. 
%
%Now to conclude the proof we check that the ultrafilter condition is equivalent to a complement preserving point of a Boolean algebra $F \rightleftarrows \mathbb{2}$.
%
%Having proven the equivalences for filter conditions, we note that for filters, the condition of being an ultrafilter of a set $S$ means that for $A, B \subseteq S$ it holds that $A \in F \iff \forall B \in F: A \cap B \neq \emptyset $. 
%
%
% Now consider that $F \subseteq \mathcal{P}(A)$ as a poset, where the empty set is an initial object, but $S \in F$ implies $A - S \notin F$ due to the ultrafilter condition. But this just means that $f_F(\neg S) = \neg f_F(S)$. 
% 
% For the converse case, the forwards implication of the ultrafilter condition is given by the fact that $F$ is a sub meet-semilattice and $\emptyset \notin F$. 
% 
% 
% 
% 
%% Moreover, it suffices to prove the following claim: for all $B \in F$ it holds that  $B \cap A \neq \emptyset \iff B \cap A \in F$ for all $B \in F$, as this claim  implies the backwards condition of the ultrafilter lemma.
%% 
%% 
%% Consider that the previous lemma implies that $\emptyset \notin F$, from which "$\impliedby$" follows immediately.
%% 
%%\textcolor{blue}{TODO} show $B \cap A \neq \emptyset \implies B \cap A \in F$. 
%
%For the backwards implication, if we assume $a \in I$ then we have $a \land \neg a = \emptyset $, which is a contradiction to the fact that for all $ B \in F$ it holds that $ A \cap B \neq \emptyset$. Therefore $A \in F$. \end{proof}



\subsection{Subdualities of $\Top \rightleftarrows \Frm^{\op}$ and the $\CohLoc \rightleftarrows \DLat^{\op}$ Duality}
In the following we will want to give a general overview of how we obtain the Stone duality.

We start by focusing on the following dual adjunction

\[\begin{tikzcd}
	\Top & \bot & \Loc
	\arrow["{\Omega(-)}", bend left, from=1-1, to=1-3]
	\arrow["{\pt(-)}", bend left, from=1-3, to=1-1]
\end{tikzcd}\]
 
 which is  the duality of our leading example. This is indeed given by $\pt(A) := \Frm(A, \mathbb{2})$, and $\Top(X, \mathbb{S}) =: \Omega(X)$, which is the frame of opens on $X$, since a continuous map into the Sierpiński space is  uniquely determined by the pre-image of $\{1\} \subseteq \mathbb{S}$,  corresponding to a unique open set of $X$. 
 
 Indeed, we may define the category of \emph{\textbf{spatial locales}} and of \emph{\textbf{sober spaces}} respectively to be the largest subcategories of $\Loc = \Frm^{\op}$ and $\Top$, such that the adjunction is an equivalence. So now we have 
 
 \[\begin{tikzcd}
	\Sob & \stackrel{\cong}{\bot} & \Spat
	\arrow["{\Omega(-)}", bend left, from=1-1, to=1-3]
	\arrow["{\pt(-)}", bend left, from=1-3, to=1-1]
\end{tikzcd}\]

where the unit and counit maps are given by\footnote{The notation $\downarrow (b)$ means the principal ideal generated by $b$} \begin{align*}
	X \to& \pt(\Omega(X))  & \Omega(\pt(B)) \to & B\\
	x \mapsto& \downarrow (X-\{x\}) & \{ p \in \pt(B) \lvert p(a) = 1\} \mapsto& a .
\end{align*}
In a sober topological space, points of $\Omega(X)$ correspond to prime elements, i.e. elements generating prime principal ideals, whose complements are the irreducible closed subsets of $X$, where sobriety is just the condition that singletons are the only irreducible closed subsets of $X$ [Joh 1.3].

Spatial locales on the other hand map distinct elements to distinct sets of points. In other words, the counit is an isomorphism if and only if for all $a, b \in B : a \not\leq b$ implies the existence of some $ p \in \pt(B)$ such that $ p(a) = 1$ and $ p(b) = 0$.  

Next we show that  dual adjunction further restricts to the duality $\CohTop	\rightleftarrows \CohLoc$, whose categories we define in  the following:

Let $B$ be a locale. Then 
\\
\begin{definition}
	We call an element $b \in B$ \textbf{finite}, if for all $S \subseteq A$ such that $\bigvee S \geq b$ there exists a finite $F \subseteq S$ such that $\bigvee F \geq b$. 
\end{definition}

This definition is equivalent to the following two statements \cite{zbMATH03940199}:
\begin{enumerate}
	\item For all directed $S \subseteq B$ such that $\bigvee S \geq b$ there exists $s \in S$ such that $s \geq b$.
	\item For all ideals $I \subseteq B$ such that $\bigvee I \geq b$ it holds that $b \in I$. 
\end{enumerate}
Notice that if $B$ is spatial, which implies the existence of a topological space $X = \pt(B)$ such that $B = \Omega(X)$, then the finite elements are precisely the compact open subsets of $X$. 
\\
\begin{definition}
	We call $B$ \textbf{coherent} if the following conditions hold: \begin{enumerate}
		\item Every element $b \in B$ can be given as a join of finite elements
		\item The finite elements $K(B)$ make up a sublattice of B. 
	\end{enumerate}

\end{definition}

Since $K(B)$ is a sub join-semilattice of $B$ \cite{zbMATH03940199}, the second condition is equivalent to $1 \in K(B)$ and $K(B)$ is closed under finite meets.

If $B$ is spatial, this means that $K(B)$ forms a basis for the topology on $X$. In fact this is an equivalence: coherent locales are spatial \cite{zbMATH03940199}.

On the other hand, any sober topological space $X$ can be given as $X \cong \pt(\Omega(X))$, but are not necessarily generated by compact elements. This motivates the following definition:

Let $X$ be a sober space. 
\\
\begin{definition}[Coherent space]
	If $K(\Omega(X))$ is closed under finite intersection and generates the topology of $X$, we call $X$ a \textbf{coherent topological space}.
\end{definition} 


Given $A, B \in \CohLoc$, we know that any lattice homomorphism $K(A) \to K(B)$ freely extends to a frame homomorphism $A \to B$, as coherent locales are freely generated by their finite elements by definition, however the converse is not true, frame homomorphisms do not necessarily preserve finiteness.

In topological terms, that translates to the fact that the pre-image of compact open subsets under continuous maps is not necessarily compact.

So we may define a locale map $B \stackrel{f}{\to} A$ between coherent locales to be a \emph{\textbf{coherent map}} if $f^*$ maps $K(A)$ to $K(B)$. Similarly a  continuous map $B \stackrel{f}{\to} A$ is coherent if $f^{-1}(K\Omega(A)) \subseteq  K\Omega (B)$.

Unlike the other examples, that means the restrictions of $\Spat$ to $\CohLoc$ and $\Sob$ to $\CohTop$ are not full.

However, the structure of our categories ensures that the inverse maps in $\Sob$ and $\Spat$ that make the unit and counit isomorphisms stay preserved after restriction. That means for any coherent map $B \stackrel{f}{\rightarrow} A$ which is also an isomorphism, the inverse $f^{-1}$ is also coherent. The proof is straight-foward and left to the reader. 

%This means we have to at the very least redefine $\pt(-)$ to be a lift of $ \CohLoc(-,\mathbb{2})$ (\textcolor{red}{TODO}: unless $\Frm(B,\mathbb{2})=\CohLoc^{\op}(B, \mathbb{2})$?)
%Thus we want to check that our functor $\pt(-)$ is unperturbed, or in other words, that $\Frm(B, \mathbb{2})=\CohLoc(B, \mathbb{2})$ if $B$ is a coherent locale.
%
%Or that $\Frm(B, \mathbb{2})$ restricted to the subset $\CohLoc(B, \mathbb{2})$ commutes with taking the

Therefore we have the following dual equivalence

 \[\begin{tikzcd}
	\CohTop & \stackrel{\cong}{\bot} & \CohLoc
	\arrow["{\Omega(-)}", bend left, from=1-1, to=1-3]
	\arrow["{\pt(-)}", bend left, from=1-3, to=1-1]
\end{tikzcd}\]

We should be careful regarding how this affects our interpretation of the schizophrenic object of this adjunction. 

The pre-image of coherent topological maps must preserve compact opens, and $\{1\}$ is a compact open set in $\mathbb{S}$, so that $\CohTop(X, \mathbb{S}) = K \Omega(X) \neq \Omega(X)$, since coherent topological spaces are only generated by compact opens, showing us that the schizophrenic object is not preserved under this restriction. 

Strictly speaking, this adjunction does not have a schizophrenic object, since we cannot obtain the opens by lifting $\Hom$-sets anymore. However in practice we still consider subdualities of dualities induced by a schizophrenic object to be concrete dualities.

Now we will want to compose. Consider the fact that a locale is coherent if and only if it is isomorphic to the locale of ideals of a distributive lattice \cite{zbMATH03940199}. This gives us a functor $\Idl(-): \DLat \to \CohLoc$, such that $\Idl(B \stackrel{f}{\rightarrow} B') = (\Idl(B') \stackrel{f^{-1}}{\rightarrow}\Idl(B)$). A similar computation as with $\Ult(-)$, but with fewer conditions,  shows that the pre-image morphism $f^{-1}$ is well defined.

We can also view $K(-)$ as a functor by sending frame homomorphisms to their restrictions on finite elements, which we can do since, as we have seen above, we have defined $\CohLoc$ to be exactly the category where such a restriction is well defined.

The unit and counit maps \begin{align*}
	B \to& K\Idl(B) & A \to & \Idl(K(A))\\
	b \mapsto& \{ (b) \lvert b \in B\} & a \mapsto& \{k \in K(A) \lvert k \leq a\}
\end{align*}
are well-defined isomorphisms [Joh 3.2].

That is to say that the finite elements of $\Idl(B)$ are exactly the principal ideals, and the ideals of $K(A)$ are uniquely determined by elements of $A$; they consist of all finite elements which sit under each $a \in A$. 

This shows us that we have a dual equivalence

\[\begin{tikzcd}
	\CohLoc & \stackrel{\cong}{\bot} & \DLat^{\op}
	\arrow["{K(-)}", bend left, from=1-1, to=1-3]
	\arrow["{\Idl(-)}", bend left, from=1-3, to=1-1].
\end{tikzcd}\]
\subsection{Schizophrenic Object of $\CohTop \rightleftarrows \DLat^{\op}$}
%In particular, this is also a natural dual adjunction. 
Now we can compose these dual equivalences $\CohTop \rightleftarrows \CohLoc \rightleftarrows \DLat^{\op}$ and it will give us a dual equivalence. One side sends a distributive lattice $A$ to what we call its \emph{\textbf{spectrum}}, in other words, $\Spec(A) := \pt(\Idl(A))$. On the other side we shall send a coherent space $X$ to its lattice of compact open subsets $K(\Omega(X))$.  So we have the dual adjunction

\[\begin{tikzcd}
	\CohTop & \stackrel{\cong}{\bot} & \DLat^{\op}
	\arrow["{K\Omega(-)}", bend left, from=1-1, to=1-3]
	\arrow["{\Spec(-)}", bend left, from=1-3, to=1-1].
\end{tikzcd}\]
%, a natural candidate for schizophrenic object might be $\Omega(\mathbb{S}) = \bot - b - \top$, which we shall call $ \mathbb{S}_{\Lat}$, whose elements are all finite, so that  $K(\Omega(\mathbb{S}))$ leaves $ \Omega(\mathbb{S})$ unchanged.
%
%We see now that $K(X) = \CohLoc(X, \mathbb{S}_{\Lat})$, as coherent frame homomorphisms are precisely the ones which preserve compact opens, which are uniquely given by the pre-image of $\{1\}$. 
%
%Moreover,  $\Idl(B) = \DLat(B, \mathbb{S}_{\Lat}), $ 
%



We claim that this  adjunction is a natural dual adjunction given by the schizophrenic object $(\mathbb{S}, 1_{\mathbb{S}},\mathbb{2})$.

It is worth remarking that the reason  $\CohTop \rightleftarrows \CohLoc$ was not natural was because $\CohTop(X, \mathbb{S}) $ gave us $K\Omega(X)$ and not $\Omega(X)$. So intuitively, one could think of this composition as precisely what what needs to do to preserve the schizophrenic object.

On the one hand, Lemma 7.1 shows us that $\DLat(B, \mathbb{2})$ lifts to $\prim(B)$. Indeed, $\Spec(B) = \Frm(\Idl(B), \mathbb{2})$, which gives us completely prime filters of $\Idl(B)$, and these correspond precisely to $\prim(B)$, which follows from 3.4 in \cite{zbMATH03940199}. 

Now, just like in the leading example, we want to give $\Spec(B)$ the weakest coherent topology such that each evaluation map $\DLat(B,\mathbb{2}) \stackrel{\ev_{B,b}}{\rightarrow}[\mathbb{S}]$ lifts to a coherent map $e_{B,b}$. So we give $\Spec(B)$ the topology $
	\{ \ \{ f \in \DLat(B, \mathbb{2})  \lvert f(b)  = 1 \} \lvert  b \in B \}$
and check that its elements are compact in $\Spec(B)$. 

Each evaluation map $\DLat(B,\mathbb{2}) \stackrel{\ev_{B,b}}{\rightarrow}[\mathbb{S}]$ corresponds to the subset of $\Spec(B)$ given by \begin{align*}
	e_{B,b}^{-1}(\{1\}) = \{ f \in \DLat(B, \mathbb{2})  \lvert f(b)  = 1 \} =& \{F \in \prim(B) \lvert b \in F \}\\ =& \{ I \in \primIdl(B) \lvert b \notin I\} =: D(b)
\end{align*} 

Consider the cover $D(b) \subseteq \bigcup_I D(a_i)$. This is equivalent to saying there is no prime ideal $I$ containing all $a_i$ and $b$, which is equivalent (by Zorn's lemma)  to saying $1 \in (\{a_i\} \cup \{b\})$, i.e. the ideal generated by $a_i$ and $b$ is the entire spectrum. By downwards closure and closure under finite meets, there exists a finite index $m \in \mathbb{N}$ such that $1 \leq \bigvee_{i=0}^m a_i$ from which it follows $D(b) \subseteq \bigcup_{i = 0}^m D(a_i)$. Thus all $D(b)$ are compact and so is $\Spec(B)$.


Now we can say that the initial coherent topology on $\Spec(B)$ making all evaluation maps coherent is exactly the one where $\{ D(b) \}_{b \in B}$ forms its topology. Just like in the leading example, one can check that any map $Z \stackrel{h}{\rightarrow} \Spec(B)$ is coherent if and only if $\ev_{B,b} \circ h$ is. 

On the other hand, maps in $\CohTop(X, \mathbb{S})$ can be uniquely determined by which compact open the pre-image maps $\{1\}$  to, so that $\CohTop(X, \mathbb{S})$ lifts to $K\Omega(X)$ as desired. Furthermore, we have already seen that the evaluations $\CohTop(X,\mathbb{S}) \stackrel{\ev_{X,x}}{\rightarrow}[\mathbb{2}]$ lift to distributive lattice homomorphisms $d_{X,x}$ if any only if $d_{X,x}$ preserves pointwise order in $\mathbb{2}$, so that an initial distributive lattice structure is obtained by pointwise inclusion. This is the exact same argumentation from the leading example.


\subsection{Schizophrenic Object of the Stone duality}
Now we aren't done yet. To obtain the Stone duality, we will  restrict the above equivalence. 

Seeing that $\Spec(A)$ is Hausdorff if and only if $A$ is a Boolean algebra \cite{zbMATH03940199}, we can restrict this equivalence to a duality between $\Bool$ and the category of coherent, Hausdorff spaces which are called \emph{\textbf{Stone spaces}}. It is straight-forward to see that $\Bool \subseteq \DLat$ and $\Stone \subseteq \CohTop$ are full subcategories. 

A Stone space is equivalent to a compact, Hausdorff, totally disconnected space \cite{zbMATH03940199} making $\Stone$ a subcategory of $\kHaus$. Moreover, if $X$ is Hausdorff, then $K(\Omega(X)) = \clop(X)$, since compact sets are exactly the closed sets of a compact Hausdorff space.

Therefore we have the following dual equivalence
\[\begin{tikzcd}
	\Stone & \stackrel{\cong}{\bot} & \Bool^{\op}
	\arrow["{\clop(-)}", bend left, from=1-1, to=1-3]
	\arrow["{\Spec(-)}", bend left, from=1-3, to=1-1]
\end{tikzcd}\]
which we call the \emph{\textbf{Stone duality}}.

In the following we will show that the Stone duality is a natural dual adjunction. \emph{It's interesting and not necessarily expected that we obtain a natural dual adjunction via restriction of another natural dual adjunction.}

Recall our triple $(\mathbb{2}, 1_{\mathbb{2}}, \mathbb{2})$ from the finite setting. To determine if this triple is a schizophrenic object, we ask ourselves: \begin{enumerate}
 	\item Do the $\Hom$-sets lift?
 	\begin{enumerate}
 	\item Can we give the spectrum as a lift of the set of points of a Boolean algebra, i.e. $[\Spec(B)] = \Bool(B, \mathbb{2})$?
 	\item  Can we give the Boolean algebra of clopens of a Stone space as a lift of the set of continuous maps over compact Hausdorff spaces, i.e., does $[\clop(X)] = \Stone(X, \mathbb{2})$?
 	\end{enumerate}
 	\item Are these lifts initial?


 \end{enumerate}

For 1.(a) recall that Lemma 7.1 told us that the points of a boolean algebra correspond to its ultrafilters, which are equivalent to prime filters in the Boolean setting.


Since Boolean algebras automatically have Hausdorff spectrums, this lift is initial by the same argument as above. Nothing else has changed, we check in the same way that any map from a Stone space $[Z] \stackrel{h}{\rightarrow} \Spec(B)$ lifts to a coherent map if and only if its composites $\ev_{B,b} \circ h$ do. 

In fact, it is enough to check continuity, since continuous maps between compact Hausdorff spaces must be coherent, as continuity implies that the pre-image of  clopens are clopen, and in $\kHaus$ they are precisely the compact opens. This also means that $\Stone$ is a full subcategory of $\Top$. 

1.(b) is only slightly more complicated. We start by remarking that $\mathbb{S} \notin \Stone$, so to pass the schizophrenic object downwards, we will want to generate a Stone topology.

%Stonefy it via the Stone space coreflection functor $\Stone(-)$, which is also called the Stoneification, i.e. the right adjoint to the forgetful functor (\textcolor{red}{TODO} find source to show this exists).
%
%  By the dual equivalence we obtain the corresponding Boolean algebra coreflection functor $\Bool(-)$, or Booleanification.
 
That is, we take the lattice of compact open subsets of a space $X$, which we will call $A$. Let $A^*$ be the set of complements in $\mathcal{P}(X)$ of members of $A$. Then the patch topology, which has as a base $C = \{U \cap V \lvert U \in A, V \in A^*\}$ , will be a Stone space (notice that we are taking the Boolean completion $B$ of $A$ and then taking its spectrum) \cite{zbMATH03940199}. We remark that this argument also invokes the Axiom of Choice.
 
 This was just to see that we can in some sense "Stoneify" the $\Hom$-set functor via its schizophrenic object: we take the Boolean completion of $\mathbb{S}$ in $\mathcal{P}(\{0,1\})$ which is simply $\mathbb{2}_s$, the two point discrete space, since by Booleanifying we must generate the complement of $\{1\}$.
 
The argument for existence of lifts however works the same way as it would if $\mathbb{S}$ was in $\Stone$: $\CohTop(X, \mathbb{S})$ lifts to $\clop(X)$ if $X$ is a Stone space, since coherent maps into $\mathbb{S}$ are uniquely determined by the pre-image of $\{1\}$, which are simply the compact open subsets of a compact Hausdorff space, which are its  clopens. 

By the same argument we have $\Stone(X, \mathbb{2}_s) = [\clop(X)]$, we only remark that continuous maps $X \stackrel{f}{\rightarrow}\mathbb{2}_s$ are still completely determined by the pre-image of $\{1\}$, since that pre-image sends $\{0\}$ to its unique complement. 

To understand the Boolean algebra structure on $\Stone(X, \mathbb{2}_s)$ we consider all $n$-ary operations $\mathbb{2}_s^n \stackrel{\gamma}{\to} \mathbb{2}_s$, where $\mathbb{2}_s^n$ is given the product topology on $\mathbb{2}_s$. We get an induced map \begin{align*}
	\Stone(X, \mathbb{2}_s)^n \cong \Stone(X, \mathbb{2}_s^n) \stackrel{\gamma \circ -}{\to} \Stone(X, \mathbb{2}_s)
\end{align*}
where the first isomorphism is the fact that $\Hom$-sets preserve limits in the second argument. 

Indeed, since $\mathbb{2}_s$ is discrete, every set function $[X] \to [\mathbb{2}]$ lifts to a continuous map, so that $\Stone(X, \mathbb{2}) \cong [\mathbb{2}]^X$. As usual, it is completely natural to define Boolean  operations on $\Stone(X,\mathbb{2}_s)$ pointwise in $\mathbb{2}$, which turns the set product structure on $\Stone(X,\mathbb{2})$ into  the  Boolean algebra  product $\mathbb{2}^X$, as the underlying set functor $V: \Bool \to \Set$ reflects limits.

Remember that the categorical product comes with projection maps $\pi_x: \mathbb{2}^X \stackrel{\pi_x}{\rightarrow } \mathbb{2}_x$. Notice that these \emph{are} the evaluation maps, since products in $\Set$ are just functions out of the indexing set and projections pick out coordinates for these functions, which is exactly what our evaluation maps are doing.

Then initiality is just the universal property of the product: any map $B \stackrel{h}{\rightarrow} \mathbb{2}^X$ is Boolean if and only if the composite with projections $\pi_x \circ h$ are Boolean, showing us that the evaluation maps admit initial lifts into $\Bool$.
%\[\begin{tikzcd}
%	&& \Loc & \bot & \Top \\
%	&& \vdash && \vdash \\
%	&& \Spat & {\stackrel{\cong}{\bot}} & \Sob \\
%	&& \vdash && \vdash \\
%	\DLat & \bot & \CohLoc & \bot & \CohTop \\
%	\vdash &&&& \vdash \\
%	\Bool && \bot && \Stone
%	\arrow["{\pt(-)}", bend left, from=1-3, to=1-5]
%	\arrow["{\Spatt(-)}", bend left, from=1-3, to=3-3]
%	\arrow["{\Omega(-)}", bend left, from=1-5, to=1-3]
%	\arrow["{\Sobb(-)}", bend left, from=1-5, to=3-5]
%	\arrow[hook', bend left, from=3-3, to=1-3]
%	\arrow["{\pt(-)}", bend left, from=3-3, to=3-5]
%	\arrow["{\Coh(-)}", bend left, from=3-3, to=5-3]
%	\arrow[hook', bend left, from=3-5, to=1-5]
%	\arrow["\Omega(-)", bend left, from=3-5, to=3-3]
%	\arrow["{\Coh(-)}", bend left, from=3-5, to=5-5]
%	\arrow["{\Idl(-)}", bend left, from=5-1, to=5-3]
%	\arrow["{\Bool(-)}", bend left, from=5-1, to=7-1]
%	\arrow[hook', bend left, from=5-3, to=3-3]
%	\arrow["{K(-)}", bend left, from=5-3, to=5-1]
%	\arrow["{\pt(-)}", bend left, from=5-3, to=5-5]
%	\arrow[hook', bend left, from=5-5, to=3-5]
%	\arrow["{\Omega(-)}", bend left, from=5-5, to=5-3]
%	\arrow["{\Stone(-)}", bend left, from=5-5, to=7-5]
%	\arrow[hook', bend left, from=7-1, to=5-1]
%	\arrow["{\Spec(-)}", bend left, from=7-1, to=7-5]
%	\arrow[hook', bend left, from=7-5, to=5-5]
%	\arrow["{\clop(-)}", bend left, from=7-5, to=7-1]
%\end{tikzcd}\]
%
%\[\begin{tikzcd}
%	\Top & \bot & \Loc \\
%	\vdash && \vdash \\
%	\Sob & {\stackrel{\cong}{\bot}} & \Spat \\
%	\vdash && \vdash \\
%	\CohTop & \bot & \CohLoc & \bot & \DLat^{\op} \\
%	\vdash &&&& \vdash \\
%	\Stone && {\stackrel{\cong}{\bot}} && \Bool^{\op}
%	\arrow["{\Omega(-)}", bend left, from=1-1, to=1-3]
%	\arrow["{\Sobb(-)}", bend left, from=1-1, to=3-1]
%	\arrow["{\pt(-)}", bend left, from=1-3, to=1-1]
%	\arrow["{\Spatt(-)}", bend left, from=1-3, to=3-3]
%	\arrow[hook', bend left, from=3-1, to=1-1]
%	\arrow["\Omega(-)", bend left, from=3-1, to=3-3]
%	\arrow["{\Coh(-)}", bend left, from=3-1, to=5-1]
%	\arrow[hook', bend left, from=3-3, to=1-3]
%	\arrow["{\pt(-)}", bend left, from=3-3, to=3-1]
%	\arrow["{\Coh(-)}", bend left, from=3-3, to=5-3]
%	\arrow[hook', bend left, from=5-1, to=3-1]
%	\arrow["{\Omega(-)}", bend left, from=5-1, to=5-3]
%	\arrow["{\Stone(-)}", bend left, from=5-1, to=7-1]
%	\arrow[hook', bend left, from=5-3, to=3-3]
%	\arrow["{\pt(-)}", bend left, from=5-3, to=5-1]
%	\arrow["{K(-)}", bend left, from=5-3, to=5-5]
%	\arrow["{\Idl(-)}", bend left, from=5-5, to=5-3]
%	\arrow["{\Bool(-)}", bend left, from=5-5, to=7-5]
%	\arrow[hook', bend left, from=7-1, to=5-1]
%	\arrow["{\clop(-)}", bend left, from=7-1, to=7-5]
%	\arrow[hook', bend left, from=7-5, to=5-5]
%	\arrow["{\Spec(-)}", bend left, from=7-5, to=7-1]
%\end{tikzcd}\]
%
%\[\begin{tikzcd}
%	\Top & \bot & \Loc & \top  & {\DLat^{\op}} \\
%	\vdash && \vdash \\
%	\Sob & \stackrel{\cong}{\bot} & \Spat  \\
%	\dashv && \vdash \\
%	\CohTop & \stackrel{\cong}{\bot} & \CohLoc & \stackrel{\cong}{\bot} & {\DLat^{\op}} \\
%	\vdash &&&& \vdash \\
%	\Stone && \stackrel{\cong}{\bot} && \Bool^{\op} \\
%	\\
%	\\
%	\\
%	\FinSet && \stackrel{\cong}{\bot} && {\Boolf^{\op}}
%	\arrow["{\Omega(-)}", bend left, from=1-1, to=1-3]
%	\arrow["\Sobb(-)", bend left, from=1-1, to=3-1]
%	\arrow["{\Coh(-)}"', bend right, from=1-1, to=5-1]
%	\arrow["{\pt(-)}", bend left, from=1-3, to=1-1]
%	\arrow["{U(-)}", bend left, hook, from=1-3, to=1-5]
%	\arrow["{\Spatt(-)}", bend left, from=1-3, to=3-3]
%	\arrow["{\Coh(-)}", bend left, from=1-3, to=5-3]
%	\arrow["{F(-)}", bend left, from=1-5, to=1-3]
%	\arrow["{(\Coh \circ F)(-)}", from=1-5, to=5-5]
%	\arrow["{U(-)}", bend left, hook', from=3-1, to=1-1]
%	\arrow["{\Omega(-)}", bend left, from=3-1, to=3-3]
%	\arrow["{U(-)}", bend left, hook', from=3-3, to=1-3]
%	\arrow["{\pt(-)}", bend left, from=3-3, to=3-1]
%	\arrow["{U(-)}"', bend right, hook', from=5-1, to=3-1]
%	\arrow["{\Omega(-)}", bend left, from=5-1, to=5-3]
%	\arrow["{\Stone(-)}", bend left, from=5-1, to=7-1]
%	\arrow["{U(-)}", bend left, hook', from=5-3, to=3-3]
%	\arrow["{\pt(-)}", bend left, from=5-3, to=5-1]
%	\arrow["{K(-)}", bend left, from=5-3, to=5-5]
%	\arrow["{\Idl(-)}", bend left, from=5-5, to=5-3]
%	\arrow["{\Bool(-)}", bend left, from=5-5, to=7-5]
%	\arrow["{U(-)}", bend left, hook', from=7-1, to=5-1]
%	\arrow["{\clop(-)}"', bend left, from=7-1, to=7-5]
%%	\arrow[from=7-1, to=11-1]
%	\arrow["{U(-)}", bend left, hook', from=7-5, to=5-5]
%	\arrow["{\Spec(-)}"', bend left, from=7-5, to=7-1]
%%	\arrow[from=7-5, to=11-5]
%	\arrow["{\mathcal{P}(-)}"', bend left, from=11-1, to=11-5]
%	\arrow["{\Ult(-)}"', bend left, from=11-5, to=11-1]
%\end{tikzcd}\]
\subsection{Restricting Stone to the Finite Case}
Now we can revisit the finite dual equivalence from the beginning of this section. Any finite Stone space is discrete, since Hausdorffness forces singletons to be clopen: we give singletons by an intersection of all point-separating clopens, of which there are finitely many, so that singletons are themselves clopen. This shows that $\FinStone = \FinSet$, noting that the power-set of finite discrete topological spaces return its clopen sets. It is thus clear that the Stone duality restricts to the finite duality of \ref{sec:stone-duality}. 

As a side remark, one may even extend  $\FinSet^{\op} \rightleftarrows \Boolf$  via $\Ind$ and $\Pro$ categories, which are beyond the scope of this text, to obtain the Stone duality, and in particular, an equivalence between the category of Stone spaces and the category of pro-finite sets: 
\begin{align*}
	\Stone \cong \Bool^{\op} \cong \Ind(\Boolf)^{\op} \cong \Ind(\FinSet^{\op})^{\op} \cong  \Pro(\FinSet).
\end{align*}
For further reading refer to Chapter VI. in \emph{Johnstone's Stone Spaces}. 
\section{Rings and Affine Schemes}

We now turn to an example that should be familiar to many graduate level students of Algebra: the duality between the category of rings and the category of affine schemes. We will use the more general category of commutative $R$-algebras, which we denote $\CAlg$. Notice that if $R = \mathbb{Z}$, then $\CAlgZ = \Ring$. 

This example is actually a non-example which we find nevertheless pedagogic, as it has a schizophrenic object on one side of the duality, but not the other. Moreover, given the nature of the functors at play it shall take the form of being a schizophrenic object without explicitly being one, and we shall elucidate why that is, and why we have nevertheless included it.

We already know from commutative algebra the adjunction that gives a dual equivalence, and we can easily show that one of these adjoint functors is isomorphic to the $\Hom$ functor. 

We will nevertheless try to give some intuition about this isomorphism.

Firstly, we discuss the adjunction that one might learn in Algebra:

\[\begin{tikzcd}
	{\CAlg} & \bot & {\Aff}^{\op}
	\arrow["X(-)" , bend left, from=1-1, to=1-3]
	\arrow["\mathcal{O}(-)", bend left, from=1-3, to=1-1]
\end{tikzcd}\]

Recall that an affine scheme $X \in \Aff$ is defined as a representable functor in the functor category $\Fun(\CAlg, \Set)$.


Now we see that there is a natural choice for our adjunction given by sending a representable object to its functor, and sending that representable functor to its object. In other words we have $X: \CAlg^{\op} \to \Aff$ that sends $A \mapsto X_A = \CAlg(A, -)$ and $\mathcal{O}: \CAlg \to \Set$, which sends $X(-) = \CAlg(\mathcal{O}(X), -)$ to $\mathcal{O}(X)$, its representable object.

The equivalence is clear, since by construction  our unit and counit are  isomorphisms, i.e. $X_{\mathcal{O}(X)} =X$ and $\mathcal{O}(X_A) = A$. 


Now on the one hand,  the Yoneda lemma shows us that   $\Aff(X, \mathbb{A}^1) \cong \mathbb{A}^1(\mathcal{O}(X)) = \CAlg(R[t], \mathcal{O}(X)) \cong [\mathcal{O}(X)]$. The final set isomorphism is due to the fact that $R[t]$ is a free commutative $R$-algebra on one free generator. Now we see that for any $X_A \in \Aff$ it holds that $[\mathcal{O}(X_A)] = \Aff(X_A, \mathbb{A}^1)$, and as such our natural candidate for a schizophrenic object is $(R[t], \tau, \mathbb{A}^1 )$.





\[\begin{tikzcd}
	A && {SX_A} && {R[t]} \\
	\\
	{[A]} && {\Aff(X_A, \mathbb{A}^1)} && {[R[t]]}
	\arrow["{1_A}", dashed, from=1-1, to=1-3]
	\arrow["U"', from=1-1, to=3-1]
	\arrow["{d_{X_A,y}}", from=1-3, to=1-5]
	\arrow["U"', from=1-3, to=3-3]
	\arrow["U", from=1-5, to=3-5]
	\arrow["\cong"', from=3-1, to=3-3]
	\arrow["{\sigma \psi_{X_A,y}}"', from=3-3, to=3-5]
\end{tikzcd}\]

Since we have a canonical choice  $[A] \cong \Aff(X_A,\mathbb{A}^1)$ of set isomorphisms, fully faithfulness of the Yoneda embedding $U$ ensures that any morphism of lifts into  $A$  are unique, and as such the lift $(A \stackrel{d_{X_A,y}}{\to} R[t])_{y \in [X_A]}$ is initial.

%$(X_A \stackrel{e_{A,x}}{\to}\mathbb{A}^1)_{x \in [A]}$

The problem on the other side however is that we cannot find an appropriate concrete functor $V$ over which $\CAlg(A, R[t]) \stackrel{\tau\varphi_{A,x}}{\rightarrow} [R[t]]$ lifts to the category of affine schemes initially. 

By Yoneda we see that  \begin{align*}
	\CAlg(A, R[t]) \cong X_A(R[t]) = \Aff(\mathbb{A}^1, X_A) 
\end{align*}
so if we could find such a functor we would have $\Aff(\mathbb{A}^1,X_A) \cong [X_A]$ suggesting that $\mathbb{A}^1$ is a free object on one free generator, however there does not exist a functor $V$ so that this isomorphism holds. Although $\Aff$ is concrete (just take the functor $[\mathcal{O}(-)]$), such a functor does not lift $\tau\varphi_{A,x}$ in a way that can identify the set $\CAlg(A,R[t])$ with the functor $X_A$.

% and that it is isomorphic to $X(A) = \CAlg(A, -)$, however it is not obvious how to think about $\mathbb{A}^1$ as a free functor on one free generator, as elements of affine schemes are not given in the same way as they are for $R[x]$, where the element $X$ is clearly our free generator. 
%
%We may however still use Yoneda to see that $\CAlg(A, R[x]) \cong X_A(R[x]) = \Aff(\mathbb{A}^1, X_A) \cong [X_A]$, and as such our intuition thus must come from the fact that $\mathbb{A}^1$ is our free affine scheme on one free generator (\textcolor{blue}{TODO} why?) and  thus by the above equality we can think of elements of our affine scheme  as global functions from $A$ to $R[x]$.
%\[\begin{tikzcd}
%	{X_A} && TA && {\mathbb{A}^1} \\
%	\\
%	{[A]} && {\CAlg(A, R[t])} && {[\mathbb{A}^1]}
%	\arrow["{1_{X_A}}", dashed, from=1-1, to=1-3]
%	\arrow["V"', from=1-1, to=3-1]
%	\arrow["{e_{A,x}}", from=1-3, to=1-5]
%	\arrow["V"', from=1-3, to=3-3]
%	\arrow["V", from=1-5, to=3-5]
%	\arrow["\cong"', from=3-1, to=3-3]
%	\arrow["{\tau \varphi_{A,x}}"', from=3-3, to=3-5]
%\end{tikzcd}\]

%
%Now we shall try to understand the unit and counits.

\section{Gelfand Duality}
In order to describe the following duality, some context is in order. To that end we assume some familiarity with general topology, specifically with respect to compact Hausdorff spaces, and some functional analysis, such as notions of point-wise versus norm convergence. A basic understanding will suffice as we will not delve too deep into the functional or topological properties, to which we refer to \emph{Engelking's General Topology} \cite{engelking1989general} and \emph{Johnstone's Stone Spaces} for further reading. Our goal is only to underline the underlying categorical framework.

Our plan is to describe a general non-natural dual adjunction and then restrict it to a natural dual adjunction, which is also an equivalence. For that we must set the scene by defining the category of \emph{Kelley spaces}.
\subsection{Kelley Spaces}
Of primary importance to a Kelley space is the notion of $k$-continuity:
\\
\begin{definition}[$k$-continuous]
	A function $f: X \to Y$ of underlying sets of a topological space is said to be \textbf{$k$-continuous} if for all compact Hausdorff $K$ and continuous functions $t: K \to X$ the composition $f \circ t$ is continuous. 
\end{definition}

\emph{Remark}. We may  restrict this to an equivalent definition which tests only on inclusions of all compact subspaces $K \stackrel{\iota}{\hookrightarrow} X$. This is because images of compact spaces $K'$ under $t$ are homeomorphic to compact subspaces $t(K) \cong K' \subseteq X$. 

\begin{definition}[k-space]
A topological space $X$ is said to be a \textbf{$k$-space}, or a \textbf{compactly generated topological space}, if for all spaces $Y$ and underlying-set functions $f: X \to Y$, it holds that $f$ is continuous if and only if $f$ is $k$-continuous.	

Such spaces are also known as $\textbf{Kelley spaces}$.
\end{definition}
We refer to ${\kTop}$ as the category of compactly generated Hausdorff spaces, whose morphisms are the continuous functions between them, making it a full subcategory of $\Top$. 

Given a topological space, we may force the compactly generated condition on it through a process which we call the \emph{Kelleyfication} of a topological space.

Explicitly, given the set of inclusions\footnote{If $X$ is already Hausdorff, we may restrict to inclusions, since the image of continuous maps of compact spaces is compact and thus homeomorphic to some compact subspace of $X$. Otherwise, just replace the inclusions with arbitrary continuous maps from arbitrary compact Hausdorff spaces.} $(K \stackrel{t_i}{\hookrightarrow} X)_{i \in I}$ of  compact subspaces $K\subseteq X$, we give $X$ the finest topology making all $t_i$ continuous. In other words,  given an underlying-set function $[X] \stackrel{[f]}{\to} [Y]$, we give $X$ the topology such that $f$ is continuous if and only if $f \circ t_i$ is continuous. 

But this is just the universal property of the colimit applied to compact Hausdorff spaces: for any set of continuous functions  $K_i \stackrel{\varphi_i}{\to}Y$ into some topological space $Y$ that satisfy commutativity (i.e., if there is a continuous map $K_i \stackrel{h}{\to} K_j $ for some $i, j \in I$, then $\varphi_i = \varphi_j \circ h$), then there exists a unique continuous map $X \stackrel{f}{\to}Y$ such that $\varphi_i = f \circ t_i$. 

This is reflected by commutativity of the following diagram:


\[\begin{tikzcd}
	{K_i} && X && Y \\
	\\
	{[K_i]} && {[X]} && {[Y]}
	\arrow["{t_i}", from=1-1, to=1-3]
	\arrow["{f \circ t_i}", bend left,  from=1-1, to=1-5]
	\arrow[from=1-1, to=3-1]
	\arrow["{f}",dashed, from=1-3, to=1-5]
	\arrow[from=1-3, to=3-3]
	\arrow[from=1-5, to=3-5]
	\arrow["{[t_i]}"', from=3-1, to=3-3]
	\arrow["{[f]}"', from=3-3, to=3-5]
\end{tikzcd}\]
%\[\begin{tikzcd}
%	& Y \\
%	& X \\
%	{K_i} && {K_j}
%	\arrow["f", dashed, from=2-2, to=1-2]
%	\arrow["{f \circ t_i}", bend left,  from=3-1, to=1-2]
%	\arrow["{t_i}", from=3-1, to=2-2]
%	\arrow["h"', from=3-1, to=3-3]
%	\arrow["{f \circ t_j}"', bend right, from=3-3,  to=1-2]
%	\arrow["{t_j}"', from=3-3, to=2-2]
%\end{tikzcd}\]
That is if $f \circ t_i$ is continuous, pre-images of open sets $V \subseteq Y$ must be open under composition, in other words $t_i^{-1}(f^{-1}(V)) \subseteq K_i$ is open. 

So $f \circ t_i$ induces a continuous $f: X \to Y$ in the following way: given a topological space with underlying set  $X$, the colimit out of compact Hausdorff subspaces will be its \emph{Kelleyfication}, which is necessarily a refinement of the topology of $X$, since continuous maps $X \stackrel{f}{\to}Y$ necessarily satisfy commutativity of the above diagram, so we want to add opens to $X$ which satisfy commutativity for arbitrary set-function $[X] \stackrel{[f]}{\to}[Y]$. So we add opens $U = [f]^{-1}(V)$, for functions $[f]$ such that  $t_i^{-1}(U) \subseteq K$ is open  but $U \subseteq X$ is not. This is the universality condition, since commutativity must be satisfied for all $[f]$. But this is our original statement: we want a topology on $X$ such that $f$ is continuous if and only if $f \circ t_i$ is continuous. 

Thus we can understand a $k$-space as a colimit of compact Hausdorff spaces, or specifically, $X \in \kTop$ if and only if $X \in \Haus $ and $X = \colim_{K \subseteq X \text{ compact}}K$.

%\textcolor{red}{Question}: are compact subspaces in $X$ compact if and only if they are compact in Kelley($X$)? No.

We may otherwise view the \emph{Kelleyfication} as the right adjoint $k(-)$ to the forgetful functor into $\Top$, which we may also get as a categorical fact, since $\kTop$ is monotopological \cite{adamek1990abstract}.\footnote{The fact the $\kTop$ is a monotopological category is treated as standard fact in sources \cite{MR1147921} and \cite{DUBUC1971281}, although we were unfortunately unable to find an explicit proof. Therefore we will be using it without proof.} As $\kTop$ is a full subcategory of $\Top$, this puts us in the setting of a coreflective subcategory of $\Top$. 

In fact we may construct $\kTop$ categorically as the coreflective hull of $\kHaus$ in $\Haus$. That is to say, the underlying functor $\kHaus \hookrightarrow \Haus$ does not have a right adjoint, since $\kHaus$ is not closed under colimits. However we can take the intersection of all coreflective subcategories of $\Haus$ which contain $\kHaus$ to obtain the category generated under colimits of $\kHaus$. To see this, note that the inclusion functors are left adjoint so they preserve all colimits, and that by the Adjoint Functor Theorem the category generated by colimits of $\kHaus$ is itself coreflective.

Products in $\kTop$ are given as the Kelleyfications of topological  products. Furthermore,  any locally compact Hausdorff space $X$ is in $\kTop$, so we have $X \times_k Y = X \times Y$ for all $Y \in \kTop$, since $k(-)$ preserves limits \cite{engelking1989general}. Furthermore, function spaces are given as Kelleyfications of the set $\Top(X,Y)$ with the compact-open topology, which we denote $Y^X$ to clarify that it's a $\Top$-object.

It then follows from results about function spaces, namely, that if $X \times Z \in \kTop$ then the exponential mapping
\begin{align*}
	\Top(X \times_k Z, Y) \stackrel{\Lambda}{\rightarrow}& \Top(Z, Y^X)\\
	f(x, z) \mapsto& \Lambda(f(x, z))= f(z)(x) 
\end{align*}
 is a well-defined homeomorphism. Since Kelleyfication only refines the topology, underlying set bijection isn't altered and continuity $\Lambda$ and $\Lambda^{-1}$ are preserved, so that $\kTop$ is a cartesian closed concrete category admitting function spaces.

With the fact that $\kTop$ is a monotopological category, this then puts us in the setting of \ref{sec: int hom}. 

%, we want to show that $\kTop$ is a caretsian closed category admitting function spaces. This will indeed give us intuition of why we have even defined $\kTop$ the way we have, as $\Haus $ is not a cartesian closed category.  
%
%What we want is the following adjunction:
%\begin{align*}
%	\kTop(X \times_k Z, Y) \cong \kTop(Z, Y^X)
%\end{align*}
%
%However in $\kTop$  (\textcolor{red}{TODO} why?).
%\textcolor{red}{ToDO} Show that this means that $\kTop$ is cartesian closed concrete category which admits function spaces.
%And as per [Dub], we see that $\kTop$ admits function spaces.
%
%First we note that $\kTop$ is a coreflective subcategory of $\Top$. Since $\kTop$ is a coreflective subcategory of a topological category, it is itself a topological category (Herrlichs theorem \textcolor{red}{TODO} Find and cite this), and in particular it is monotopological. 
%
%Note that $\Top$ is a topological category since  for any source in $\Set$ we can always take the initial topology making every map in the source continuous.

Now let us define the category $\kAlg$ of complex $k$-algebras, whose objects are $\mathbb{C}$-algebras endowed with a Kelley topology, whose morphisms are the continuous algebra homomorphisms, and whose algebra operations are continuous with respect to $k$-products. 

Now since $\kTop$ is a monotopological category which admits function spaces, we are in the setting of a category which admits an internal hom-functor. 

To put ourselves completely in the situation of \ref{sec: int hom} we notice that $\kAlg \subseteq \kTop$ is a full subcategory, so that we have an underlying functor $U: \kAlg \to \kTop$, which satisfies the conditions of \ref{sec: int hom} in the following. As $\mathbb{C}$ is a colimit of its compact Hausdorff subspaces which is moreover a $\mathbb{C}$-algebra, we see that $\mathbb{C} \in \kAlg$, and we call it $\mathbb{C}_a$,  so that $U(\mathbb{C}_a) = \mathbb{C}_s$, by analogous notation, and clearly $\tau = 1_{\mathbb{C}}. $

Since  we are in a monotopological category, then we know that $\kAlg(A, \mathbb{C}_a) \to [\mathbb{C}_s]$
lifts initially to a $\kTop$-morphism, which we will denote $\Hom_k(A, \mathbb{C}_a) \to \mathbb{C}_s$.

Now all we need to see is that $\kTop(X,\mathbb{C}_s) \to [\mathbb{C}_a]$ lifts along $U$ functorially to a $\kAlg$-morphism $C_k(X, \mathbb{C}_s) \to \mathbb{C}_a$, which we must show directly. 

To show this we first want to see that $C_k(X, \mathbb{C}_s)$ has the necessary internal operations. For that we first notice that the functor $C_k(X, -)$ is product preserving (as a coreflection we can compute limits in $\Top$ and then Kelleyfy). 

So for instance we obtain addition by applying $C_k(X, -)$ to the corresponding addition operation  $+:\mathbb{C}_a \times \mathbb{C}_a \to \mathbb{C}_a$ in $\mathbb{C}_a$ but viewed as $\mathbb{C}_s$, so that $C_k(X,+):C_k(X,  \mathbb{C}_s \times \mathbb{C}_s) \to C_k(X, \mathbb{C}_s)$  is isomorphic to \begin{align*}C_k(X,+):C_k(X,  \mathbb{C}_s) \times C_k(X, \mathbb{C}_s) \to C_k(X, \mathbb{C}_s).
 \end{align*}
For scalar multiplication we notice that the composition \begin{align*}
	\mathbb{C}_s \times C_k(X, \mathbb{C}_s)\times  X \stackrel{1_{\mathbb{C}_s} \times \text{ev}}{\longrightarrow} \mathbb{C}_s \times \mathbb{C}_s \stackrel{m}{\to}\mathbb{C}_s
\end{align*}
induces a continuous scalar multiplication on $C_k(X, \mathbb{C}_s)$, by applying the Heyting implication to $\Big(\mathbb{C}_s \times C_k(X, \mathbb{C}_s)\Big) \times X \to \mathbb{C}_s$, to get \begin{align*}
	s: \mathbb{C}_s \times C_k(X, \mathbb{C}_s) \to C_k(X, \mathbb{C}_s).
\end{align*}
As such we have an adjunction 
\begin{align*}
	C: \kTop & \to \kAlg &  S: \kAlg & \to \kTop\\
	X &\mapsto C_k(X, \mathbb{C}_s) & A &\mapsto \Hom_k(A, \mathbb{C}_a)
\end{align*}

which, by remarks of \ref{sec: int hom}, is clearly not natural.

This is called the \emph{generalized Gelfand-Naimark Duality}. 

In the following we will want to restrict $\kTop$ and $\kAlg$ to their full subcategories under which the above adjunction is a dual equivalence. For that we will introduce the category of $C^*$-algebras.
\subsection{$C^*$-Algebras}
Consider the functor $C^*: \Top \to \Set$, which sends  $X \mapsto C^*(X) = \Top_{bd.}(X, \mathbb{C})$ for all $X \in \Top$. 
From pointwise operations $C^*(X)$ can be seen to be an associative, commutative, unital $\mathbb{C}$-algebra. Pointwise conjugation gives us the operation $f \mapsto f^*$, so that $C^*(X)$ is an involutive algebra, and the supremum norm $\lVert f \lVert = \sup_{x \in X}\lvert f(x)\lvert $ turns $C^*(X)$ into a normed algebra satisfying $\lVert f \lVert^2 = \lVert f \cdot f^* \lVert$. If we consider the involution preserving unital $\mathbb{C}$-algebra homomorphisms, we obtain a category $C^*$.

Note that $C^* \subseteq \kAlg$ is a full subcategory.

 We could view $C^*$ as a concrete category via the usual underlying-set functor, however we might find it more interesting in this case to consider a different faithful functor, namely the functor $\bigcirc: C^* \to \Set$ which sends any $C^*$-algebra $A$ to it's unit ball $\{ a \in A \  \lvert  \ \lVert a \lVert \leq 1\}$ and each morphism $f$  to its restriction $f_\bigcirc$ to the unit ball of its domain.
 
 \subsection{Gelfand-Naimark Duality}
 
 For any compact Hausdorff space $X$, we have $C_k(X,\mathbb{C}_s) = C(X, \mathbb{C}_s)$ \cite{engelking1989general}, so that the $C^*$-algebra $C^*(X)$ and the function $k$-algebra $C(X) := C_k(X, \mathbb{C}_s)$ coincide algebraically, since continuous functions over compact spaces are bounded in $\mathbb{C}$ and since $X \times_k X = X \times X$. Moreover the topology of $C^*(X)$ for $X$ compact is the compact open topology \cite{engelking1989general},  and therefore they also coincide topologically. That means we have can restrict $C$ to a functor $C: \kHaus \to C^*$.
 
 For initiality we notice that any other algebraic structure on $C^*(X)$ that is compatible with the underlying point-wise structure must factor uniquely through $C^*(X)$, as operations must preserve involution and the norm. 
 
% use the typical argument that we define all operations on $C_k(X, \mathbb{C}_s)$ pointwise: addition and multiplication is standard; for involution we have $f^*(x) = \overline{f(x)}$ and for the norm we have $\lVert f \lVert_{\infty} = \sup_{x \in X} \lvert f(x) \lvert $ so that $\lVert \ev_{X,x}(f)\lVert_{\infty, \mathbb{C}} \leq \lVert f \lVert_{\infty}$.
 
 Can we restrict $S$ accordingly? For every $C^*$-algebra $A$, the space $S(A) = \Hom_k(A, \mathbb{C}_a)$ is compact, and its topology is that of pointwise convergence \cite{DUBUC1971281}. In other words, it is the initial topology making the inclusion $S(A) \hookrightarrow \mathbb{C}^A$ continuous, which is just the initial topology making all evaluation maps continuous. Hausdorffness is clear, since the product of Hausdorff spaces is Hausdorff, and the subspace of a Hausdorff space is Hausdorff. We call $S: C^* \to \kHaus$ the \emph{Spectrum}-functor, and it follows that the generalized Gelfand-Naimark adjunction restricts to a dual adjunction
 
 \begin{align*}
	C: \kHaus & \to C^* &  S: C^* & \to \kHaus\\
	X &\mapsto C_k(X, \mathbb{C}_s) & A &\mapsto \Hom_k(A, \mathbb{C}_a).
\end{align*}
 
 
 
 However this still doesn't give us the schizophrenic object, as $\mathbb{C}_s \notin \kHaus$. Remember that we want to consider $C^*$ as a concrete category via the functor $\bigcirc$, so that for any compact Hausdorff space $X$, one has $\bigcirc C(X) = \kHaus(X, D)$ where $D = \{ c \in \mathbb{C}_s \ \lvert \ \lVert c \lVert \leq 1 \}. $
 
 Thus we can conclude that there is a natural dual adjunction between concrete categories $(\kHaus, U)$ and $(C^*, \bigcirc)$ with schizophrenic object $(D, 1_D, \mathbb{C}_a)$. 
 
 We would like to remark that this duality is an equivalence. This fact is indeed interesting, however to show it requires machinery that is beyond the scope of this thesis. For details, refer to chapters III. and IV. in \cite{zbMATH03940199}.
 
 \section{The Galois Correspondence}
 Let us dive into another familiar example called the \emph{Galois correspondence}.
 
 For this example we assume familiarity with Galois theory. That is we assume familiarity with field extensions and introductory group theory, specifically with respect to orbits and stabilizers. We interpret a $\GSet$ to be a set equipped with a left group action, that is, a group homomorphism $\theta: G \to \Aut(S)$. Our main reference text shall be \emph{Szamuely's Galois Groups and Fundamental Groups} \cite{szamuely2009galois}, though we also recommend \emph{Bosch's Algebra} \cite{zbMATH07745132}, for the non-categorical approach.
 
  One can phrase the fundamental theorem of Galois theory as the following contravariant adjunction between the directed posets of subfield extensions of the finite field extension $k \hookrightarrow M\hookrightarrow L$ and of the corresponding subgroups $H$ of the Galois group $\Gal(L/k)$:
 
 \[\begin{tikzcd}
	{\{L|M|k\} } & \bot & {\{H \leq \Gal(L/k)\}}^{\op}
	\arrow["{\Gal(L/-)}", shift left=3, bend left, from=1-1, to=1-3]
	\arrow["{L^{(-)}}", bend left, from=1-3, to=1-1]
\end{tikzcd}\]

That is to say ${\{L|M|k\} }^{\op}$ is codirected since $L|k$ is an initial object of the poset, and similarly ${\{H \leq \Gal(L/k)\}}$ is directed  since $\Gal(L/k)$ is terminal, corresponding to the full subgroup $\Gal(L/k)$. 

It is easily checked that this defines an adjunction, however the schizophrenic object doesn't live here, so we would like to focus on a more general adjunction, which we can derive from the above adjunction in the following way. 

The first important consideration is that we want our left (and by consequence, our right) category to include the separable closure of $k$, which we denote $k_s$, and naturally, all its intermediate finite field extensions. Intuitively, we want a field extension $k_s$ whose intermediate field extensions $k^s|L$ are all normal \emph{and} separable. 

That's why we take $k_s$ and not $ \bar k$,  the algebraic closure of $k$. For perfect fields, where every algebraic extension is separable,  we have $k_s = \bar k$, by definition. However in general, we only have $k_s \subseteq \bar k$ so we cannot guarantee that an arbitrary algebraic extension of $k$ is separable.


Now from the above left adjoint functor one may obtain an adjunction with respect to the set of cosets of each subgroup $H$ in $\Gal(k_s/k) =: G$, which has a canonical $G$-action on it.

Explicitly this is a map $L|k \mapsto \CAlgk(L, k_s)$, where $\CAlgk(L, k_s)$  can be given by $ \Gal(k_s/L) \setminus \Gal(k_s/k)$ (this shall be clear by the end of this section).
 
 For any finite field extension $L|k$, homomorphisms into the closure $k_s$ are given by maps which permute the roots of the minimal polynomials over $k$ of the finite generators of $L|k$. For background reading one may consider any introductory textbook on Galois theory, such as \emph{Bosch's Algebra}, where this claim follows directly from Lemma 3.4/8 \cite{zbMATH07745132} . 
 
 Since every element of $k_s$ is a separable root over  $k$, the canonical $G$-action on $k_s$ which defines $G$ in the first place is completely determined by how its elements permute the roots of separable minimal polynomials with coefficients in $k$.
 \subsection*{Schizophrenic Object of $\CAlgk \rightleftarrows \GSet^{\op}$}
 
 In the following we shall want to show explicitly that the generalized Galois adjunction fits the schizophrenic framework to be considered a concrete duality.
  
 To be sufficiently general we consider the concrete category $(\CAlgk, U) $, where $U(-) = \CAlgk(k[t],-)$,  of commutative unital $k$-algebras with $k$-algebra homomorphisms, and the concrete category $(\GSet,V)$, where $V(-) = \GSet(\{pt\},-)$, of sets with a $\Gal(k_s/k)$-action and $G$-equivariant maps between them. The representable functors should already be clear based on previous examples. We only plan to show that $(k_s, 1_{k_s}, k_s)$ is a schizophrenic object which induces the following adjunction 
  \[\begin{tikzcd}
	\CAlgk & \bot & \GSet^{\op}
	\arrow["{\CAlgk(-, k_s)}", shift left=3, bend left, from=1-1, to=1-3]
	\arrow["{\GSet(-, k_s)}", bend left, from=1-3, to=1-1].
\end{tikzcd}\]

Just to remark, the free objects on one free generator in these categories should already be clear by now from previous examples: $k[t]$ and $\{pt\}$.

%That also means in this case that the set $\CAlgk(L,k_s)$ is finite. 
First  we  seek $G$-equivariant lifts of the evaluation maps $\CAlgk(L,k_s) \to [k_s]$, i.e. lifts that preserves the $G$-action, which just means we can define the $G$-action of our lift of $\CAlgk(L,k_s)$ pointwise on $k_s$, since \begin{align*}
	(g \varphi)(x) := \ev_{x,L}(g \cdot \varphi) =  g \cdot  \ev_{x, L}(\varphi) = g \cdot \varphi(x).
\end{align*}

Since $k_s$ comes with a canonical action, we know $L|k \mapsto \CAlgk(L, k_s)$ lifts to the category $\GSet$.

Note that there is nothing specific about the canonical action for the above equalities to hold, so that for any $G$-action $\star$ on $k_s$ there  automatically exists a lift of $\CAlgk(L,k_s) \to [k_s]$ whose $G$-action is defined pointwise by $\star$. 

%We want a lift of the evaluation  $\ev_{x,L}:\CAlgk(L,k_s) \to [k_s]$ to exist for all $L \in \CAlgk$ and all $x \in L$, so 

But we have good reason to consider the canonical $G$-action on $k_s$. Remember as a schizophrenic object this $G$-action must be fixed.

The question then becomes, given this $G$-action on $k_s$, can we derive a obtain a different $G$-action on $\CAlgk(L,k_s)$ than the one we defined above?

Explicitly the inhereted $G$-action on $\CAlgk(L,k_s)$ is given by  postcomposition, i.e., $g \cdot_{\CAlgk(L,k_s)} \varphi = g \circ \varphi$ for each $L \in \CAlgk$. As $G$ is a group of autormorphisms on $k_s$, a $G$-equivariant map 
	$\CAlgk(f,k_s): \CAlgk(L',k_s) \stackrel{- \circ f}{\to} \CAlgk(L,k_s)
$ means \begin{align*} g \cdot (\varphi \circ f) = (g \cdot \varphi) \circ f  \end{align*}
holds. For postcomposition the equality is satisfied for all $g \in G$. 

Notice that such a $G$-equivariant map which is compatible with lifts of the evaluation maps is given by a natural transformation $\CAlgk(-,k_s) \stackrel{\eta}{\implies}  \CAlgk(-,k_s)$, who by Yoneda is given by unique $k_s$-automorphisms $u_g: k_s \to k_s$, so that such an action is given by a group homomorphism $\theta: G \to G$, which sends $g \mapsto u_g$. As such, a $G$-equivariant map $f$ satisfies the equality \begin{align*}
	g \cdot (\varphi \circ f) = (\theta(g) \circ \varphi)(f).
\end{align*}
We want to show that $\theta = 1_G$. 

In our set up we want a $G$-action on $\CAlgk(L,k_s)$ such that for all $g \in G$ and $\varphi \in \CAlgk(L,k_s)$ we have \begin{align*}
	g \cdot \varphi(x) = g \cdot \ev_{x,L}(\varphi) = \ev_{x,L}(\theta(g) \circ \varphi) = (\theta(g) \circ\varphi)(x).
\end{align*}
Since this must hold for all $L \in \CAlgk$, $x \in L$ and all $\varphi \in \CAlgk(L, k_s)$, it must hold in particular for $L=k_s$ and $\varphi = 1_{k_s}$, then the above equality becomes $g(x) = \theta(g)(x)$ for all $x \in k_s$ and $g \in G$. Indeed this lift only  exists if it is the canonical one, i.e. if $\theta = 1_G$.

That means there is only one way to lift $\CAlgk(L,k_s) \to [k_s]$ to $\GSet$ if $k_s$ is considered with the canonical $G$-action, and therefore such a lift is  initial.

On the other side, we want to show that $\GSet(H,k_s)$ lifts initially to $\CAlgk$, where we take $k_s$ to be equipped with the canonical $G$-action.

It is easily checked that the ring axioms can be defined pointwise in $k_s$. 

Since $G$ fixes $k$, then we have $\const_a \in \GSet(H,k_s)$ for each $a \in k $, since $	\const_a(\theta(g)\cdot y) = g \circ \const_a(y)$ holds true for all $g \in G$ and independent of $\theta$. That means for any $G$-equivariant map $H \to H'$ the induced map $\GSet(H',k_s) \to \GSet(H,k_s)$ is stable on these constant functions. In other words it commutes with the map $k \stackrel{\gamma}{\rightarrow}\GSet(H,k_s)$, which sends $a \mapsto \const_a$, which defines a ring homomorphism, so that $\GSet(H,k_s) \in \CAlgk$. Moreover this lift is clearly functorial in $H$, since composition preserves all pointwise operations. 

Therefore we can determine our $\CAlgk$ object by the datum $(\GSet(H,k_s), +, \cdot, \gamma)$. And let $e_{y,H}$ be the lift of $\ev_{y,H}$ to $\CAlgk$.  

Now let $(\oplus, \otimes, \gamma')$ be any other $\CAlgk$ structure on $\GSet(H,k_s)$ such that every $\ev_{y,H}$ lifts to a $\CAlgk$-morphism $e_{y,H}'$. Then \begin{align*}
 	(f \oplus g)(y) = e_{y,H}'(f \oplus g) =& e_{y,H}'(f) + e_{y,H}'(g)\\ =& f(y)+g(y) = e_{y,H}(f) +e_{y,H}(g)\\ =& e_{y,H}(f+g)  = (f+g) (y)
 \end{align*}
for all $y \in H$ and $H \in \GSet$, so that $f \oplus g = f +g$. Similarly we get $f \otimes g = f \cdot g$ and $\gamma = \gamma'$. Therefore our pointwise algebra structure is the unique $\CAlgk$ structure making all evaluation maps $\CAlgk$-morphisms, and in particular, it is the \emph{initial} such lift.  

%Any other $\CAlgk$-morphism whose underlying set map is compatible with  all  evaluation maps $\GSet(H,k_s) \stackrel{\ev_{y,H}}{\rightarrow} [k_s]$ must factor uniquely through our lift of $\GSet(H,k_s) $, since $\CAlgk$-morphisms into $k_s$ are uniquely determined by permutation of roots in $k_s$, permutations which are defined on the set level.

This is analogous to many familiar examples where we derived our structure pointwise from the structure of our schizophrenic object, and as a result such a structure is the weakest such compatible structure, i.e. an initial lift.

%\[\begin{tikzcd}
%	{K'} && TH && {k_s} \\
%	\\
%	{[K']} && {\GSet(H,L^k)} && {[k_s]}
%	\arrow[dashed, from=1-1, to=1-3]
%	\arrow[from=1-1, to=3-1]
%	\arrow["{e_{y,H}}", from=1-3, to=1-5]
%	\arrow[from=1-3, to=3-3]
%	\arrow[from=1-5, to=3-5]
%	\arrow["h", from=3-1, to=3-3]
%	\arrow["{\ev_{y,H}}", from=3-3, to=3-5]
%\end{tikzcd}\]
%
%
%(\textcolor{blue}{TODO} show that $\GSet(H,k_s)$ lifts initially to $\CAlgk$, I still don't know, but idea is that  a $G$-equivariant map of some set should determine some Galois subfield extension since its "part of that action" being preserved by a map. Determine how this set of maps should look like, and what exactly is the datum of the $G$-set as well as datum of the $\CAlgk$; How big or small is this $\Hom$-set? Is there a function for every element of the field, or is it enough to realize the datum in some sense as that of the generating elements of the extension? ). 


Therefore our triple $(k_s,1_{k_s},k_s)$ is indeed a schizophrenic object, which means the diagram we gave at the beginning of this section is in fact a well defined adjunction, as it is the induced adjunction of our schizophrenic object.

\subsection*{Restricting to an Equivalence of Categories}

Even though at this point the goal of this example, showing that we have a concrete duality, is complete, we would like to continue our discussion to bring us back to concrete Galois theory.

That is we would like to work only with the objects that we see in Galois theory, in particular, with finite extensions of $k$, as opposed to general algebras over $k$. 

Given a finite separable field extension $L|k$, the set $\CAlgk(L,k_s)$ is finite and the $G$-action it inherits is transitive [Szam]. In particular, our above adjunction restricts to the following dual equivalence:


 \[\begin{tikzcd}
	\{\text{finite separable extensions of } k \} & \stackrel{\cong}{\bot} & \GFSet_{\text{transitive}}^{\op}
	\arrow["{\CAlgk(-, k_s)}", shift left=3, bend left, from=1-1, to=1-3]
	\arrow["{\GSet(-, k_s)}", bend left, from=1-3, to=1-1]
\end{tikzcd}\]

One may refer to Theorem 1.5.2 in \cite{szamuely2009galois} to see that the left adjoint functor induces a dual equivalence. We will indeed inadvertantly end up reproving this, however, our goal in this part is rather to underline what some constituents of this adjunction look like as we're now in a position to see our lifts a little more concretely.

Firstly, for a finite field extension $L|k$ we have an explicit description of $\CAlgk(L,k_s)$ as a set of homomorphisms which permute the roots of minimal polynomials of the generators of $L$. This is clearly a transitive finite $G$-set, since $G$ permutes these finite roots transitively.

From the proof of Theorem 1.5.2 we also know that any transitive $G$-set $H$ is given as $\CAlgk(L,k_s)$ for some finite field extension $L$ by the assignment $g \circ \iota = g \cdot x$ for some $x \in H$, where $\iota: L \to k_s$ is the inclusion homomorphism. It is a group theoretic fact that then this is isomorphic to the left coset space $U_x \setminus G$, where $U_x$ is the stabilizer of $x$. 

That means a transitive $G$-set is encoded by the information of the stabilizer $U_x$ of $x \in H$. This will be important to understand what finite field extension $\GSet(H,k_s)$ is. 

Consider that $\GSet(H,k_s) = \GSet(\CAlgk(L,k_s),k_s)$ for some finite separable field extension $L|k$. We want both an explicit description of  $L$ that only depends on the $G$-action of $H$, and to see why  $\GSet(\CAlgk(L,k_s),k_s) = L$.

Notice that a $G$-equivariant map $ H \stackrel{f}{\to} k_s$ is completely determined by the image of a single element $x \in H$, since transitivity of $H$ and the $G$ action on $k_s$ give us a full description of $f$ via \begin{align*}
	f(g\cdot x) = g \cdot f(x).
\end{align*}

Remember that $U_x$ is a subgroup of $G$, so that by the fundamental theorem it fixes a Galois extension $L' = (k_s)^{U_x}$. We want to see that indeed $L' = L$. 

We show first that the map  $\GSet(H,k_s) \stackrel{\psi}{\rightarrow} (k_s)^{U_x}$ given by $ f \mapsto f(x)$ is a bijection. 

For any $h \in U_x$ we have $f(x) = f(h \cdot x) = h \cdot f(x)$ by $G$-equivariance, so that $U_x$ fixes $f(x)$. 

For an inverse consider $a \mapsto f_a$ where $f_a(x) = a$, given by $f_a(g \cdot x) = g \cdot a$. We just want to see that $f_a$ is well defined and $G$-equivariant. 

If $g_1 \cdot x = g_2 \cdot x$ then $ g_2^{-1}\cdot g_1 \in U_x$, which means $g_1 \cdot a = g_2 (g_2^{-1} g_1) \cdot a = g_2 \cdot a$. 

Given $g' \in G$ and that $y = g \cdot x$, we have \begin{align*}
	f(g' \cdot y) = f(g' \cdot (g \cdot x)) = f((g'g)\cdot x) = (g'g) \cdot a = g' \cdot (g \cdot a) = g'\cdot (f(g\cdot x)) = g'\cdot f(y).
\end{align*}
It is easily checked that these maps are inverse to one another.

Now replacing $H$ with $\CAlgk(L,k_s)$, we can use an analog to the above well-definedness argument to show why the stabilizer of $\iota$ is exactly $U_x$. And as $\iota$ is just the inclusion we have then $h(a) = a$ for any $a \in L$ and all $h \in U_{x}$, so that $L \subseteq (k_s)^{U_x}$.

For the other inclusion we simply see use the fact that any finite field extension is given as a fixed field of some subgroup of $G$, so that our stabilizer $U_{\iota} = \{ g \in G \lvert g \circ \iota = \iota \}$ is exactly  $\Gal(k_s/L)$. That means if $g(a) = a$ for all $a \in (k_s)^{U_x}$, then $a \in L$. 

Therefore $L|k = \GSet(\CAlgk(L,k_s),k_s)$. Using this we can also derive directly that  $H = \CAlgk(\GSet(H,k_s),k_s))$, since restating gives us \begin{align*}
	\CAlgk(\GSet(H,k_s),k_s))= \CAlgk(k_s^{U_x},k_s) = \Gal(k_s/k_s^{U_x}) \setminus \Gal(k_s/k) = U_x \setminus G = H
\end{align*}


%$\{k_s|L|k\} $ ${\{H \leq \Gal(L/k)\}}$
This adjunction will then extend to another subdual equivalence of the Galois correspondence, namely between the full subcategory $\ket \subseteq \CAlgk$ of finite dimensional algebras over $k$, defined to be isomorphic to a finite product of finite separable field extensions, and the full subcategory $\GFSet \subseteq \GSet$ of finite $G$-sets:

 \[\begin{tikzcd}
	\ket & \stackrel{\cong}{\bot} & \GFSet^{\op}
	\arrow["{\CAlgk(-, k_s)}", shift left=3, bend left, from=1-1, to=1-3]
	\arrow["{\GSet(-, k_s)}", bend left, from=1-3, to=1-1]
\end{tikzcd}\]

For this we only remark that any $G$-set can be seen as a disjoint union of transitive $G$-sets over each $G$-orbit, and that \begin{align*}
	\CAlgk(\prod_I L_i, k) = \coprod_I\CAlgk(L_i,k)
\end{align*}
where $I$ is the finite index set of the set of orbits $G \setminus \CAlgk(\prod_I L_i, k)$. For details see \cite{szamuely2009galois}.


As a side remark,  if our subfield extension $L|k$ is $Galois$, that is,  it is a \emph{normal} (separable, algebraic) field extension, i.e. $L$ contains the roots of every polynomial over itself, then the fundamental theorem of Galois theory states that $\Gal(k_s/L) \unlhd \Gal(k_s/k)$. This gives us a Galois group over a different functor, namely the functor $\Gal(-/k)$ which yields a profinite system $({k_s|L|k} , \Gal(L/k))$.

Consider that $k_s|k = \colim_{\{k_s|L|k\}}L|k$, then we have \begin{align*}
	\Gal(k_s/k)=& \CAlgk(\colim_{\{k_s|L|k\}}L, k_s)\\ =& \lim_{\{k_s|L|k\}}\CAlgk(L,k_s)\\ =& \lim_{\{k_s|L|k\}}\Gal(k_s/k)/\Gal(k_s/L)\\
	=& \lim_{\{k_s|L|k\}} \Gal(L/k)
\end{align*}

so that $\Gal(k_s/k)$ can be realized as a profinite group.

% (\textcolor{red}{TODO}: finish this part, if necessary, but the basic structure is the following: show that $\kTop$ is a monotopological category, show that monotopological categories admit function spaces... if theres an easier way to do it we may not need to define lifts and topological categories, so do that later)
 
 

%
%Our adjunction has the unit
%\begin{align*}
%	\epsilon:& 1_{\Frm} \to ST\\
%	&A \mapsto STA\\
%	&A \mapsto \Top(\Frm(A, \mathbb{2}), \mathbb{S})
%\end{align*}
%so that 
%\begin{align*}
%	\epsilon_A:& A \to \Top(\Frm(A, \mathbb{2}), \mathbb{S})
%\end{align*}
%where 
%\begin{align*}
%	\epsilon_A(x):& \Frm(A, \mathbb{2}) \to \mathbb{S}\\
%	&p \mapsto p(x)
%\end{align*}
%Notice that the topology on $\Frm(A, \mathbf{2})$ is the initial topology making all $(\epsilon_A(x))_{x \in A}$ continuous.
%
%When passing to $U$ and $V$, we have
%\begin{align*}
%	[\epsilon_{\t A}]:& [\t A] \to [\Top(\Frm(\t A, \mathbb{2}), \mathbb{S})] = \Top(\Frm(\t A, \mathbb{2}), \mathbb{S})\\
%	&(\t x: 1 \to x) \mapsto (p \to p(x))
%\end{align*}
%which is equal to
%\begin{align}
%	[\epsilon_{\mathbb{2}}]:& [\mathbb{2}] \to [\Top(\Frm(\mathbb{2}, \mathbb{2}), \mathbb{S})] = \Top(\Frm(\mathbb{2}, \mathbb{2}), \mathbb{S})\\
%	&(\t x: 1 \to x) \mapsto (1_{\mathbb{2}} \to 1_{\mathbb{2}}(x))
%\end{align}
%So we have 
%\begin{align*}
%	[\epsilon_{\mathbb{2}}](\t x):& \Frm(\mathbb{2},\mathbb{2}) \to \mathbb{S}\\
%	&1_{\mathbb{2}} \mapsto 1_{\mathbb{2}}(x)
%\end{align*}
%and now
%\begin{align*}
%	[[\epsilon_{\mathbb{2}}](\t x)]:& [\Frm(\mathbb{2},\mathbb{2})] \to [\mathbb{S}]\\
%	&(1 \to 1_{\mathbb{2}}) \mapsto (\{pt\} \to1_{\mathbb{2}}(x))
%\end{align*}
%so that 
%\begin{align*}
%	[[\epsilon_{\mathbb{2}}](\t x)](1_{\mathbb{2}}) = (\{pt\} \to 1_{\mathbb{2}}(x)).
%\end{align*}
%Now we can see that the map \begin{align*}
%	\tau: [\mathbb{2}] &\to [\mathbb{S}]\\
%	(\t x: 1 \to x) &\mapsto (\{pt\} \to 1_{\mathbb{2}}(x))
%\end{align*}
%is the identity in $\Set$, as the underlying set on both sides is $[\mathbb{2}] = [\mathbb{S}] = \{0,1\}$ so that we are looking at the set map \begin{align*}
%	\{0,1\} &\to \{0,1\}\\
%	x &\mapsto 1_{\{0,1\}}(x)= x
%\end{align*}

\chapter{Questions}
\begin{enumerate}
%	\item Why do we know that Kelleyfied products and kelleyfied function spaces with the compact open topology are the categorical products and exponentials in $\kTop$
%	\item ask Georg if pointwise order preserving evaluation morphism into $\mathbb{2}$ preserves finite limits and colimits by some pointwise argument in the presheaf category
%	\item Let $C(X) = C_k(X,Y)$ be the kelleyfied function space with the compact open topology. Let $C^*(X) = \Top_{\text{bdd}}(X, \mathbb{C})\subseteq C(X)$. Why is $C^*(X) \subset C(X)$ 
%	\item If $X$ is compact Hausdorff, then is $C(X)$
	\item Show that $C_k(X, \mathbb{C}) = C(X, \mathbb{C})$ if $X$ is locally compact. Is it true? Maybe $X$ needs to be compact? 
	\item Why does $C_k(A, \mathbb{C}_s) \mapsto [\mathbb{C}_a]$ lift to $C^*$ initially? 
	\item Why does $\Hom_k(B, \mathbb{C}_a) \stackrel{\ev_{B,y}}{\rightarrow} [\mathbb{C}_s]$ lift to $\kHaus$ initially?
\end{enumerate}
%\begin{enumerate}
%	\item Is the above actually the case, i.e., am I actually supposed to look at the underlying sets in $\Set$ to understanding that $\tau$ is the identity?
%	\subitem am I to understand $p(x) \in \mathbb{S}$ as "translating" from $\Frm$ to $\Top$ via $\Set$-elements?
%	\item Is it actually true the remark about a free generator on one free object fully determining the set $\HomA(A_0,A) = A$ by where it sends the free generator to?
%	\item In the commutative triangle we have the upper part of the lift being an isomorphism by the adjunction, so that a set of $V$ initial lifts $(TA \to B)_{x \in A}$ should be fully determined by existence of the maps $\tau \circ (\varphi_{A, x})$, correct? What I mean is, to show that a schizophrenic object exists, or to check the axioms for an S.O., I need only to show that $\tau \circ (\varphi_{A, x})$ is well defined for all $A \in \mathcal{A}$ and $x \in A$, and behaves as expected?
%	\item In general,  a schizophrenic object should be fully determined by the adjunction and the representing objects. I mean to say if I understand the adjunction and know what the representing objects are, I have everything I need to check my hypothesis about the schizophrenic object, i.e., a hypothesis about a schizophrenic object leads to simply checking the adjunction and representable objects.
%	\item Why are the representable objects \emph{necessarily} free objects on one free generator, and what is meant by free here? Does the necessary part come from the fact that only such an object could lead to a schizophrenic object, or is it for some reason necessary a priori?
%\end{enumerate}
\bibliography{references}

\end{document}