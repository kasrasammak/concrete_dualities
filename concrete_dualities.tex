\documentclass[12pt,a4paper]{article}
\usepackage[margin=3cm]{geometry}
\usepackage{fancyhdr}
\usepackage{amssymb}
\usepackage{amsthm}
\usepackage{amsfonts}
\usepackage{bbold}
\usepackage[utf8]{inputenc}
\usepackage{parskip}
\usepackage{mathtools}
\usepackage[english]{babel}
\usepackage{tikz-cd}
\usepackage{bbm}
\usepackage{stackrel}
% change subsections to alphabetic
\renewcommand\thesubsection{\thesection.\alph{subsection}}

\newtheorem{theorem}{Theorem}[section] % part
\newtheorem{lemma}{Lemma}[section] % part
\newtheorem{definition}{Definition}[section] % part

\newcommand{\D}[2]{#1^\prime(#2)}
\newcommand{\norm}[1]{\left\lVert#1\right\rVert}
\newcommand{\abs}[1]{\left|#1\right|}
\newcommand{\scalar}[2]{\langle#1, #2\rangle}

\DeclareMathOperator{\Ima}{Im}
\DeclareMathOperator{\Gal}{Gal}
\DeclareMathOperator{\colim}{colim}
\DeclareMathOperator{\Hom}{Hom}
\DeclareMathOperator{\Set}{Set}
\DeclareMathOperator{\Frm}{Frm}
\DeclareMathOperator{\Top}{Top}
\DeclareMathOperator{\Bool}{Bool}
\DeclareMathOperator{\CAlg}{CAlg_R}
\DeclareMathOperator{\CAlgZ}{CAlg_\mathbb{Z}}
\DeclareMathOperator{\CAlgk}{CAlg_k}
\DeclareMathOperator{\Ring}{Ring}
\DeclareMathOperator{\Aff}{Aff}
\DeclareMathOperator{\Fun}{Fun}
\DeclareMathOperator{\Nat}{Nat}
\DeclareMathOperator{\Ob}{Ob}
\DeclareMathOperator{\Mor}{Mor}
\DeclareMathOperator{\kSp}{kSp}
\DeclareMathOperator{\kHaus}{kHaus}
\DeclareMathOperator{\Haus}{Haus}
\DeclareMathOperator{\ev}{ev}
\DeclareMathOperator{\Lift}{Lift}
\DeclareMathOperator{\kAlg}{kAlg}
\DeclareMathOperator{\GSet}{G-Set}
\DeclareMathOperator{\GFSet}{G-FinSet}
\DeclareMathOperator{\ket}{k_{\text{ét}}}
\DeclareMathOperator{\op}{op}
\DeclareMathOperator{\Ult}{Ult}
\DeclareMathOperator{\clop}{Clop}
\DeclareMathOperator{\FinSet}{FinSet}
\DeclareMathOperator{\prim}{prime}
\DeclareMathOperator{\Spat}{SpatLoc}
\DeclareMathOperator{\Sob}{SobTop}
\DeclareMathOperator{\DLat}{DLat}
\DeclareMathOperator{\Loc}{Loc}
\DeclareMathOperator{\CohLoc	}{CohLoc}
\DeclareMathOperator{\CohTop}{CohTop}
\DeclareMathOperator{\Stone}{Stone}









\def\HomA{\ensuremath\mathcal{A}}
\def\HomB{\ensuremath\mathcal{B}}
\def\HomC{\ensuremath\mathcal{C}}
\def\HomD{\ensuremath\mathcal{D}}

\def\concA{\ensuremath(\mathcal{A},U)}
\def\concB{\ensuremath(\mathcal{B},V)}


\def\HomFrm{\ensuremath\Hom_\Frm}
\def\HomTop{\ensuremath\Hom_\Top}
\def\t{\ensuremath\tilde}

\DeclarePairedDelimiter\ceil{\lceil}{\rceil}
\DeclarePairedDelimiter\floor{\lfloor}{\rfloor}
\renewcommand\qedsymbol{$\blacksquare$}
% https://q.uiver.app/#q=WzAsMyxbMCwwLCJcXG1hdGhjYWx7QX1ee29wfSJdLFsyLDAsIlxcbWF0aGNhbHtCfSJdLFsxLDAsIlxcYm90Il0sWzAsMSwiVCIsMCx7ImN1cnZlIjotMn1dLFsxLDAsIlMiLDAseyJjdXJ2ZSI6LTJ9XV0=


\begin{document}

\pagestyle{fancy}
\fancyhf{}
\rhead{Kasra Sammak}
\lhead{Concrete Dualities}
\rfoot{\thepage}
\section{Introduction}
\section{Preliminaries/Index}

Throughout this thesis we will be using the language of category theory, for which a mild introduction is in order. To that degree we plan to define the terms used as well as prove some well known theorems which we use quite freely, so not to distract us from the argument.

We assume basic knowledge of Category theory (such as the definitions of Categories, functors, natural transformations, limits, colimits, initial object, terminal object, etc), and will choose to define terms as necessary as they relate to our investigation.
\\
\subsection{Adjunctions, Concrete Categories}
\begin{definition}
	Adjunction
\end{definition}
\
\begin{definition}[Concrete Category]
	Let $\mathcal{X}$ be a category. A \textbf{concrete category} over $\mathcal{X}$ is a pair $(\mathcal{A}, U)$ where $\mathcal{A}$ is a category and $U: \mathcal{A} \to \mathcal{X}$ is a faithful functor. Sometimes $U$ is called the \textbf{underlying functor} over $\mathcal{X}$ and $\mathcal{X}$ is called the \textbf{base category} for $(\mathcal{A},U)$. A concrete category over $\Set$ is called a \textbf{construct}.
\end{definition}
\
\begin{definition} [Concrete functor]
Let $(\mathcal{A}, U)$ and $(\mathcal{B}, V)$ be concrete categories over $\mathcal{X}$. A \textbf{concrete functor} from $(\mathcal{A}, U)$ to $(\mathcal{B},V)$ is a functor $F: \mathcal{A} \to \mathcal{B}$ such that $U = V \circ F$. 
\end{definition}
Note that a concrete functor is necessarily faithful, since $V$ is faithful. In general, for any functors $F: \mathcal{A} \to \mathcal{B}$ and $G: \mathcal{B} \to \mathcal{C}$, if $G \circ F$ is faithful, then $F$ is. The proof is easy: consider elements  $f, g \in \HomA(A, A')$ such that $Ff = Fg$. Then $GFf = GFg$ and since $G\circ F$ is faithful, then $f = g$. 
\subsection{Types of arrows}
\begin{definition}[Source and sink]
	 A \textbf{source} is a pair $(Y, (f_i))$ consisting of an object $Y$  of a category $\mathcal{C}$, and a family of morphisms $(Y \stackrel{f_i}{\to} X_i)_{i \in I}$ over a class $I$. Equivalently it is a family of objects in the under category  $\mathcal{C}_{Y/}$. 
	 
	 For short we will use the notation $(Y \to X_i)_I$
	 
	 If $I$ is a finite index set $\{1, \cdots, n\}$, we call our source an \textbf{$n$-source}. 
	 
	 The dual concept to a source is called a \textbf{sink}. 
	 \\
\end{definition}
\
\begin{definition}
	A source $\mathcal{S} = (A \stackrel{f_i}{\to}A_i)_I$ is called a \textbf{mono-source} if it cancels from the left, i.e., if for any two parallel morphisms $B \stackrel[h]{g}{\rightrightarrows A}$ the equation $ \mathcal{S} \circ g = \mathcal{S} \circ h$, that is, if $f_i \circ g = f_i \circ h$ for all $i \in I$, implies $g = h$.
\end{definition}
For the following definitions, let $G: \mathcal{A} \to \mathcal{B}$ be a functor, and $B \in \Ob(\mathcal{B})$.

\

\begin{definition}[$G$-structured map]
		A \textbf{$G$-structured arrow with domain $B$} is a a pair $(f, A)$ consisting of an $\mathcal{A}$-object $A$ and a $\mathcal{B}$-morphism $f: B \to GA$. 
\end{definition}
\
\begin{definition}[$G$-structured lift]
	Let $(B \stackrel{\varphi_i}{\to} GA_i)_I$ be a $G$-structured source. If there exists an object $A$ and a map of morphisms  $(A \stackrel{f_i}{\to} A_i)_I$, for which there exists a map $GA \stackrel{h}{\to} B$ such that $Gf_i = \varphi_i \circ h$ for all $i \in I$, we call $A$ a \textbf{$G$-structured lift} of the source. 
	
	For notation we will interchangably refer to $A$ or the source $(A \to A_i)_I$ as the lift.
\end{definition}
\
\begin{definition}[Morphism of $G$-structured lifts]
	We call an  $\mathcal{A}$-morphism $A' \stackrel{\phi}{\to} A$ a \textbf{morphism of $G$-structured lifts} if there exists another lift $(A' \stackrel{f_i'}{\to} A_i)_I$ such that $GA' \to GA_i$ factors  through $GA \to GA_i$ for all $i \in I$. 
	
	That is, there exists a  morphism $GA' \stackrel{h'}{\to} B$ such that $h' = h \circ G\phi$ and $f_i' = f_i \circ \phi$. 
	
	
	\end{definition}
\begin{definition}
	The lift $A \in \Ob(\mathcal{A})$ of $(B \stackrel{\varphi_i}{\to} GA_i)_I$ is called a \textbf{$G$-initial lift} if  every $G$-structured lift $(A' \stackrel{f_i'}{\to} A_i)_I$ factors uniquely through $(A \stackrel{f_i}{\to} A_i)_I$.
\end{definition}
	
For the previous definitions we refer to the following commutative diagram for clarity:

\[\begin{tikzcd}
	{A'} && A && {} && {A_i} \\
	\\
	{GA'} && GA && B && {GA_i}
	\arrow["\phi", from=1-1, to=1-3]
	\arrow["{f_i'}", bend left, from=1-1, to=1-7]
	\arrow[from=1-1, to=3-1]
	\arrow["{f_i}", from=1-3, to=1-7]
	\arrow[from=1-3, to=3-3]
	\arrow[from=1-7, to=3-7]
	\arrow["{G\phi}", from=3-1, to=3-3]
	\arrow["{h'}", bend right, from=3-1, to=3-5]
	\arrow["h", from=3-3, to=3-5]
	\arrow["{Gf_i}", bend right, from=3-3, to=3-7]
	\arrow["{\varphi_i}", from=3-5, to=3-7].
\end{tikzcd}\]

\emph{Remark}. Often $h$ is the identity map, in which case our lift is called \textbf{strict}. However in practice we assume that $h$ is the identity unless stated otherwise.

\

Consider that here we mean initial in the sense of the initial or induced topology (in $(\Top, U)$ as  construct), however applied to arbitrary concrete category $(\mathcal{A}, U)$ over arbitrary category $\mathcal{X}$. 

That is, this lift is initial in the poset $(\Lift(B), \subset)$  of $G$-structured lifts of $B$ where the preorder is given by $A\subset A'$ if and only if $A'$ factors through $A$ as a lift, i.e. the morphisms are $A' \stackrel{\phi}{\to} A$ such that $(GA' \stackrel{h'}{\to}B) = (GA' \stackrel{G\phi}{\to}GA\stackrel{h}{\to}B) $. 

This is a very important concept throughout this paper, which deserves a remark about its intuition, which here should come from topology, where the initial topology is the limit topology, i.e., the coarsest topology on $GA$ making all $(A \stackrel{f_i}{\to}A_i)_I$ continuous. 

So for arbitrary category, we are looking for the weakest or initial $\mathcal{A}$-structure on $GA$ such that $(A \stackrel{f_i}{\to}A_i)_I$ are $\mathcal{A}$-morphisms, which ensures that for any $\mathcal{A}$-structure on $GA'$ such that $(A' \stackrel{f_i'}{\to}A_i)_I$ are $\mathcal{A}$-morphisms which factor through the lift $(A\stackrel{f_i}{\to} A_i)_I$, that this factorization $(A' \stackrel{\phi}{\to}A)$ is unique.
 
Formally this is a limit  in $\mathcal{A}$ over $\mathcal{A}$-structures on $GA$ with the property that all $(A \stackrel{f_i}{\to}A_i)_I$ are $\mathcal{A}$-morphisms.

\

\begin{definition}[cogenerator]
A \textbf{cogenerator} of a category $\mathcal{C}$ is an object $c \in \mathcal{C}$ such that the Hom-set functor $\HomC(-, c): \mathcal{C}^{\op} \to \Set$ is injective.

That is, given two maps $e \stackrel[f_2]{f_1}{\rightrightarrows} d$, if the induced maps $\HomC(d, c) \stackrel[- \circ f_2]{- \circ f_1}{\rightrightarrows}  \HomC(e,c)$ are equal, then $f_1 = f_2$, i.e., for any $\phi: d \to c$ it holds that $\phi \circ f_1 = \phi \circ f_2 \Rightarrow f_1 = f_2$. 

The dual concept is of a \textbf{generator}, which applies to $\HomC(c, -)$.
\end{definition}
\subsection{Topological and algebraic categories}
In the following we  introduce monotopological categories. 

Let $(A, U)$ be a concrete category over $\mathcal{X}$. 
\begin{definition}[Topological functor]
A functor $\mathcal{A} \stackrel{G}{\to}\mathcal{B}$ is called \textbf{topological} if every $G$-structured source has a unique $G$-initial lift. 
 \end{definition}
 
 \begin{definition}[Topological category]
 A concrete category $(\mathcal{A}, U)$ over $\mathcal{X}$ is said to be a \textbf{topological category} if $U$ is a topological functor. 
 \end{definition}
 
 Notice here again that our intuition of such a category should come from topology, which is even reflected in its name, namely, from the construct $(\Top, U)$,  over which every source lifts initially, since the arbitrary intersection of topologies is a topology. That is, we are taking the intersection of all topologies on $A$ such that all $(UA \stackrel{f_i}{\to} UA_i)_I$ are continuous.
 
  That means there is a systematic way to choose open sets of $UA$. In other words,  a subset $S \subset UA$ lifts to an open set in $A$ if and only if $S$ equals the preimage $Uf_i^{-1}(U(V))$ for some  $f_i \in (UA \stackrel{f_i}{\to} UA_i)_I$ and $V \in  \tau_{A_i}$ (\textcolor{red}{i.e., the topology of $A_i$: introduce this notation somewhere}).
 

\textbf{Notes on free objects and the free functor}

It is important to note that in some category $\mathcal{A}$ with a representing object $A_0$ that is a free generator on one free object, it holds that $\HomA(A_0,A) \cong A$ on the level of set, as the morphisms are uniquely determined by where they send the free generator to. 

More concretely, given a forgetful functor $U: \mathcal{A} \to \Set$, we have $\HomA(A_0,A) = U(A)$. This due to how we construct $A_0$ and $B_0$ using the the left adjoint functor $F: \Set \to \mathcal{A}$ to the forgetful functor, which is called the free functor. 

In general, such a functor does not have to exist, and we can still get the construction by defining $A_0$ as the left adjoint object of $U$ at $\{pt\} \in \Set$, however, we will assume that such a functor does exist for purposes of explication: $A_0 = F(\{\text{pt}\})$. Such an assumption is fair since the properties of the adjunction by definition of the left adjoint object will still hold (in particular the middle equality of the equation below), regardless of existence of the functor as a whole: \begin{align*}
	\HomA(A_0, A) = \HomA(F(\{\text{pt}\}), A) = \HomB(\{pt\}, U(A)) = U(A).
\end{align*}

The left adjoint object to the forgetful functor has the following universal property (\textcolor{red}{TODO}).

In the  examples which are to be discussed in this thesis, we only need to check that such a left adjoint object does exist in the category, and we show what they are and that they satisfy the universal property.
\section{Notes: Concrete Dualities}	


\textcolor{red}{BIGTODO}:
Edit all relevant parts of this section that I wrote before I understood concrete categories and identified $S$ and $T$ with $\HomA(-,\t A)$ and $\HomB(-,\t B)$ directly, rather than by means of the forget functor.

Let $\mathcal{A}$ and $\mathcal{B}$ be two categories with the following representable functors with representing objects $A_0 \in \mathcal{A}$, and $B_0 \in \mathcal{B}$. We assume these are free objects on one free generator, which we will later make precise. 
\begin{align*}
	&\HomA(A_0, -) \cong U: \mathcal{A} \to \Set\\
	&\HomB(B_0, -) \cong V: \mathcal{B} \to \Set\\
\end{align*}
Let $T: \mathcal{A} \to \mathcal{B}$ and $S: \mathcal{B} \to \mathcal{A}$ be contravariant functors with natural transformations $\eta: 1_\mathcal{B} \to TS$ and $\epsilon :  1_\mathcal{A} \to ST$. We can view this as the following \emph{dual adjunction}
\[\begin{tikzcd}
	{\mathcal{A}} & \bot & {\mathcal{B}^{\op}}
	\arrow["T", bend left, from=1-1, to=1-3]
	\arrow["S", bend left, from=1-3, to=1-1]
\end{tikzcd}\]

When these natural transformations are natural isomorphisms we are speaking of a \emph{dual equivalence}. 

Like any adjunction we have the following triangle equalities \begin{align*}
	&T{\epsilon_A} \circ \eta_{TA} = 1_{TA} \ &\text{and}& \ &  S{\eta_B} \circ \epsilon_{SB} = 1_{SB}&
\end{align*}
and $\Hom$ set isomorphisms \begin{align*}
	\HomA(A, SB) \cong \HomB(B, TA) \cong \HomB^{\op}(TA,B)
\end{align*}

Given some $\mathcal{A}$-morphism $f: A\to A'$, consider the map $Uf: UA \to UA'$. More explicitly, this is \begin{align*}
 	\HomA(A_0, -) (f) : \HomA(A_0, A) \stackrel{f \circ (-)}{\to} \HomA(A_0, A')
 \end{align*}
We shorten this notation by using $[f]: [A] \to [A']$.

The following pair of objects will be of central importance to this these, which are defined as the following:
\begin{align*}
	\t A := SB_0&&
	\t B := TA_0.
\end{align*}


From these characteristics we can deduce how $S$ and $T$ should be defined, to which a few lemmas will illuminate the bigger picture of our situation.
\begin{lemma}
\begin{align*}
		VT \cong \HomA(-, \t  A)&&
	US \cong \HomB(-, \t  B)\\
\end{align*}
\end{lemma}
\begin{proof}
As presheafs, $V$ and $VT$ (respectively $U$ and $US$) may be computed pointwise. 
\begin{align*}
	VT(A) \cong \HomB(B_0, TA) \cong \HomA(A,SB_0) = \HomA(A, \t A) \\
	\implies VT \cong \HomA(-, \t A)
\end{align*}
\begin{align*}
	US(B) \cong  \HomA(A_0, SB) \cong  \HomB(B,TA_0) =  \HomB(B, \t B) \\
	\implies US \cong \HomB(-, \t B)
\end{align*}
\end{proof}
Should we have strict identities
\begin{align*}
		VT = \HomA(-, \t  A)&&
	US = \HomB(-, \t  B)\\
\end{align*} 

we say that the adjunction is \emph{strictly represented} by $\t A$ and $\t B$. 

Given our assumption that $A_0$ and $B_0$ are free objects on one free generator, this result should already give us an idea of how our adjunction is to be induced, which the goal of the whole introduction to concrete dualities is to make precise. 

However the broad strokes of it uses the free-forget  adjunction ($\bar F \dashv \bar U$)  to show via \begin{align*}
	\HomA(A, \t A) = VT(A) = \HomB(B_0, T A) = \bar U(TA)
\end{align*} 
that the underlying sets of $T$ and $S$ respectively, are $\Hom_A(-, \t A)$ and $\HomB(-, \t B)$. \footnote{In the course of this thesis we will often prove results about $\mathcal{A}$, respectively $T: \mathcal{A}\to \mathcal{B}$ with unit $\epsilon:1_\mathcal{A} \to ST $, from which the results about $\mathcal{B}$, respectively $S: \mathcal{B} \to \mathcal{A}$ with counit $\eta: 1_\mathcal{B} \to TS$, follow completely analogously. Unless we state that results do not follow analogously, we assume this to be the case.}

\begin{lemma}
	$V \t B \cong U \t A$
\end{lemma}
\begin{proof}
	\begin{align*}
		V \t B \stackrel{Def. V}{=}& \HomB(B_0,\t B) \stackrel{Def. \t B}{=} \HomB(B_0, TA_0)\\ \stackrel{Adjunction}{\cong}& \HomA(A_0,SB_0) \stackrel{Def. \t A}{=}\HomA(A_0,\t A) \stackrel{Def. U}{=} U\t A
	\end{align*} 
	\end{proof}
	
	The aim of the following will be to show how the objects $\t A$ and $\t B$ actually induce the adjunction $T \dashv S$. In doing so, we first show that the units and counits of the adjunction are given \emph{by evaluation}, that is \begin{align*}
		\epsilon_A(x): \HomA(A, \t A) &\to \t B& \eta_B(y) : \HomB(B, \t 
		B) &\to \t A\\
		f & \mapsto f(x)& g &\mapsto g(y).
	\end{align*}
	
	In the following we define the canonical "evaluation" maps, $\varphi_{A, x}$ and $\psi_{B, y}$ and the canonical bijections $\tau$ and $\sigma$:  
\begin{align*}
	\varphi_{A, x} : \HomA(A, \t A) &\to [\t A ]& \psi_{B, y} : \HomB(B, \t B) &\to [\t B ]\\
	 s &\mapsto [s](x) &t &\mapsto [t](y) \\\\
	 \tau: [\t A] &\to [\t B]&\sigma: [\t B] &\to [\t A]\\
	 \t x &\mapsto  [[\epsilon_{\t A}](\t x)](1_{\t A})&\t y &\mapsto  [[\eta_{\t B}](\t y)](1_{\t B})
\end{align*}
which evaluate the maps $[s]$ and $[t]$ at $x$ and $y$ respectively:
\begin{align*}
	[s]: [A] &\to [\t A]&[t]: [B] &\to [\t B]\\
	x &\mapsto [s](x)& y &\mapsto [t](y)
\end{align*}
as for any $s \in \HomA(A, \t A) $, we have the induced map $[s]: [A] \to [\t A]$.

So for every $x \in \HomA(A_0,  A)$ we have the following diagram.
\[\begin{tikzcd}
	s && {[s](x)} && {\tau([s](x))} \\
	\HomA(A, \t A) && \HomA(A_0, \t A) && \HomB(B_0, \t B) \\
	&& \HomA(A_0, SB_0) && \HomB(B_0, TA_0)
	\arrow[maps to, from=1-1, to=1-3]
	\arrow[maps to, from=1-3, to=1-5]
	\arrow["\varphi_{A,x}", from=2-1, to=2-3]
	\arrow["\tau", from=2-3, to=2-5]
	\arrow["\cong",from=2-3, to=3-3]
	\arrow["\cong",from=2-5, to=3-5]
	\arrow[from=3-3, to=3-5]
\end{tikzcd}\]
By the dual adjunction we have $\HomA(A, SB_0) \cong \HomB(B_0, TA)$.




\begin{lemma}
	$\tau$ and $\sigma$ are inverses, and the following identities hold:
	\begin{align*}
		&[[\epsilon_A](x)] = \tau \varphi_{A,x}& & [[\eta_B](y)]= \sigma \psi_{B,y}
	\end{align*}
\end{lemma}
\begin{proof}
	First we check the identities, as understanding them will help us prove that $\tau$ and $\sigma$ are inverses. We only check the left identity, and as the right identity will follow analogously. 
	
	First we have $\tau \varphi_{A,x}(s) = \tau([s](x))$ by definition of $\varphi_{A,x}$. But then by definition of $\tau$ we have $\tau([s](x)) = [[\epsilon_{\t A}][s](x)](1_{\t A})$. 
	
	Since $\epsilon$ is a natural transformation, we have the following commutative diagram for all $A, A' \in \mathcal{A}$ such that there exists a map $A \to A'$. In particular, given $s: A \to \t A$ we have:
\[\begin{tikzcd}
	{1_\mathcal{A}(A)} && {ST(A)} \\
	\\
	{1_\mathcal{A}(\t A)} && {ST(\t A)}
	\arrow["{\epsilon_A}", from=1-1, to=1-3]
	\arrow["s"', from=1-1, to=3-1]
	\arrow["STs", from=1-3, to=3-3]
	\arrow["{\epsilon_{\t A}}"', from=3-1, to=3-3]
\end{tikzcd}\]
so that $[[\epsilon_{\t A}][s](x)](1_{\t A}) = [[STs][\epsilon_A](x)](1_{\t A})$. 

By Lemma 3.1, we know that $[STs] = US(Ts)  = \HomB(Ts, \t B)$, which is just a map  $\HomB(TA, \t B) \to \HomB(T\t A, \t B)$, induced by $Ts: T\t A \to TA$. \emph{Notice that $US(-)$ and $T(-)$ are both contravariant, so that $UST(-)$ is covariant. }

Now as  $[\epsilon_A](x) \in \HomB(TA, \t B)$, we have the induced precomposition

\[\begin{tikzcd}
	&& {} \\
	{T\t A} && {\t B} \\
	{TA}\\
	\arrow["{[\epsilon_A](x)\circ Ts}", dashed, from=2-1, to=2-3]
	\arrow["Ts"', from=2-1, to=3-1]
	\arrow["{[\epsilon_A](x)}"', from=3-1, to=2-3]
\end{tikzcd}\]
which can be otherwise phrased as a right action of $Ts$ on $[\epsilon_A](x)$ so that \begin{align}
[[STs][\epsilon_A](x)](1_{\t A}) = [[\epsilon_A](x)\circ Ts](1_{\t A}).
 \end{align}
% 
% Now we look more closely at what these maps do: Firstly, Ts is simply a precomposition map \begin{align*}
% 	Ts = \HomA(-,\t A): \HomA(\t A, \t A) &\stackrel{- \circ s}{\to} \HomA( A, \t A)\\ f &\mapsto f \circ s
% \end{align*}
% while $US(Ts)$ is the map
% \begin{align*}
% 	US(Ts) = \HomB(-,\t B)(Ts): \HomB(\HomA(A, \t A), \t B) &\stackrel{- \circ Ts}{\to}\HomB(\HomA(\t A, \t A), \t B)\\
% 	g^* &\mapsto g^* \circ Ts.
% \end{align*}
% The map $\epsilon_A(x)$ is given by evaluation, or in other words, it is that map that sends $p \in \Hom(A, \t A)$ to $p(x) \in [\t B]$, up to canonical isomorphism $[\t A] \cong [\t B]$ given by Lemma 1.1.
% 
% Now $Ts$ is sent by $1_{\t A}$ to $1_{\t A} \circ s$, and we see then that $[\epsilon_A](x) \circ Ts$ sends $1_{\t A} \mapsto 1_{\t A} \circ s(x) = s(x)$. 
% 
% Now we can see that the maps $[[\epsilon_A](x)\circ T(-)](1_{\t A})$ and $[[\epsilon_A](x)](-)$ both send $s \mapsto s(x)$, so that $[[\epsilon_A](x)\circ Ts](1_{\t A})=[[\epsilon_A](x)](s)$.
% 

From Lemma 3.1 we know that $VT = \HomA(-, \t A)$ so that we get the induced diagram
\[\begin{tikzcd}
	&& {} \\
	{\HomA(\t A, \t A)} && {[\t B]} \\
	{\HomA(A, \t A)}\\
	\arrow["{[[\epsilon_A](x)\circ Ts]}", dashed, from=2-1, to=2-3]
	\arrow["\text{[$Ts$]}"', from=2-1, to=3-1]
	\arrow["{[[\epsilon_A](x)]}"', from=3-1, to=2-3]
\end{tikzcd}\]
Notice that then $[Ts]$ becomes an evaluation of the precomposition functor $\HomA(-, \t A)$ at $s$, i.e. $[Ts] = - \circ s$, which sends $1_{\t A} \mapsto s$. Therefore it holds that \begin{align*}
 	[[\epsilon_A](x)\circ Ts](1_{\t A})=[[\epsilon_A](x)](s).
 \end{align*}
 All together we have \begin{align*}
 	\tau \circ \varphi_{A,x}(s)= &\tau([s](x))\\
 	=&[[\epsilon_{\t A}][s](x)](1_{\t A})\\
 	 =& [[STs][\epsilon_A](x)](1_{\t A})
 	\\=&[[\epsilon_A](x)\circ Ts](1_{\t A})\\
 	=&[[\epsilon_A](x)](s)
 \end{align*}
 which gives us the desired identity.
 
 Now we check that $\tau$ and $\sigma$ are inverses. 
 
 The above identity gives us the particular instance \begin{align*}
 	\tau \varphi_{S\t B, 1_{\t B}}(s) = [[\epsilon_{S\t B,\t y}](1_{\t B})](s)
 \end{align*}
 for all $s \in \HomA(S\t B, \t A)$.
 
 For $s = [\eta_{\t B}](\t y)$ with $\t y \in [\t B]$, noticing the maps \begin{align*}
 	[\eta_{\t B}](\t y): S\t B &\to \t A & \ \ \ \varphi_{S\t B, 1_{\t B}}: \HomA(S\t B, \t A) \to & [\t A]\\
 	1_{\t B} &\mapsto 1_{\t B}(\t y)& s \mapsto& [s](1_{\t B})
 \end{align*}
 we see that $[[\eta_{\t B}](\t y)](1_{\t B}) = \varphi_{S \t B, 1_{\t B}}(s)$ and so we  have \begin{align*}
 	\tau \sigma (\t y) =& \tau ( [[\eta_{\t B}](\t y)](1_{\t B}))\\
 	 =&  \tau(\varphi_{S\t B, 1_{\t B}}(s))\\
 	 =& [[\epsilon_{S\t B}](1_{\t B})][\eta_{\t B}](\t y) = \t y.
 \end{align*}
 We only need to show the last equality. 
 
 Consider the triangle equality $S\eta_{\t B} \circ \epsilon_{S\t B} = 1_{S \t B}$ which induces $[S\eta_{\t B}] [\epsilon_{S\t B}] = 1_{[S \t B]}$.
 
 We can extrapolate from (1)  using the result $US = \HomB(-, \t B)$ from Lemma 3.1 that for some $A \in \mathcal{A}$, $x \in A$, $B \in \mathcal{B}$ and $f \in \HomB(B, TA)$ the left action of $[Sf]$ on $[\epsilon_A](x) \in \HomB(TA, \t B)$ becomes a right action of $f$ on $[\epsilon_A](x)$, i.e. \begin{align*}
 	[Sf][\epsilon_A](x) = [\epsilon_A](x) \circ f \in \HomB(B, \t B).
 \end{align*}
 Therefore \begin{align*}
 	1_{\t B} = 1_{\HomB(\t B, \t B)}(1_{\t B}) =1_{[S \t B]}(1_{\t B}) = [S\eta_{\t B}] [\epsilon_{S\t B}](1_{\t B}) = [\epsilon_{S\t B}](1_{\t B}) \circ \eta_{\t B}.
 \end{align*}
 
 In other words, the induced map is the identity on $[\t B]$:\begin{align*}
 	[[\epsilon_{S\t B}](1_{\t B})][ \eta_{\t B}] = 1_{[\t B]}.
 \end{align*}
 
  Thus we have $\tau \sigma = 1_{[\t B]}$ as desired.
\end{proof}
% Recall that the following map is defined as evaluating at $1_{\t B}$:\begin{align*}
% 	[[\epsilon_{S\t B}](1_{\t B})]: [\HomA(\HomB(\t B , \t B), \t A)] &\to [\t B]\\
% 	p &\mapsto p(1_{\t B})
% \end{align*}
%% with \begin{align*}
%% 	[\epsilon_{S\t B}]:\HomB(\t B, \t B) &\to \HomB(\HomA(\HomB(\t B, \t B), \t A), \t B)\\
%%1_{\t B} & \mapsto (p \to p(1_{\t B}))\\
%% \end{align*}
%which means $[[\epsilon_{S\t B}](1_{\t B})][\eta_{\t B}](\t y) =[\eta_{\t B}](\t y)(1_{\t B})$ and so
%\begin{align*}
% 	[[\epsilon_{S\t B}](1_{\t B})][\eta_{\t B}]: [\t B] &\to [\t B]\\
% 	\t y &\mapsto [\eta_{\t B}](\t y)(1_{\t B}) = 1_{\t B}(\t y) = \t y.
% \end{align*}

% Remembering that $[S\eta_{\t B}][\eta_{S \t B}] = 1_{[S \t B]}$ 
% since \begin{align*}
% 	\eta_{\t B}: \t B \to \HomA(\HomB(\t B, \t B),\t A)
% \end{align*}
% means \begin{align*}
% 	S\eta_{\t B}: \HomB(\HomA(\HomB(\t B, \t B), \t A), \t B) \to \HomB(\t B, \t B)
% \end{align*}
% and \begin{align*}
% 	\epsilon_{S \t B}: \HomB(\t B, \t B) \to \HomB(\HomA(\HomB(\t B, \t B), \t A), \t B)
% 	 \end{align*}
% 	 evaluated at $1_{\t B}$ gives $[\epsilon_{S \t B}](1_{\t B}) = 1_{\t B}$, from which we \emph{somehow} deduce that \begin{align*}
% 	 	\tau \sigma = 1_{[\t B]}
% 	 \end{align*} (\textcolor{red}{TODO}: why?)
Lemma 3.3 makes precise the notion that our unit and counit are given "by evaluation", given
\begin{align*}
	[[\epsilon_A](x)]: \HomA(A, \t A) &\to [\t B] & [[\eta_B](y)]:\HomB(B, \t B) &\to [\t A]\\
	f &\mapsto f(x) & g &\mapsto g(y) 
\end{align*}
via canonical bijections $\tau$ and $\sigma$. From here we can (\textcolor{red}{TODO}: deduce or allude to?) deduce the fact that $T = \HomA(-, \t A)$ and $S = \HomB(-, \t B)$, as $\tau$ and $\sigma$ send the precise evaluation maps to the other pairing, and as such we can view the unit and counit as such:
\begin{align*}
	\epsilon_A:  A&\to \HomB(\HomA(A, \t A),\t B) & \eta_B: B&\to \HomA(\HomB(B, \t B),\t A)\\
	x &\mapsto (f \to f(x)) & y &\mapsto (g\to g(y) )
\end{align*}
\section{Schizophrenic Objects}
\subsection{Natural dual adjunction}
What we want for our set up is that the composition of these maps induces a $U$-structured lift, i.e. for every $A \in \mathcal{A}$ and  $x \in A$ there exists an $e_{A,x} \in \HomB(TA, \t B)$, that induces  $[e_{A,x}]: [TA] \to [\t B]$, such that $[e_{A,x}] = \tau \varphi_{A,x}$, i.e., the following diagram commutes: 
\[\begin{tikzcd}
	& TA && {\t B} \\
	\\
	{\HomB(B_0, TA)} & {\HomA(A, \t A)} && {\HomB(B_0, \t B)}
	\arrow["e_{A,x}",dashed, from=1-2, to=1-4]
	\arrow["U", from=1-2, to=3-2]
	\arrow["U", from=1-4, to=3-4]
	\arrow["{=}", shift right=1, draw=none, from=3-1, to=3-2]
	\arrow["{\tau \varphi_{A,x}}", from=3-2, to=3-4]
\end{tikzcd}\]
%Although we know the existence of an isomorphism given by the adjunction, the family of lifts criterium is stronger, requiring  a choice of isomorphism for every $A \in \mathcal{A}$ and $x \in A$.

We will shortly see that to obtain the desired dual adjunction, we may additionally require such a lift to be intial, which lends itself to a precise definition of the \emph{schizophrenic object}, which is the central notion of this thesis:

\begin{definition}
	A triple $(\t A, \tau, \t B)$ with a pair of objects $(\t A, \t B) \in \mathcal{A} \times \mathcal{B}$ and a bijective map $\tau: [\t A] \to [\t B]$ is called a \emph{schizophrenic object} (for concrete categories $\mathcal{A}$ and $\mathcal{B}$) if the following conditions are satisfied:
\begin{enumerate}
		\item[SO1.] For every $A \in \mathcal{A}$ the family $(\tau\varphi_{A,x}: \HomA(A, \t A) \to [\t B])_{x \in [A]}$ admits a $V$-initial lifting $(e_{A,x}:TA \to \t B)_{x \in [A]}$
		\item[SO2.] For every $B \in \mathcal{B}$ the family $(\sigma\psi_{B,y}: \HomB(B, \t B) \to [\t A])_{y \in [B]}$ admits a $U$-initial lifting $(d_{B,y}:SB \to \t A)_{y \in [B]}$
	\end{enumerate}
\end{definition}

What these conditions actually mean is two-fold: Firstly, the $V$-structured lifting property yields the existence of  a $\mathcal{B}$-morphism $e_{A,x} \in \HomB(TA, \t B)$ for every $A \in \mathcal{A}$ and $x \in A$, such that $[TA] = \HomA(A, \t A)$ and $[e_{A,x}] = \tau \varphi_{A,x}$.

But secondly,  the $V$-initiality means that for any $Z \in \mathcal{B}$ and a map $h: [Z] \to [TA]$, if  all composite maps $\tau \varphi_{A,x} \circ h$ are the underlying-set maps for $\mathcal{B}$-morphisms in $\HomB(Z, \t B)$, then there exists a unique $\mathcal{B}$-morphism $h' \in \HomB(Z, TA)$  whose underlying set map is $h$.

In other words, the $V$-structured lift is initial among all such lifts: if $Z$ is any other $\mathcal{B}$-object whose underlying set maps into $\HomA(A, \t A)$ in a way that is compatible with all $\tau \varphi_{A,x}$ composites, then that map factors uniquely through $TA$ in $\mathcal{B}$. 



\[\begin{tikzcd}
	Z && TA && {\t B} \\
	\\
	{[Z]} && {\HomA(A, \t A)} && {[\t B]}
	\arrow["h'", from=1-1, to=1-3]
	\arrow["{ ({\tau\varphi_{A,x}\circ h})'}", bend left, from=1-1, to=1-5]
	\arrow[from=1-1, to=3-1]
	\arrow["{e_{A,x}}", from=1-3, to=1-5]
	\arrow["V", from=1-3, to=3-3]
	\arrow["V", from=1-5, to=3-5]
	\arrow["h", from=3-1, to=3-3]
	\arrow["{\tau  \varphi_{A,x}}", from=3-3, to=3-5]
\end{tikzcd}\]


We now show a central theorem to this thesis.

\begin{theorem}
	Every schizophrenic object $(\t A, \tau , \t B)$ induces a natural dual adjunction strictly represented by $(\t A, \t B)$, such that $\tau$ and $\sigma = \tau^{-1}$ are the canonical bijections defined in the previous section.
\end{theorem}
\begin{proof}
	First we show that $T$ and $S$ are well defined functors. The conditions (SO1.) and (SO2.) show us how $T$ and $S$ act on objects up to underlying-set isomorphism. 
	
	Now we show how $T$ acts on morphisms. To that effect, given some $f: A \to A'$, we seek to show the existence of $Tf: TA' \to TA$ whose underlying set map is $[Tf] = \HomA(f, \t A): \HomA(A', \t A) \to \HomA(A, \t A)$, which sends $s \mapsto s \circ f$. As we have just seen, by $(SO1)$ it suffices to show that $\tau \varphi_{A,x}\circ \HomA(f, \t A)$ are the underlying set maps of  $\mathcal{B}$-morphisms in $\HomB(TA', \t B).$
	
	Considering that $[Tf]$ is simply the precomposition map $- \circ f$, we see that given some $s \in \HomA(A', \t A)$, it holds that $\tau \varphi_{A,x}\circ \HomA(f, \t A)(s) =\tau \varphi_{A,x}(sf)$. But since $[sf](x) = [s][f](x)$, where $[f](x) \in [A']$, we have $\tau \varphi_{A,x}(sf) = \tau \varphi_{A',[f](x)}(s) = [e_{A',[f](x)}](s)$, which is the underlying set map of a $\mathcal{B}$-morphism in $\HomB(TA', \t B)$ by definition, and which exists by the lifting property given by $(SO1.)$. 
	
	Therefore $T$ and $S$ are well-defined functors, where preservation of the identity and the composition law follow the same logic as above, using $V$-initiality and the fact that the underlying-set map is defined by precomposition.
	
	Now we show that $T$ and $S$ are adjoint, and to do that we shall construct  unit and counit maps $\epsilon$ and $\eta$. In order to establish the existence of $\eta_B$ by playing the same game we first define $[\eta_B]: [B] \to [TSB]$ and show that each $\tau \varphi_{SB, t} \circ [\eta_B]$ with $t \in [SB]$, can be lifted along $V$, i.e. that it is the underlying set map of a $\mathcal{B}$-morphism in $\HomB(B, \t B)$. So we define in light of Lemma 3.3 under (SO2.)
	\begin{align*}
		[\eta_B]:[B] &\to \HomA(SB, \t A)\\
		y &\mapsto d_{B,y}.
	\end{align*}
Then by definitions and (SO2.) we have  \begin{align*}
	\tau \varphi_{SB, t} \circ [\eta_B](y) &= \tau \varphi_{SB, t} (d_{B,y})\\ &= \tau [d_{B,y}](t)\\ &= \tau \sigma \psi_{B,y}(t)\\ &= [t](y)
\end{align*}
which shows that $\tau \varphi_{SB, t} \circ [\eta_B]: [B] \to [\t B]$ is the underlying set map of a $\mathcal{B}$-morphism in $\HomB(B, \t B)$, proving the existence of $\eta_B$. 

Furthermore we see that \begin{align}
	e_{SB, t} \circ \eta_B = t \ \ \text{ for all } t \in [SB]= \HomB(B, \t B)
\end{align}
Since $[\eta_B]$ lifts for all $B$, we may verify naturality on underlying-set maps; we verify that given a $\mathcal{B}$-morphism $f: B \to B'$, that we have $ [\eta_{B'}]\circ (f \circ -) = [TSf] \circ [\eta_B]$, or in other words: \begin{align}
	d_{B', f \circ y} = [TSf] \circ d_{B,y}
\end{align} 
Remember that the left action of $[TSf]$ on $d_{B,y}$ is a right action of $Sf$ on $d_{B,y}$. But the underlying map of $d_{B,y} \circ Sf$ is $\sigma \psi_{B,y} \circ \HomB(f, \t B)$ which is, up to the bijection $\sigma$, just the evaluation map of a morphism in $\HomB(B, \t B)$ at some $y \in [B]$ precomposed with $f \in \HomB(B, B')$, which yields the evaluation map of a morphism in $\HomB(B',\t B)$ at $f \circ y \in [B']$. In other words $[d_{B,y}] [Sf] = [\sigma \psi_{B', f \circ y}] = [d_{B', f \circ y}]$, which is clearly the underlying set map of $d_{B', f \circ y}$, giving us (3) by uniqueness of the lift. 



The definition of $S$ gives us that $[S\eta_B][\epsilon_{SB}](t) = \HomB(\eta_B, \t B)(e_{SB, t})$ and by (2) we have $\HomB(\eta_B, \t B)(e_{SB, t}) = e_{SB,t} \circ \eta_B = t$. Since $U$ is faithful, any map $[Sf] \in \Set([SB'], [SB])$ is the underlying-set map to a unique map $Sf \in \HomA(SB', SB)$, and we deduce that $[S\eta_B][\epsilon_{SB}] = 1_{[SB]}$ is the underlying set map of the triangle identity $S\eta_B \circ \epsilon_{SB} = 1_{SB}$.

Finally, to show that $\tau$ is induced by this adjunction it suffices to see that it maps $\t x \mapsto [[\epsilon_{\t A}](\t x)](1_{\t A})$ as desired. For every $\t x \in [\t A]$ we have
\begin{align*}
	[[\epsilon_{\t A}](\t x)](1_{\t A}) = \tau \varphi_{\t A, \t x}(1_{\t A}) = \tau([1_{\t A}](\t x)) = \tau(\t x).
\end{align*}
\end{proof}
A dual adjunction induced by a schizophrenic object in this way is called \emph{natural}. However there exist also dual adjunctions which are not natural in this sense, although some modifications are in order to make it natural. 
\subsection{Non-natural dual adjunction}

Dual adjunctions which are not induced by a schizophrenic object are called \emph{non-natural}. The key difference here is initiality of the lifts. That is, an arbitrary dual adjunction $(S', T')$ in the situation described in $\S 3$ already determines a triple $(\t A, \tau, \t B)$ such that the following weakened conditions of $SO1$ and $SO2$ are fulfilled:
\begin{enumerate}
		\item[WSO1.] For every $A \in \mathcal{A}$ the family $(\tau\varphi_{A,x}: \HomA(A, \t A) \to [\t B])_{x \in [A]}$ admits a $V$-structured lift $(e_{A,x}:T'A \to \t B)_{x \in [A]}$ which extends functorially, i.e., for every $A \stackrel{f}{\to} A'$ in $\mathcal{A}$ there exists a $\mathcal{B}$-morphism $T'A' \stackrel{T'f}{\to}T'A$ with $[T'f] = \HomA(f,\t A)$. 
		\item[WSO2.] For every $B \in \mathcal{B}$ the family $(\sigma\psi_{B,y}: \HomB(B, \t B) \to [\t A])_{y \in [B]}$ admits a $U$-structured lift $(d_{B,y}:S'B \to \t A)_{y \in [B]}$ which extends functorially.
	\end{enumerate}
	
Though $(S',T')$ determins a triple, such a triple does not necessarily induce $(S',T')$ like the schizophrenic object induces $(S,T)$. In particular, a non-natural dual adjunction is not a dual equivalence, i.e., $\epsilon$ and $\eta$ are not natural isomorphisms. (\textcolor{red}{TODO/Question:} Why are $\epsilon$ and $\eta$ not isomorphisms?).

However, if we are in this situation, there are potential modifications which may give us a natural dual adjunction. There are in particular two methods which we will remark. 

Firstly, one may use the triple $(\t A, \tau, \t B)$ to induce a natural dual adjunction $(S,T)$ on the concrete categories $(\mathcal{A}, U)$ and $(\mathcal{B},V)$. Such a method requires additional assumptions on $(\mathcal{A},U)$ and $(\mathcal{B},V)$ which we will not discuss. For more details, see [1 1-D]. 

The second method is the one which we will later see in action in our examples, which is to restrict our dual adjunction to full subcategories of our categories under which the unit and counit are isomorphisms. 

In the next part we will discuss a situation which induces a non-natural dual adjunction between concrete categories.
\subsection{Internal Hom-Functors}

Let $(\mathcal{A}, U)$ be a concrete category. An \textbf{internal hom-functor} is a functor $H: \mathcal{A}^{\op} \times \mathcal{A} \to \mathcal{A}$ such that $UH = \HomA(-,-)$. Moreover \begin{align*}
	\text{all evaluation maps } \phi_{A,A',x}: \HomA(A,A') &\to [A']\\
	h &\mapsto [h](x)\\
	\text{lift to $\mathcal{A}$-morphisms } p_{A,A',x}: H(A,A') &\to A' \text{ for all } A, A' \in \mathcal{A}, x \in [A]
\end{align*}
Any cartesian closed concrete category which admits function spaces is a good example of a category with internal hom-functors. Recall the following definitions (from [2]):

\
\begin{definition}[Cartesian closed category] A category $\mathcal{A}$ is \textbf{cartesian closed} if it has finite products and for each $\mathcal{A}$-object $A$ the functor $(A \times -)$ has a right adjoint  $(-)^A$, called the \textbf{Heyting implication}. For $B \in \mathcal{A}$ we call $B^A$ an \textbf{exponentiable object}. 
\end{definition}
\

\begin{definition}
	A construct $ \concA$ is said to admit \textbf{function spaces} if $\mathcal{A}$ is cartesian closed, $\concA$ admits finite concrete products, and the evaluation morphisms $A \times B^A \stackrel{\ev}{\to} B$ can be chosen in such a way that $U(B^A)= \HomA(A,B)$ where $\ev$ is the restriction of the canonical evaluation map in $\Set$.  \end{definition}

A cartesian closed concrete category which admits function spaces admits an internal hom-functor by definition, since we can choose $B^A$ to be our lift.

 
Now we want to see how and under what conditions can  a category which admits an internal hom-functor  induce a dual adjunction. Let $\concA$ admit an internal hom-functor $H$, and let there be a concrete category $\concB$ and a concrete functor $\lvert - \lvert : \mathcal{B} \to \mathcal{A}$ such that $\HomB(B, C) \hookrightarrow \HomA(\lvert B \lvert, \lvert C \lvert)$ lift to $\mathcal{A}$-morphisms\footnote{$\HomB_\mathcal{A}(-,-)$ is notation for the hom set in $\mathcal{B}$ as $\mathcal{A}$-object} $\gamma_{B,C}: \HomB_\mathcal{A}(B,C) \to H(\lvert B \lvert, \lvert C\lvert)$.  In a monotopological category, this can be done by lifting initially.

Given $\t B \in \mathcal{B}$ with $\t A := \lvert \t B \lvert $ and $\tau = 1_{[\t A]}$, we can check that $WSO2$ is fulfilled. In other words we are seeking a lift of the map $\sigma \psi_{B,y}: \HomB(B, \t B) \to [\t A]$. But such a lift is given by the internal hom-functor, given that the bottom part of the following diagram commutes:
\[\begin{tikzcd}
	{\HomB_\mathcal{A}(B, \t B)} && {H(\lvert B\lvert, \t A)} && {\t A} \\
	\\
	{\HomB(B, \t B)} && {\HomA(\lvert B \lvert , \t A) } && {[\t A]}
	\arrow["{\gamma_{B, \t B}}", from=1-1, to=1-3]
	\arrow[from=1-1, to=3-1]
	\arrow["{p_{\lvert B \lvert, \t A, y}}", from=1-3, to=1-5]
	\arrow[from=1-3, to=3-3]
	\arrow[from=1-5, to=3-5]
	\arrow["\iota", hook, from=3-1, to=3-3]
	\arrow["{\sigma \varphi_{B,y}}", bend right, from=3-1, to=3-5]
	\arrow["{\phi_{\lvert B\lvert , \t A,x }}", from=3-3, to=3-5]
\end{tikzcd}\]

That is, if $\sigma \varphi_{B,y} = \phi_{\lvert B\lvert, \t A, x} \circ \iota$ then we have $d_{B, y} = p_{\lvert B \lvert, \t A, y} \circ \gamma_{B, \t B}$, which gives us $WSO2$. 

The question is if the concrete functor commutes with the evaluation map, since $\t A = \lvert \t B \lvert $ and $\tau$ (and therefore also $\sigma$) is the identity.

\[\begin{tikzcd}
	{\HomB(B, \t B)} && {[\t B]} \\
	\\
	{\HomA(\lvert B \lvert, \t A) } && {[\t A]}
	\arrow["{\varphi_{B, y}}", from=1-1, to=1-3]
	\arrow["i"', hook', from=1-1, to=3-1]
	\arrow["\sigma", from=1-3, to=3-3]
	\arrow["{\phi_{\lvert B \lvert, \t A, x}}"', from=3-1, to=3-3]
\end{tikzcd}\]

In other words, we check $\sigma([f](y)) = [\lvert f \lvert ](\iota(y))$, but this follows from faithfulness of the concrete functor $\lvert - \lvert $. (\textcolor{red}{TODO/Question}: is this true/really so immediate?)

Therefore we have a contravariant functor $S(B) = \HomB_\mathcal{A}(B, \t B)$. 

Now if for every $A \in \mathcal{A}$ we can lift the $\mathcal{A}$-source $(p_{A, \t A, x}: H(A, \t A) \to \t A)$ along $\lvert - \lvert$ functorially, then we also have WSO1, inducing the contravariant functor $T: \mathcal{A} \to \mathcal{B}$ such that $\lvert T(A) \lvert = H(A, \t A)$, and as such we have the desired dual adjunction $(S, T)$. We will however only show this example-wise. 


\section{Motivation}
We are now in the position to understand what this schizophrenic object really affords us, however first we will explicate what that is by  saying a few words about motivation that will hopefully make the above seemingly abstract definition of the schizophrenic object more understandable. 

First remember that there is a bijection between $[\t A]$ and $[\t B]$, since  \begin{align*}
	[\t A] = \HomA(A_0, SB_0) = \Hom_B(B_0, TA_0) = [\t B].
\end{align*}

In particular, we have shown that $\tau$ and $\sigma$ are necessarily such bijections and inverses of one another. 

What is interesting about the schizophrenic object is that the object itself defines the adjunction via $\Hom$ sets in our respective categories. In other words, our question, in general, is that given a schizophrenic object $(\t A, \tau, \t B)$ do the sets $\Hom_A(A, SB_0)$ have $\mathcal{B}$ structure for every $A \in \mathcal{A}$ and $\HomB(B, TA_0)$ have $\mathcal{A}$ structure for every $B \in \mathcal{B}$, and moreover, are they the initial objects with $\mathcal{B}$-structure. 

More explicitly, do we have initial lifts via the forgetful functor, such that the following diagrams commute for every $A \in \mathcal{A}$ and for every $B \in \mathcal{B}$?

\[\begin{tikzcd}
	&& {\mathcal{B}^{\op}} &&&&& {\mathcal{A}^{\op}} \\
	&& {} && {\text{and}} \\
	{\mathcal{A}} && \Set &&& {\mathcal{B}} && \Set \\
	A && {\HomA(A, \t A)} &&& B && {\HomB(B, \t B)} \\
	&&&&& {}
	\arrow["{\t U(-)}", from=1-3, to=3-3]
	\arrow["{\t U(-)}", from=1-8, to=3-8]
	\arrow["\exists", dashed, from=3-1, to=1-3]
	\arrow["{\HomA(-,\t A)}"', from=3-1, to=3-3]
	\arrow["\exists", dashed, from=3-6, to=1-8]
	\arrow["{\HomB(-,\t B)}"', from=3-6, to=3-8]
	\arrow[maps to, from=4-1, to=4-3]
	\arrow[maps to, from=4-6, to=4-8]
\end{tikzcd}\]
If such lifts exist, then by construction, we will have the following adjunction
\[\begin{tikzcd}
	{\mathcal{A}} & \bot & {\mathcal{B}^{\op}}
	\arrow["{T:=\HomA(-, \t A)}",  bend left, from=1-1, to=1-3, from=1-1, to=1-3]
	\arrow["{S:=\HomB(-, \t B)}", bend left, from=1-3, to=1-1, from=1-3, to=1-1]
\end{tikzcd}\]

To understand this adjunction and see that it does indeed correspond to the morphisms explained in the first section, we will go through several examples.

\section{Examples}
For the examples note that we may sometimes by abuse of notation and for lack of a better functorial description use $\HomA(-,\t A)$ to denote $T$, unless we denote otherwise.
\section{Leading example}

%\emph{A note about notation: When speaking about $\Hom$ sets of a general category $A$, we use the notation as before $\HomA(-,-)$. However to shorten notation and ease readability, when speaking of specific categories, such as $\Frm, \Top,$ or $\Set$, we use the notation $\Frm(-,-), \Top(-,-),$ $ \Set(-.-).$}

We begin with the adjunction 

\[\begin{tikzcd}
	{\Frm} & \bot & {\Top^{\op}}
	\arrow["{\text{T = $\Frm$(--, $\mathbb{S}$)}}" , bend left, from=1-1, to=1-3]
	\arrow["{\text{S = $\Top$(--, $\mathbb{2}$)}}", bend left, from=1-3, to=1-1]
\end{tikzcd}\]

with the following unit and counit
\begin{align*}
	\tau:& [\mathbb{2}] \to [\mathbb{S}] & \sigma: [\mathbb{S}]&\to [\mathbb{2}]\\
	&\t x \to  [[\epsilon_{\mathbb{2}}](\t x)](1_{\mathbb{2}})&\t y &\mapsto [[\eta_{\mathbb{S}}](\t y)](1_{\mathbb{S}})
\end{align*}
In this setting  the respective representable objects are $A_0 = a$ and $B_0 = \{pt\}$, where $a$ is here  notation for the free frame generated by one object, i.e., the free 3-chain $\bot - a - \top$. 

Now we know that $[TA_0] = \Frm(a, \mathbb{2}) = \mathbb{2}$ and $[SB_0] = \Top(\{pt\}, \mathbb{S}) = \mathbb{S}$, however it less immediate  why $\Frm(A, \mathbb{2})$  lifts to the category $\Top$ and why $\Top(B, \mathbb{S})$ lifts to the category $\Frm$.

For a $\Frm$-structure on $\Top(B, \mathbb{S})$ we desire a lift which preserves the evaluation map $\tau\psi_{B,y}$, and which is initial among such lifts. This means we want the weakest $\Frm$-structure such that for all $y \in B$ and $B \in \Top$ evaluation maps $\tau \psi_{B,y}$ lift to frame homomorphisms $d_{B,y}$.

We know that the evaluation is a frame homomorphism if and only if it preserves pointwise order in $\mathbb{2}$, since   $d_{B,y}(u \leq v) = d_{B,y}(u) \leq d_{B,y}(v) = u(y) \leq v(y)$, and conversely a pointwise order in $\mathbb{2}$ determines limits and colimits by definition;  $u(y) \land v(y) \leq (u \land v)(y)$ holds since the pointwise intersection is less than or equal to $u(y)$ and $v(y)$ respectively, and $(u \land v)(y) \leq u(y) \land v(y)$ holds since the intersection in the frame is less than or equal to $u$ and $v$ respectively and  evaluating preserves pointwise order.
	

(\textcolor{blue}{TODO} does a pointwise order preserving evaluation morphism into $\mathbb{2}$ preserve finite limits and colimits by some pointwise argument in the presheaf category? in other words, is there a way to abstract this argument so i dont have to prove it on elements?)



So for $B \in \Top$ we construct the $\Frm$-structure on $\Top(B, \mathbb{S})$ by a preorder that preserves the pointwise order in $\mathbb{2}$ for all $y \in B$. As we want the weakest such structure, we define it such that for  arbitrary $ u, v \in \Top(B, \mathbb{S})$ we have $u \leq v$ if and only if $u(y) \leq v(y)$ for all $y \in B$. 

This is the initial $\Frm$-structure making all evaluation maps frame homomorphisms.




For a topology on $\Frm(A, \mathbb{2})$, we consider the family \begin{align*}
	\{ \  \{p \in \Frm(A, \mathbb{2}) \   \lvert \  p(x) = 1 \} \ \lvert \ x \in A \ \} 
\end{align*} 

Our adjunction has the unit
\begin{align*}
	\epsilon:& 1_{\Frm} \to ST\\
	&A \mapsto STA\\
	&A \mapsto \Top(\Frm(A, \mathbb{2}), \mathbb{S})
\end{align*}
so that 
\begin{align*}
	\epsilon_A:& A \to \Top(\Frm(A, \mathbb{2}), \mathbb{S})
\end{align*}
where 
\begin{align*}
	\epsilon_A(x):& \Frm(A, \mathbb{2}) \to \mathbb{S}\\
	&p \mapsto p(x)
\end{align*}
Notice that the topology on $\Frm(A, \mathbf{2})$ is the initial topology making all $(\epsilon_A(x))_{x \in A}$ continuous.

When passing to $U$ and $V$, we have
\begin{align*}
	[\epsilon_{\t A}]: [\t A] &\to [\Top(\Frm(\t A, \mathbb{2}), \mathbb{S})] = \Top(\Frm(\t A, \mathbb{2}), \mathbb{S})\\
	(\t x: a \to x) &\mapsto (p \to p(x))
\end{align*}
which is equal to
\begin{align}
	[\epsilon_{\mathbb{2}}]: [\mathbb{2}] &\to [\Top(\Frm(\mathbb{2}, \mathbb{2}), \mathbb{S})] = \Top(\Frm(\mathbb{2}, \mathbb{2}), \mathbb{S})\\
	(\t x: a \to x) &\mapsto (1_{\mathbb{2}} \to 1_{\mathbb{2}}(x))
\end{align}
So we have 
\begin{align*}
	[\epsilon_{\mathbb{2}}](\t x): \Frm(\mathbb{2},\mathbb{2}) &\to \mathbb{S}\\
	1_{\mathbb{2}} &\mapsto 1_{\mathbb{2}}(x)
\end{align*}
and now
\begin{align*}
	[[\epsilon_{\mathbb{2}}](\t x)]: [\Frm(\mathbb{2},\mathbb{2})] &\to [\mathbb{S}]\\
	(a \to 1_{\mathbb{2}}) &\mapsto (\{pt\} \to1_{\mathbb{2}}(x))
\end{align*}
so that 
\begin{align*}
	[[\epsilon_{\mathbb{2}}](\t x)](1_{\mathbb{2}}) = (\{pt\} \to 1_{\mathbb{2}}(x)).
\end{align*}
Now we can see that the map \begin{align*}
	\tau: [\mathbb{2}] &\to [\mathbb{S}]\\
	(\t x: a \to x) &\mapsto (\{pt\} \to 1_{\mathbb{2}}(x))
\end{align*}
is the identity in $\Set$, as the underlying set on both sides is $[\mathbb{2}] = [\mathbb{S}] = \{0,1\}$ so that we are looking at the set map \begin{align*}
	\{0,1\} &\to \{0,1\}\\
	x &\mapsto 1_{\{0,1\}}(x)= x
\end{align*}

This map is clearly the identity. This boils down to the fact that our choice of morphism from $\Frm(\mathbb{2},\mathbb{2})$ was easy to determine since the set $\Frm(\mathbb{2},A) = \{pt\}$, as $\mathbb{2}$ is an initial object in $\Frm$, and in particular it is clear in $(1)$ that $\Frm(\mathbb{2},\mathbb{2}) = \{1_{\mathbb{2}}\}$. From this we can deduce that $\tau$ is the identity on $\bar U(\mathbb{2})$.
In general, our schizophrenic object will not necessarily be initial in arbitrary concrete duality, and as such, our choice $p \in \HomA(\t A, \t A)$ may not be unique nor easy to determine, so that $\tau$, though always a bijection, is not necessarily always the identity. 

\section{Stone duality}

(\textcolor{blue}{TODO}: introduce boolean algebras, as well as ideals, filters, ultrafilters, principle ultrafilters, maybe in the beginning? question for Georg) 


The following section will take some development to get to the actual Stone duality, so we describe our strategy as follows: first we describe a non-natural dual adjunction between $\FinSet$ and $\Bool_{fin}$ which determines the triple of our sought after Stone duality, however for the wrong categories.

Then we describe how to obtain the Stone duality via restriction and composition of other dualities, namely the dualities  $\Loc \rightleftarrows \Top$ and   $\DLat \rightleftarrows \CohTop$.

And finally we will show why the original triple is a schizophrenic object for the Stone duality.
\subsection{Duality between finite boolean algebras and finite sets}
The leading example is modeled from a more general duality, called the Stone duality. Before we speak about the Stone duality let us discuss a more general situation.

Consider the category $\Bool$ of Boolean algebras and boolean algebra homomorphisms between them. We claim there is an adjunction between $\Bool_{fin}$ and $\FinSet$ given by $\Ult(A)$, which sends a boolean algebra $A$ to the set $\Ult(A)$ of ultrafilters on $A$, and by $\mathcal{P}(X)$, which sends a set $X$ to its power-set.

\[\begin{tikzcd}
	\Bool_{\text{fin}} & \bot & \FinSet^{\op}
	\arrow["{\Ult(-)}", shift left=2, bend left, from=1-1, to=1-3]
	\arrow["{\mathcal{P}(-)}", shift left=3, bend left, from=1-3, to=1-1]
\end{tikzcd}\]

Notice that $\mathcal{P}(X)$ is a boolean algebra, since every subset $S$ has a well-defined complement $X - S$. 


It is also the case that $\Ult(A)$ is  fully faithful as long as $A$ is a finite boolean algebra, since ultrafilters on those are principle, meaning that they are generated by a single element, an element which we can then identify with the points of $X$ in a faithful way. And so the counit $\Ult(\mathcal{P}(X)) \to X$ of the adjunction is an isomorphism.
 
Indeed, this implies that the unit $A \to \mathcal{P}(\Ult(A))$ is an isomorphism, since (\textcolor{blue}{TODO} prove this), and so the adjunction above is a dual equivalence.

However that is not to say it is a natural dual adjunction. 

In the non-natural case, remember that an adjunction determines a triple which could be a candidate for a schizophrenic object. We will however see later, that our candidate is not a schizophrenic object of this adjunction, but of a slightly mended construction.

To determine the triple first consider the free boolean algebra on one generator $a$ is the set $\diamondsuit : =\{\bot, a, \neg a, \top \}$, since our generator needs to induce complements (\textcolor{red}{TODO} and has finite limits and arbitrary colimits), and the free set on one generator is $\{pt\}$. 

Now  an ultrafilter on $\diamondsuit$ is determined by which element the bottom (and analogously top) associates to, which is an element of the set $\{a, \neg a\}= \mathbb{2}$, which we will prove in the following. It  therefore holds that  $\Ult(\diamondsuit) = \mathbb{2}$.

Meanwhile $\mathcal{P}(\{pt\}) = \{\emptyset, \{pt\}\} = \mathbb{2}$ as boolean algebra, since the point and the empty set are complements, and thus form the truth-value boolean algebra which is an initial object of $\Bool$. 

 Now in the following we want to show that such lifts exist, and we check that by showing that the adjunction we gave serves as a lift of the respective hom sets. 
 
For the power-set functor, it is easy to see that $\mathcal{P}(X) = \Set(X, \mathbb{2}) $ since any set map  $X \to \mathbb{2}$ is given uniquely by a subset $ S \subset X$, via characteristic functions \begin{equation*}
  \mathcal{X}_{ S}(x) = \begin{cases}
   1 & x \in  S \\
    0 & \text{else}.
  \end{cases}
\end{equation*}

On the other hand, for a boolean algebra $A$ we have the following bijection \begin{align*}
  	\phi: \Ult(A) \to& \Bool(A, \mathbb{2})\\
  	F \mapsto& f_F &f_F(x) = \begin{cases}
   1 & x \in F \\
    0 & \text{else}
  \end{cases}
  \end{align*}
The proof is straightforward and boils down to an equivalence between properties of boolean algebra homomorphisms and prime filter axioms. 

Lattice homomorphisms are  order preserving, finite limit preserving, and finite colimit preserving. 

Through the above map, we see that this corresponds to axioms of being a prime filter $F$, which are sub meet-semilattices (closed under finite meets), upwards closed, and the prime property: $\bot \notin F$ and $a \lor b \in F \implies a \in F $ or $b \in F$.


Frame homomorphisms additionally require preservation of arbitrary colimits, which simply upgrades the prime condition to completely prime, i.e. $\bigvee a_i \in F \implies \exists i \in I: a_i \in F$. However in our above setting of $A$ in $\Bool_{fin}$, this is no condition at all.

Now if our setting is Boolean, then we have that prime filters correspond to ultrafilters, which additionally require $A \in F \iff \ \forall B \in F: B \cap A \neq \emptyset$. 

 %\begin{lemma}
%	Let $I \subset P$ be an ideal of a lattice $P$. Then the following are equivalent 
%	\begin{enumerate}
%		\item The complement of $I$ is a filter $F \subset P$,
%		\item $1 \notin I$ and $I$ is prime.
%		\item $I$ can be given as the kernel of a lattice homomorphism $F \stackrel{f}{\rightleftarrows} \mathbb{2}$
%	\end{enumerate}
%\end{lemma}
%\begin{proof}
%	$(1) \implies (2)$: Since the complement of $I$ is a filter $F$, which contains the top element by upwards closure, we know the top element can't lie in $I$. Furthermore, $a \land b \in I$ implies that $a \in I$ or $b \in I$ by the fact that a filter is a sub meet-semilattice (if both $a, b \in F$ then so is $a \land b \in F$).
%	
%	$(2) \iff (3)$: One can check that  $f(a) =  \begin{cases}
%   1 & a \notin I \\
%    0 & a \in I
%  \end{cases}$ is a lattice homomorphism, and that the kernel of a lattice homomorphism defines a prime ideal. In fact, we will show this equivalence in the following lemma, but for frame homomorphisms. 
%  
%  \emph{Note that a lattice homomorphism is an order preserving, finite limit and finite colimit preserving morphism between lattices, while a frame homomorphism asks extra that it preserves arbitrary colimits, which we will see, upgrades the condition on the ideal (and dually to the filter) from being prime to being completely prime.}
%  
%  $(2) \& (3) \implies (1)$:  For ease we call $I = f^{-1}(0)$ and $F:= f^{-1}(1)$, and we implicitly use the fact that these are by definition set complements. Notice that the condition of being a prime ideal $a \land b \in I \implies a \in I$ or $b \in I$ is equivalent by contraposition to $a, b \in F \implies a \land b \in F$, so that $f^{-1}(1)$ is a sub meet-semilattice.
%  
%  Furthermore upwards closure of $F$ is derived directly from downwards closure of $I$. In other words, assuming $a, b \in P$ such that $a \leq b$, we have $a \in F \implies b \in F$ if and only if $ b \in I\implies a \in I$, which holds since $I$ is downwards closed.
%\end{proof}
%\begin{lemma}
%	There is a bijection \begin{align*}
%  	\phi: \Ult(A) \to& \Bool(A, \mathbb{2})\\
%  	F \mapsto& f_F &f_F(x) = \begin{cases}
%   1 & x \in F \\
%    0 & \text{else}
%  \end{cases}
%  \end{align*}
%  This restricts to a bijection $\phi': \prim(A) \to \Frm(A, \mathbb{2})$, where $\prim(A)$ is the set of all completely prime filters of $A$.
%\end{lemma}
%
%
% 
%\begin{proof}
% 	Our strategy will be to first identify  points of a frame $F \rightleftarrows \mathbb{2}$  with  completely prime filters on $A$ and then to show the ultrafilter condition in the boolean setting. 
% 	
% 	First we show that $f_F$ is finite limit, colimit, and order preserving. 
%  
% That $F$ is upwards closed is the condition that $x \in F$, $y \in A \implies y \in F$. This implies that
%  \begin{align*}
% 	f_F(x \wedge y) = 1 &\iff x \wedge y \in F\\ &\iff x \in F\text{ and } y \in F\\ &\iff f_F(x) \wedge f_F(y) =1 \
% \end{align*}
% 
% 
%  That $F$ is a sub meet-semilattice is the condition $x, y \in F \implies x \land y \in F$. This implies that. \begin{align*}
% 	f_F(x \wedge y) = 0 &\iff x \wedge y \notin F\\ &\iff x \notin F\text{ or } y \notin F\\ &\iff f_F(x)  \wedge f_F(y) =0 \
% \end{align*}
%
%
%That $F$ is a completely prime filter is the condition that $\bigvee x_i \in F$ implies that there exists an $i \in I$ such that $x_i \in F$, which implies that \begin{align*}
%	f_F( \bigvee x_i) = 1 &\iff \bigvee x_i \in F\\ &\iff \exists i \in I : x_i \in F\\
%	&\iff \bigvee f_F(x_i) = 1
%\end{align*}
%and similarly \begin{align*}
%	f_F( \bigvee x_i) = 0 &\iff \bigvee x_i \notin F\\ &\iff \forall i \in I:x_i \notin F \\
%	&\iff \bigvee f_F(x_i)  = 0
%\end{align*}
%Consider that all these statements are actually equivalences, since you can obtain each condition by swapping the two way implications from what was our condition---the second two way implication of each of these chains of equivalences---to what we have proven---each chains end statement.
%
%However note that when we swap we must think about $f_F$ as arbitrary boolean algebra (or frame) homomorphism $f$, and consider $I$ and $F$ as nothing but preimages under $F$, for example,
%
%\begin{align*}
%	\bigvee x_i \in f^{-1}(1)&\iff f( \bigvee x_i) = 1 \\ 
%	&\iff \bigvee f(x_i) = 1 \\
%	&\iff \exists i \in f^{-1}(0) : x_i \in f^{-1}(1)
%\end{align*}
%
%To check that $f_F$ is order preserving, consider that the only type of map that contradicts the condition is $ 1 =f_F(a) \leq f_F(b) = 0$, so we only need to check that for any $a \in F$ and $b \in P$ with $a \leq b$ it holds that $b \in F$. But this is exactly the condition of being upwards closed. 
%
%This shows that $\phi'$ is well defined and has an inverse. 
%
%Now to conclude the proof we check that the ultrafilter condition is equivalent to a complement preserving point of a boolean algebra $F \rightleftarrows \mathbb{2}$.
%
%Having proven the equivalences for filter conditions, we note that for filters, the condition of being an ultrafilter of a set $S$ means that for $A, B \subset S$ it holds that $A \in F \iff \forall B \in F: A \cap B \neq \emptyset $. 
%
%
% Now consider that $F \subset \mathcal{P}(A)$ as a poset, where the empty set is an initial object, but $S \in F$ implies $A - S \notin F$ due to the ultrafilter condition. But this just means that $f_F(\neg S) = \neg f_F(S)$. 
% 
% For the converse case, the forwards implication of the ultrafilter condition is given by the fact that $F$ is a sub meet-semilattice and $\emptyset \notin F$. 
% 
% 
% 
% 
%% Moreover, it suffices to prove the following claim: for all $B \in F$ it holds that  $B \cap A \neq \emptyset \iff B \cap A \in F$ for all $B \in F$, as this claim  implies the backwards condition of the ultrafilter lemma.
%% 
%% 
%% Consider that the previous lemma implies that $\emptyset \notin F$, from which "$\impliedby$" follows immediately.
%% 
%%\textcolor{blue}{TODO} show $B \cap A \neq \emptyset \implies B \cap A \in F$. 
%
%For the backwards implication, if we assume $a \in I$ then we have $a \land \neg a = \emptyset $, which is a contradiction to the fact that for all $ B \in F$ it holds that $ A \cap B \neq \emptyset$. Therefore $A \in F$. \end{proof}


The unit and counit maps then follow the exact same logic as the leading example, and in particular, $\mathbb{2}$ is also initial in $\Bool$ so that by the same logic, our $\tau$ is equal to the set identity $1_{\mathbb{2}}$. 

\textbf{Put this in the right place}:

The notion of a point map of a lattice homomorphism $F \stackrel{f}{\rightleftarrows} \mathbb{2}$ is asking ourselves which subsets of our lattice is compatible with our desired structure restrictions. Since taking preimages of $f$ must split our lattice $F$ into two parts, thus effectively, 

\section{Rings and Affine schemes}

We now turn to an example that any graduate Algebra student has encountered, the duality between the category of rings and the category of affine schemes. We will use the more general category of commutative $R$-algebras, which we denote $\CAlg$. Notice that if $R = \mathbb{Z}$, then $\CAlgZ = \Ring$. 

In this example we already know from commutative algebra the adjunction that gives a dual equivalence, and we can easily show that one of these adjoint functors is isomorphic to the $\Hom$ functor. For the other adjoint, it is not so easy to see that the two functors are isomorphic directly, though we can easily conclude from uniqueness of the adjoint.  We will nevertheless try to give some intuition about this isomorphism.

Firstly, we discuss the adjunction that one might learn in Algebra:

\[\begin{tikzcd}
	{\CAlg^{\op}} & \bot & {\Aff}
	\arrow["X(-)" , bend left, from=1-1, to=1-3]
	\arrow["\mathcal{O}(-)", bend left, from=1-3, to=1-1]
\end{tikzcd}\]

Recall that an affine scheme $X \in \Aff$ is defined as a representable functor in the functor category $\Fun(\CAlg, \Set)$ (\emph{to avoid confusion, we use the notation $\Nat_{\mathcal{A}}^{\mathcal{B}}(-,-)$, or more often when the context is clear simply $\Nat(-,-)$, to refer to the set of morphisms, or natural transformations, in an arbitrary functor category $\Fun(\mathcal{A},\mathcal{B})$.})

Now we see that there is a natural choice for our adjunction given by sending a representable object to its functor, and sending that representable functor to its object. In other words we have $X: \CAlg^{\op} \to \Aff$ that sends $A \to X_A = \CAlg(A, -)$ and $\mathcal{O}: \CAlg \to \Set$, which sends $X(-) = \CAlg(\mathcal{O}(X), -)$ to $\mathcal{O}(X)$, its representable object.

The equivalence is clear, since by construction  our unit and counit are  isomorphisms, i.e. $X_{\mathcal{O}(X)} =X$ and $\mathcal{O}(X_A) = A$. 


Now on the one hand,  the Yoneda lemma shows us that   $\Aff(X, \mathbb{A}^1) \cong \mathbb{A}^1(\mathcal{O}(X)) = \CAlg(R[x], \mathcal{O}(X)) \cong [\mathcal{O}(X)]$. The final set isomorphism is due to the fact that $R[x]$ is a free commutative $R$-algebra on one free generator. Now we see that for any $X_A \in \Aff$ it holds that $[\mathcal{O}(X_A)] = \Aff(X_A, \mathbb{A}^1)$, and as such we have a natural choice for a schizophrenic object $(R[x], \tau, \mathbb{A}^1 )$ (\textcolor{red}{TODO} compute $\tau$ if necessary). 


We may use the same argument to show why $\CAlg(A, R[x])$ lifts to the category of affine schemes, and that it is isomorphic to $X(A) = \CAlg(A, -)$, however it is not obvious how to think about $\mathbb{A}^1$ as a free functor on one free generator, as elements of affine schemes are not given in the same way as they are for $R[x]$, where the element $X$ is clearly our free generator. 

We may however still use Yoneda to see that $\CAlg(A, R[x]) \cong X_A(R[x]) = \Aff(\mathbb{A}^1, X_A) \cong [X_A]$, and as such our intuition thus must come from the fact that $\mathbb{A}^1$ is our free affine scheme on one free generator (\textcolor{blue}{TODO} why?) and  thus by the above equality we can think of elements of our affine scheme  as global functions from $A$ to $R[x]$.

\[\begin{tikzcd}
	{X_A} && TA && {\mathbb{A}^1} \\
	\\
	{[X_A]} && {\CAlg(A, R[x])} && {[\mathbb{A}^1]} \\
	\\
	A && {SX_A} && {R[x]} \\
	\\
	{[A]} && {\Aff(X_A, \mathbb{A}^1)} && {[R[x]]}
	\arrow["{1_{X_A}}", from=1-1, to=1-3]
	\arrow["V"', from=1-1, to=3-1]
	\arrow["{e_{A,x}}", from=1-3, to=1-5]
	\arrow["V"', from=1-3, to=3-3]
	\arrow["V", from=1-5, to=3-5]
	\arrow["\cong"', from=3-1, to=3-3]
	\arrow["{\tau \varphi_{A, x}}"', from=3-3, to=3-5]
	\arrow["{1_A}", from=5-1, to=5-3]
	\arrow["U"', from=5-1, to=7-1]
	\arrow["{d_{X_A,y}}", from=5-3, to=5-5]
	\arrow["U"', from=5-3, to=7-3]
	\arrow["U", from=5-5, to=7-5]
	\arrow["\cong"', from=7-1, to=7-3]
	\arrow["{\sigma\psi_{X_A,y}}"', from=7-3, to=7-5]
\end{tikzcd}\]

Since we have a canonical choice $[X_A] \cong \CAlg(A, R[x])$ and $[A] \cong \Aff(X_A,\mathbb{A}^1)$ of set isomorphisms, fully faithfulness of the Yoneda embeddings $V$ and $U$ ensures that any morphism of lifts into $X_A$ and $A$ respectively are unique, and as such the lifts $(X_A \stackrel{e_{A,x}}{\to}\mathbb{A}^1)_{x \in [A]}$ and $(A \stackrel{d_{X_A,y}}{\to} R[x])_{y \in [X_A]}$ are initial.
%
%Now we shall try to understand the unit and counits.

\section{Gelfand Duality}
In order to describe the following duality, some context is in order. In setting up a more general dual adjunction, whose restriction to an equivalence later defines the \emph{Gelfand duality}, we must set the scene, and in doing so we first define the following category, \emph{Kelley spaces}.
\subsection{Kelley Spaces}
Of primary importance to a Kelley space is the notion of a compactly generated topological space.
\\
\begin{definition}[$k$-continuous]
	A function $f: X \to Y$ of underlying sets of a topological space is said to be \textbf{$k$-continuous} if for all compact $K$ and continuous functions $t: K \to X$ the composition $f \circ t$ is continuous. 
\end{definition}
\
\begin{definition}[k-space]
A topological space $X$ is said to be a \textbf{$k$-space}, or a \textbf{compactly generated topological space}, if for all spaces $Y$ and underlying-set functions $f: X \to Y$, it holds that $f$ is continuous if and only if $f$ is $k$-continuous.	
\end{definition}
\



That is to say a space is compactly generated if all its continuous functions are continuous on compact subspaces. Note that the domain in the definition of arbitrary compact space $Y$ could be restricted to compact subsets $K \subset X$ since images of compact spaces by continuous maps $f(Y) \subset X$ are homeomorphic to compact subspaces  $K \subset X$, so in particular, $f$ factors through the inclusion $\iota: K \hookrightarrow X$. 

For the following we will assume that $k$-spaces are additionally Hausdorff, and refer to ${\kSp}$ as the category of compactly generated Hausdorff spaces, whose morphisms are the continuous functions between them, making it a full subcategory of $\Top$, and in particular, of $\Haus \subset \Top$. 

If given a topological space which may or may not be a $k$-space, we may force the compactly generated condition on it through a process which we call the \emph{Kelleyfication} of a topological space.

That is, given the set of inclusions $(K \stackrel{t_i}{\hookrightarrow} X)_{i \in I}$ of  compact subspaces $K\subset X$, we give $X$ the finest topology making all $t_i$ continuous. 

That is, given an underlying-set function $X \stackrel{f}{\to} Y$, we give $X$ the topology such that $f$ is continuous if and only if $f \circ t_i$ is continuous. 

But this is just the universal property of the colimit applied to topological spaces (and their full subcategories): for any set of continuous functions  $K_i \stackrel{\varphi_i}{\to}Y$ into some topological space $Y$ that satisfy commutativity (i.e., if there is a continuous map $K_i \stackrel{h}{\to} K_j $ for some $i, j \in I$, then $\varphi_i = \varphi_j \circ h$), then there exists a unique continuous map $X \stackrel{f}{\to}Y$ such that $\varphi = f \circ t_i$. 

This is reflected by commutativity of the following diagram:

\[\begin{tikzcd}
	& Y \\
	& X \\
	{K_i} && {K_j}
	\arrow["f", dashed, from=2-2, to=1-2]
	\arrow["{f \circ t_i}", bend left,  from=3-1, to=1-2]
	\arrow["{t_i}", from=3-1, to=2-2]
	\arrow["h"', from=3-1, to=3-3]
	\arrow["{f \circ t_j}"', bend right, from=3-3,  to=1-2]
	\arrow["{t_j}"', from=3-3, to=2-2]
\end{tikzcd}\]
That is if $f \circ t_i$ is continuous, preimages of open sets $V \subset Y$ must be open under composition, in other words $t_i^{-1}(f^{-1}(V)) \subset K_i$ is open. 

So $f \circ t_i$ induces a continuous $f: X \to Y$ in the following way: given a topological space with underlying set  $X$, the colimit out of compact Hausdorff subspaces will be its \emph{Kelleyfication}, which is necessarily a refinement of the topology of $X$, since continuous maps $X \stackrel{f}{\to}Y$ necessarily satisfy commutativity of the above diagram, so we want to add opens to $X$ which satisfy commutativity for arbitrary function $X \stackrel{f}{\to}Y$. 

%(\textcolor{red}{TODO}: make this better, why exactly is this universal property equivalent to the first statement ($f$ continuous iff $f \circ t_i$ continuous)?)


That is, if $X$ is not already Kelleyfied, we add opens $U = f^{-1}(V)$, for functions $f$ such that  $t_i^{-1}(U) \subset K$ is open  but $U \subset X$ is not. This is the universality condition, since commutativity must be satisfied for arbitrary function $f$. But this is our original statement: we want a topology on $X$ such that $f$ is continuous if and only if $f \circ t_i$ is continuous. 

Thus we can understand a $k$-space as a colimit of compact Hausdorff spaces, or specifically, $X \in \kSp$ if and only if $X = \colim_{K \subset X \text{ compact}}K$.

\textcolor{red}{Question}: are compact subspaces in $X$ compact if and only if they are compact in Kelley($X$)? No.

We may otherwise view the \emph{Kelleyfication} as the left adjoint $k(-)$ to the forgetful functor. As $\kSp$ is a full subcategory of $\Haus$, this puts us in the setting of a coreflective subcategory of $\Haus \subset \Top$. 

Another way to think about $\kSp$ is as the coreflective hull of $\kHaus$ in $\Haus$. That is to say, the forgetful functor $\kHaus \hookrightarrow \Haus$ does not have a left adjoint (\textcolor{red}{TODO}: Why? Does it have something to do with that $\kHaus$ is not closed under colimits), however we may take the intersection of all coreflective subcategories of $\Haus$ which contain $\kHaus$, and this will precisely be the category generated under colimits of $\kHaus$  (\textcolor{red}{TODO}: make this precise.)

Note that products are given as the Kelleyfications of topological  products, and furthermore, that for all locally compact spaces $X$, we have $X \times_k Y = X \times Y$ for all $Y \in \kSp$. (\textcolor{blue}{TODO} show this).

It is important that our category admits function spaces, or in other words, that we are in a closed category (i.e. all exponentiables exist). In the following we will show why $\kSp$ is a closed category. This will indeed give us intuition of why we have even defined $\kSp$ the way we have, as $\Haus $ is not a closed category.  

What we want is the following adjunction:
\begin{align*}
	\Haus(A \times_{\Haus}C, B) \cong \Haus(C, B^A)
\end{align*}
For all $A, B, C \in \Ob(\Haus)$. However this does not hold in general. 

However in $\kSp$ function spaces are given as Kelleyfications of the set $\Top(X,Y)$ with the compact-open topology. (\textcolor{blue}{TODO} why?).
\textcolor{red}{ToDO} Show that this means that $\kSp$ is cartesian closed concrete category which admits function spaces.

\textcolor{red}{ToDO} Show that $\kSp$ is a monotopological category

Now let us define the category $\kAlg$ of complex $k$-algebras, whose objects are $\mathbb{C}$-algebras endowed with a Kelley topology, whose morphisms are the continuous algebra homomorphisms, and whose algebra operations are continuous with respect to $k$-products. 

Now since $\kSp$ is a monotopological category which admits function spaces, we are in the setting of a category which admits an internal hom-functor. 

To put ourselves completely in the situation of \textbf{4.c} we notice that $\kAlg \subset \kSp$ is a full subcategory, so that we have an underlying functor $U: \kAlg \to \kSp$, which satisfies the conditions of $4.c$ in the following. As $\mathbb{C}$ is a colimit of its compact Hausdorff subspaces which is moreover a $\mathbb{C}$-algebra, we see that $\mathbb{C} \in \kAlg$, and we call it $\mathbb{C}_a$,  so that $U(\mathbb{C}_a) = \mathbb{C}_s$, by analogous notation, and clearly $\tau = 1_{\mathbb{C}}. $

Since  we are in a monotopological category, then we know that $\kAlg(A, \mathbb{C}_a) \to [\mathbb{C}_s]$
lifts initially to a $\kSp$-morphism, which we will denote $\Hom_k(A, \mathbb{C}_a) \to \mathbb{C}_s$.

Now all we need to see is that $\kSp(X,\mathbb{C}_s) \to [\mathbb{C}_a]$ lifts along $U$ functorially to a $\kAlg$-morphism $C_k(X, \mathbb{C}_s) \to \mathbb{C}_a$, which we must show directly. (\textcolor{red}{TODO}: show this)

As such we have an adjunction 
\begin{align*}
	C: \kSp & \to \kAlg &  S: \kAlg & \to \kSp\\
	X &\mapsto C_k(X, \mathbb{C}_s) & A &\mapsto \Hom_k(A, \mathbb{C}_a)
\end{align*}

which is not natural.

This is called the \emph{generalized Gelfand-Naimark Duality}. 

In the following we will want to restrict $\kSp$ and $\kAlg$ to their full subcategories under which the above adjunction is a dual equivalence. For that we will introduce the category of $C^*$-algebras.
\subsection{$C^*$-algebras}
Consider the functor $C^*: \Top \to \Set$, which sends  $X \mapsto C^*(X) = \Top_{bd.}(X, \mathbb{C})$ for all $X \in \Top$. 
From pointwise operations $C^*(X)$ can be seen to be an associative, commutative, unital $\mathbb{C}$-algebra. Pointwise conjugation gives us the operation $f \mapsto f^*$, so that $C^*(X)$ is an involutive algebra, and the supremum norm $\lVert f \lVert = \sup_{x \in X}\lvert f(x)\lvert $ turns $C^*(X)$ into a normed algebra satisfying $\lVert f \lVert^2 = \lVert f \cdot f^* \lVert$. If we consider the involution preserving unital $\mathbb{C}$-algebra homomorphisms, we obtain a category $C^*$.

Note that $C^* \subset \kAlg$ is a full subcategory.

 We could view $C^*$ as a concrete category via the usual underlying-set functor, however we might find it more interesting in this case to consider a different faithful functor, namely the functor $\bigcirc: C^* \to \Set$ which sends any $C^*$-algebra $A$ to it's unit ball $\{ a \in A \  \lvert  \ \lVert a \lVert \leq 1\}$ and each morphism $f$  to its restriction $f_\bigcirc$ to the unit ball of its domain.
 
 \subsubsection{Gelfand-Naimark Duality}
 
 For any compact Hausdorff space $X$, we see that the $C^*$-algebra $C^*(X)$ and the function $k$-algebra $C(X) = C_k(X, \mathbb{C}_s)$ coincide algebraically, since continuous functions over compact spaces are bounded in $\mathbb{C}$ and since $X \times_k X = X \times X$ (\textcolor{red}{TODO}: why  is $C^*(X) \subset C(X)$. Also make a comment about why $X \in \kSp$ for $X$ locally compact ). Moreover the topology of $C^*(X)$ is the compact open topology, since $X$ is compact, and therefore they also coincide topologically. That means we have can restrict $C$ to a functor $C: \kHaus \to C^*$.
 
 Can we restrict $S$ accordingly? For every $C^*$-algebra $A$, the space $S(A) = \Hom_k(A, \mathbb{C}_a)$ is compact, and its topology is that of pointwise convergence, due to basic results about the topology of function spaces. (\textcolor{red}{TODO} cite basic results). We call $S: C^* \to \kHaus$ the \emph{spectrum}- functor, and it follows that the generalized Gelfand-Naimark adjunction restricts to a dual adjunction
 
 \begin{align*}
	C: \kHaus & \to C^* &  S: C^* & \to \kHaus\\
	X &\mapsto C_k(X, \mathbb{C}_s) & A &\mapsto \Hom_k(A, \mathbb{C}_a).
\end{align*}
 
 However this still doesn't give us the equivalence, as $\mathbb{C}_s \notin \kHaus$. Remember that we want to consider $C^*$ as a concrete category via the functor $\bigcirc$, so that for any compact Hausdorff space $X$, one has $\bigcirc C(X) = \kHaus(X, D)$ where $D = \{ c \in \mathbb{C}_s \ \lvert \ \lVert c \lVert \leq 1 \}. $
 
 Thus we can conclude that there is a natural dual adjunction between concrete categories $(\kHaus, U)$ and $(C^*, \bigcirc)$ with schizophrenic object $(D, 1_D, \mathbb{C}_a)$. 
 
 \section{Galois theory}
 Let us dive into another familiar example called the \emph{Galois correspondence}. The example we want to discuss is actually a generalization of the usual Galois correspondence, given by the following adjunction between the directed poset (\textcolor{red}{TODO} defined this) of subfield extensions of the finite field extension $k \hookrightarrow M\hookrightarrow L$ and the codirected poset (\textcolor{red}{TODO} figure out which is directed and which codirected, details) of the corresponding subgroups $H$ of the Galois group $\Gal(L/k)$:
 
 \[\begin{tikzcd}
	{\{L|M|k\} } & \bot & {\{H \leq \Gal(L/k)\}}
	\arrow["{\Gal(L/-)}", shift left=3, bend left, from=1-1, to=1-3]
	\arrow["{L^{(-)}}", bend left, from=1-3, to=1-1]
\end{tikzcd}\]

We know ${\{L|M|k\} }$ is directed since $L|k$ is a terminal object of the poset, and similarly ${\{H \leq \Gal(L/k)\}}$ is codirected over ${\{L|M|k\} }^{\op}$ whose terminal object is $k$, corresponding to the full subgroup $\Gal(L/k)$. 

However the schizophrenic object doesn't live in the above adjunction, for that we consider a more general adjunction, which we can derive from the above adjunction in the following way. 

The first important consideration is that we want our left (and by consequence, our right) category to include the separable closure of $k$, which we denote $k^s$, (\textcolor{red}{TODO} clarify/figure out why separable and why/why not algebraic closure) and naturally, all its intermediate finite field extensions. To be sufficiently general we consider the category $\CAlgk$  of commutative unital $k$-algebras with $k$-algebra homomorphisms. 

The second important consideration is that the above left adjoint functor induces an adjunction with respect to the quotient groups of Galois groups under normal subgroups, or more generally the set of cosets of each subgroup $H$ in $\Gal(k^s/k) =: G$.

Explicitly this is a map $L|k \mapsto \CAlgk(L, k^s)$. That is, $\CAlgk(L, k^s) \cong \Gal(k^s/k)/\Gal(k^s/L)$ (\textcolor{red}{TODO}: Show this (make sure its even true), and also think about how $\Gal(k^s/L)$ is distinguished from $\CAlgk(L, k^s)$, intuitively speaking). Notice that $\CAlgk(L,k^s)$ is finite and there exists a natural $G$-action on it, since homomorphisms into the closure are given by maps which permute the roots of the minimal polynomials of the elements which generate $L|k$ (\textcolor{red}{TODO} refer to literature). Furthermore, this extends to a natural $G$-action on $k^s$, as the $G$-action on $k^s$ is fully determined by permutation of roots not lying in $k$, or elements of $\CAlgk(k,k^s)$. (\textcolor{red}{TODO} explain more in detail). 

Therefore $L|k \mapsto \CAlgk(L, k^s)$ lifts to the category $\GSet$, consisting of sets with a $G$-actionand $G$-equivariant homomorphisms between them, where $k^s$ also lives.

(\textcolor{blue}{TODO} show that $\GSet(H,k^s)$ lifts to $\CAlgk$, I still don't know, but idea is that  a $G$-equivariant map of some set should determine some Galois subfield extension since its "part of that action" being preserved by a map. Determine how this set of maps should look like, and what exactly is the datum of the $G$-set as well as datum of the $\CAlgk$; How big or small is this hom-set? Is there a function for every element of the field, or is it enough to realize the datum in some sense as that of the generating elements of the extension? ). 

Remember that if our subfield extension $M|k$ is $Galois$, that is,  it is a \emph{normal} (separable, algebraic) field extension, i.e. $M$ contains the roots of every polynomial over itself, then the fundamental theorem of Galois theory states that $\Gal(k^s/M) \unlhd \Gal(k^s/k)$. This gives us a Galois group over a different functor, namely the functor $\Gal(-/k)$ which yields a profinite system $({k^s|M|k} , Gal(M/k))$.

Consider that $k^s|k = \colim_{\{k^s|L|k\}}L|k$, then corespectively we have \begin{align*}
	\Gal(k^s/k)=& \CAlgk(\colim_{\{k^s|L|k\}}L, k^s)\\ =& \lim_{\{k^s|L|k\}}\CAlgk(L,k^s)\\ =& \lim_{\{k^s|L|k\}}\Gal(k^s/k)/\Gal(k^s/L)
\end{align*}

so $\Gal(k^s/k)$ can be realized as a profinite set, which is necessarily uncountable. 
All together we have schizophrenic object $(k^s,1_{k^s},k^s)$ which induces the following dual adjunction


 \[\begin{tikzcd}
	\CAlgk & \bot & \GSet
	\arrow["{\CAlgk(-, k^s)}", shift left=3, bend left, from=1-1, to=1-3]
	\arrow["{\GSet(-, k^s)}", bend left, from=1-3, to=1-1]
\end{tikzcd}\]

%$\{k^s|L|k\} $ ${\{H \leq \Gal(L/k)\}}$
This adjunction restricts to the categories $\ket$ of finite dimensional $k$-algebras which are isomorphic to a finite product of separable field extensions with the usual $k$-algebra homomorphisms between them, and the category $\GFSet$ of finite $G$-sets and the usual $G$-equivariant maps between them. 

 \[\begin{tikzcd}
	\ket^{\op} & \stackrel{\cong}{\bot} & \GFSet
	\arrow["{\CAlgk(-, k^s)}", shift left=3, bend left, from=1-1, to=1-3]
	\arrow["{\GSet(-, k^s)}", bend left, from=1-3, to=1-1]
\end{tikzcd}\]

To see this we just check that $L|k = \GSet(\CAlgk(L,k^s),k^s)$ and we check that $H = \CAlgk(\GSet(H,k^s),k^s))$. (\textcolor{red}{TODO}: prove this or refer to usual essential surj + fully faithful proof from literature)

% (\textcolor{red}{TODO}: finish this part, if necessary, but the basic structure is the following: show that $\kSp$ is a monotopological category, show that monotopological categories admit function spaces... if theres an easier way to do it we may not need to define lifts and topological categories, so do that later)
 
 

%
%Our adjunction has the unit
%\begin{align*}
%	\epsilon:& 1_{\Frm} \to ST\\
%	&A \mapsto STA\\
%	&A \mapsto \Top(\Frm(A, \mathbb{2}), \mathbb{S})
%\end{align*}
%so that 
%\begin{align*}
%	\epsilon_A:& A \to \Top(\Frm(A, \mathbb{2}), \mathbb{S})
%\end{align*}
%where 
%\begin{align*}
%	\epsilon_A(x):& \Frm(A, \mathbb{2}) \to \mathbb{S}\\
%	&p \mapsto p(x)
%\end{align*}
%Notice that the topology on $\Frm(A, \mathbf{2})$ is the initial topology making all $(\epsilon_A(x))_{x \in A}$ continuous.
%
%When passing to $U$ and $V$, we have
%\begin{align*}
%	[\epsilon_{\t A}]:& [\t A] \to [\Top(\Frm(\t A, \mathbb{2}), \mathbb{S})] = \Top(\Frm(\t A, \mathbb{2}), \mathbb{S})\\
%	&(\t x: 1 \to x) \mapsto (p \to p(x))
%\end{align*}
%which is equal to
%\begin{align}
%	[\epsilon_{\mathbb{2}}]:& [\mathbb{2}] \to [\Top(\Frm(\mathbb{2}, \mathbb{2}), \mathbb{S})] = \Top(\Frm(\mathbb{2}, \mathbb{2}), \mathbb{S})\\
%	&(\t x: 1 \to x) \mapsto (1_{\mathbb{2}} \to 1_{\mathbb{2}}(x))
%\end{align}
%So we have 
%\begin{align*}
%	[\epsilon_{\mathbb{2}}](\t x):& \Frm(\mathbb{2},\mathbb{2}) \to \mathbb{S}\\
%	&1_{\mathbb{2}} \mapsto 1_{\mathbb{2}}(x)
%\end{align*}
%and now
%\begin{align*}
%	[[\epsilon_{\mathbb{2}}](\t x)]:& [\Frm(\mathbb{2},\mathbb{2})] \to [\mathbb{S}]\\
%	&(1 \to 1_{\mathbb{2}}) \mapsto (\{pt\} \to1_{\mathbb{2}}(x))
%\end{align*}
%so that 
%\begin{align*}
%	[[\epsilon_{\mathbb{2}}](\t x)](1_{\mathbb{2}}) = (\{pt\} \to 1_{\mathbb{2}}(x)).
%\end{align*}
%Now we can see that the map \begin{align*}
%	\tau: [\mathbb{2}] &\to [\mathbb{S}]\\
%	(\t x: 1 \to x) &\mapsto (\{pt\} \to 1_{\mathbb{2}}(x))
%\end{align*}
%is the identity in $\Set$, as the underlying set on both sides is $[\mathbb{2}] = [\mathbb{S}] = \{0,1\}$ so that we are looking at the set map \begin{align*}
%	\{0,1\} &\to \{0,1\}\\
%	x &\mapsto 1_{\{0,1\}}(x)= x
%\end{align*}

\section{Questions}
\begin{enumerate}
	\item How do I approach the preliminaries part? Just a list of definitions? Or should I write some kind of narrating text? Should I move it to the appendix?
	\item Ask Georg how to differentiate between $k$-Alg in terms of kelleyfied complex algebras and $k$-Alg in terms of algebras over a field $k$. We know that $\kAlg \subset \CAlgk$ as subcategory (not full), should this be reflected in the notation?
	\item How do I introduce subjects and where do i draw the line for assumed knowledge and not.
	\item $\mathbb{A}^1$ is a free object on one free generator in $\Aff$
	\item Why do we know that Kelleyfied products and kelleyfied function spaces with the compact open topology are the categorical products and exponentials in $\kSp$
	\item $\GSet(H, k^s)$ lifts to $\CAlgk$ (possible for later)
	\item $X \times_k Y = X \times Y$ for $X$ locally compact, $Y \in \kSp$
	\item ask Georg if pointwise order preserving evaluation morphism into $\mathbb{2}$ preserves finite limits and colimits by some pointwise argument in the presheaf category
	\item why  is $C^*(X) \subset C(X)$
	\item $B \cap A \neq \emptyset $ for all $B \in F$ (filter) implies $B \cap A \in F$
	\item Why is $\kSp$ a monotopological category? Why does it have function spaces?
	\item can same schizophrenic object induce different dualities when looked at different categories? In what sense am I supposed to think of sub-dualities? (I guess so yeah)
	\item Show that $C_k(X, \mathbb{C}) = C(X, \mathbb{C})$ if $X$ is locally compact.
\end{enumerate}
%\begin{enumerate}
%	\item Is the above actually the case, i.e., am I actually supposed to look at the underlying sets in $\Set$ to understanding that $\tau$ is the identity?
%	\subitem am I to understand $p(x) \in \mathbb{S}$ as "translating" from $\Frm$ to $\Top$ via $\Set$-elements?
%	\item Is it actually true the remark about a free generator on one free object fully determining the set $\HomA(A_0,A) = A$ by where it sends the free generator to?
%	\item In the commutative triangle we have the upper part of the lift being an isomorphism by the adjunction, so that a set of $V$ initial lifts $(TA \to B)_{x \in A}$ should be fully determined by existence of the maps $\tau \circ (\varphi_{A, x})$, correct? What I mean is, to show that a schizophrenic object exists, or to check the axioms for an S.O., I need only to show that $\tau \circ (\varphi_{A, x})$ is well defined for all $A \in \mathcal{A}$ and $x \in A$, and behaves as expected?
%	\item In general,  a schizophrenic object should be fully determined by the adjunction and the representing objects. I mean to say if I understand the adjunction and know what the representing objects are, I have everything I need to check my hypothesis about the schizophrenic object, i.e., a hypothesis about a schizophrenic object leads to simply checking the adjunction and representable objects.
%	\item Why are the representable objects \emph{necessarily} free objects on one free generator, and what is meant by free here? Does the necessary part come from the fact that only such an object could lead to a schizophrenic object, or is it for some reason necessary a priori?
%\end{enumerate}
\end{document}